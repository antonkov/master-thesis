\documentclass[t,13pt,graphics=pdflatex,xcolor=table,aspectratio=43]{beamer}

%% Misc packaging
\usepackage[utf8]{inputenc}
\usepackage[T2A]{fontenc}
\usepackage[english,main=russian]{babel}
\usepackage{amssymb,amsmath}
\usepackage{graphicx}
\usepackage{multirow}
\usepackage{blkarray}
\usepackage{tikz}
\usepackage[table]{xcolor}
\newcommand{\inputTikZ}[1]{\input{#1.tikz}}
\usepackage{subcaption}

%% Theorems
\newtheorem{theoremRus}{Теорема}[section]
\newtheorem{lemmaRus}[theorem]{Лемма}
\newtheorem{corollaryRus}[theorem]{Следствие}
\newtheorem{exampleRus}{Пример}[section]
\newtheorem{propRus}{Свойство}[section]
\newtheorem{definitionRus}{Определение}[section]

\setbeamertemplate{blocks}[rounded][shadow=false]
%% Добавляйте новые пути с картинками в эту команду.
%% Будет примерно так:
%% \graphicspath{{itmo-logo/}{pic/}}
\graphicspath{{itmo-logo/}}

%% ИТМОшные цвета
\definecolor{itmoblue}{RGB}{25,70,186}
\definecolor{itmored}{RGB}{236,11,67}

%% Установка основных цветов в тексте
\setbeamercolor{normal text}{fg=itmoblue}
\setbeamercolor{alerted text}{fg=itmored}
\setbeamercolor{frametitle}{fg=white}
\setbeamercolor{palette quaternary}{fg=white,bg=itmoblue}
\setbeamercolor{frametitle}{bg=itmoblue}

%% Дизайн заголовка обычного слайда
\setbeamertemplate{frametitle}{
\begin{beamercolorbox}[ht=0.15\textheight,wd=\paperwidth]{frametitle}
    \begin{minipage}[b][0.15\textheight][t]{0.35\paperwidth}
    \includegraphics[width=\textwidth]{itmo_horiz_blue_rus.png}
    \end{minipage}~\begin{minipage}[b][0.15\textheight][c]{0.6\paperwidth}
    \begin{flushright}\normalsize\insertframetitle\end{flushright}
    \end{minipage}
\end{beamercolorbox}
}
\beamertemplatenavigationsymbolsempty

%% Хрень всякая
\newcommand{\alertAt}[2]{\alt<#1>{\alert{#2}}{#2}}

\begin{document}

%% Идиотский способ сделать титульную страницу
\begingroup
\setbeamercolor{background canvas}{bg=itmoblue}
\setbeamercolor{normal text}{fg=white}
\begin{frame}[plain]
\color{white}
\centering\bfseries
\includegraphics[width=0.35\textwidth]{itmo_small_blue_rus.png}

{\Large Ковшаров Антон Павлович \par}

\vspace{0pt plus 0.3filll}

{\large {<<}Исследование зависимости вероятности ошибки на блок от спектра 
            графа Таннера для МППЧ-кодов{>>}}

\vspace{0pt plus 0.3filll}

{\small Научный руководитель:\par
 докт. техн. наук, проф. Кудряшов Борис Давидович}

\vspace{0pt plus 0.3filll}

{\small Кафедра КТ \hfill } 

\vspace{0pt plus 1filll}
\end{frame}
\endgroup

\begin{frame}{Доклад \\ VI Конгресс молодых ученых}
\begin{block}{Оглавление}
\begin{enumerate}
    \item Цели работы
    \item Общие понятия 
    \item Алгоритм подсчета спектра
    \item Экспериментальные исследования
    \item Выводы
\end{enumerate}
\end{block}
\end{frame}

\begin{frame}{Цели работы}
\begin{itemize}
  \item Проверить гипотезу зависимости эффективности итеративного декодирования от числа коротких циклов
  \item Разработать алгоритм подсчета спектра (числа циклов определенной длины) графа Таннера
\end{itemize}
\end{frame}

\begin{frame}{Общие понятия \\ Линейный код}
  Линейный $(n, k)$ код \\
  $G$ -- порождающая матрица \\
  $H$ -- проверочная матрица \\
  $G \cdot H^T = 0$ -- проверка на четность \\ МППЧ-код -- код с малой плотностью проверок на четность (мало единиц в $H$) \\ 

\begin{exampleblock}{Пример}
\begin{align*}
    G&=
    \begin{pmatrix}
        1&0&0&   1\\
        0&1&1& 0\\
    \end{pmatrix} \text{  } \\
    H&=
    \begin{pmatrix}
        0&1&1&0 \\
        1&0&0&1 \\
    \end{pmatrix} \text{  } 
\end{align*}
\end{exampleblock}
\end{frame}
\newcommand\colorBox[2]{\setlength{\fboxsep}{2pt}\colorbox{#1!10}{#2}}
\begin{frame}{Общие понятия \\ Графическое представление}
\begin{itemize}
  \item Символьный узел -- кодовый символ \tikz{\filldraw[thick,draw=black!50,fill=red!10] circle(1ex);}
  \item Проверочный узел -- проверка на четность \tikz{\filldraw[thick,draw=black!50,fill=green!10] rectangle(2ex,2ex);}
  \item Линия между если символ состоит в проверке
\end{itemize}
\begin{minipage}[c]{.3\textwidth}
  \[
    \begin{blockarray}{ccccc}
        \colorBox{red}{1} & \colorBox{red}{2} & \colorBox{red}{3} & \colorBox{red}{4} \\
        \begin{block}{(cccc)c}
            1&1&0&1&\colorBox{green}{5}\\
            1&1&1&0&\colorBox{green}{6}\\
            1&0&1&1&\colorBox{green}{7} \\
        \end{block}
    \end{blockarray}
  \]
\end{minipage}
\begin{minipage}[c]{.68\textwidth}
\begin{figure}[!h]
  \centering
  \inputTikZ{../tikz/ex_graph1}
  \label{fig1}
\end{figure}
\end{minipage}
\end{frame}

\begin{frame}{Общие понятия \\ Декодирование}
\begin{figure}[c]
  \only<1>{\inputTikZ{../tikz/decoding1}}
  \only<2>{\inputTikZ{../tikz/decoding2}}
  \only<3>{\inputTikZ{../tikz/decoding3}}
  \only<4>{\inputTikZ{../tikz/decoding4}}
\end{figure}
\end{frame}

\begin{frame}{Общие понятия \\ Цикл}
\begin{figure}[c]
  \only<1>{\inputTikZ{../tikz/cycle1}}
  \only<2>{\inputTikZ{../tikz/cycle2}}
  \only<3>{\inputTikZ{../tikz/cycle3}}
\end{figure}
\end{frame}

\begin{frame}{Алгоритм подсчета спектра}
  1!
\end{frame}

\begin{frame}{Алгоритм подсчета спектра}
  2!
\end{frame}

\begin{frame}{Алгоритм подсчета спектра}
  3!
\end{frame}

\begin{frame}{Алгоритм подсчета спектра}
  4!
\end{frame}

\newcommand{\plotstandard}[1]{\centerline{\includegraphics[height=3in]{#1}}}
\newcommand{\plotsmall}[1]{\includegraphics[height=1.3in]{#1}}

\begin{frame}{Экспериментальные исследования \\ Ричардсон-Урбанке 4x8 576}
  \plotstandard{../images/r4_576.pdf}
\end{frame}

\begin{frame}{Экспериментальные исследования \\ Ричардсон-Урбанке 4x8 2304}
  \plotstandard{../images/r4_2304.pdf}
\end{frame}

\begin{frame}{Экспериментальные исследования}
\begin{table}[!t]
  \begin{tabular}{ccc}
    \plotsmall{../images/g3_576.pdf}&
    \plotsmall{../images/g3_2304.pdf}&
    \plotsmall{../images/q3_576.pdf}\\
    \plotsmall{../images/q3_2304.pdf}&
    \plotsmall{../images/q4_576.pdf}&
    \plotsmall{../images/g4_576.pdf}\\
  \end{tabular}
\end{table}
\end{frame}

\begin{frame}{Выводы}
  \begin{itemize}
    \item Получены экспериментальные доказательства обратной зависимости между количеством коротких циклов 
      и вероятностью ошибки на блок для различных ансамблей кодов
    \item Разработан вычислительно эффективный алгоритм подсчета спектра графа Таннера, который может
      быть использован для ускорения отбора кодов
  \end{itemize}
\end{frame}

\end{document}
%%% Local Variables:
%%% mode: latex
%%% TeX-master: t
%%% End:
