\documentclass[t,13pt,graphics=pdflatex,xcolor=table,aspectratio=43]{beamer}

%% Misc packaging
\usepackage[utf8]{inputenc}
\usepackage[T2A]{fontenc}
\usepackage[english,main=russian]{babel}
\usepackage{amssymb,amsmath}
\usepackage{graphicx}
\usepackage{multirow}
\usepackage{blkarray}
\usepackage{tikz}
\usepackage[table]{xcolor}
\newcommand{\inputTikZ}[1]{\input{#1.tikz}}
\usepackage{subcaption}

%% Theorems
\newtheorem{theoremRus}{Теорема}[section]
\newtheorem{lemmaRus}[theorem]{Лемма}
\newtheorem{corollaryRus}[theorem]{Следствие}
\newtheorem{exampleRus}{Пример}[section]
\newtheorem{propRus}{Свойство}[section]
\newtheorem{definitionRus}{Определение}[section]

\setbeamertemplate{blocks}[rounded][shadow=false]
%% Добавляйте новые пути с картинками в эту команду.
%% Будет примерно так:
%% \graphicspath{{itmo-logo/}{pic/}}
\graphicspath{{itmo-logo/}}

%% ИТМОшные цвета
\definecolor{itmoblue}{RGB}{25,70,186}
\definecolor{itmored}{RGB}{236,11,67}

%% Установка основных цветов в тексте
\setbeamercolor{normal text}{fg=itmoblue}
\setbeamercolor{alerted text}{fg=itmored}
\setbeamercolor{frametitle}{fg=white}
\setbeamercolor{palette quaternary}{fg=white,bg=itmoblue}
\setbeamercolor{frametitle}{bg=itmoblue}

%% Дизайн заголовка обычного слайда
\setbeamertemplate{frametitle}{
\begin{beamercolorbox}[ht=0.15\textheight,wd=\paperwidth]{frametitle}
    \begin{minipage}[b][0.15\textheight][t]{0.35\paperwidth}
    \includegraphics[width=\textwidth]{itmo_horiz_blue_rus.png}
    \end{minipage}~\begin{minipage}[b][0.15\textheight][c]{0.6\paperwidth}
    \begin{flushright}\normalsize\insertframetitle\end{flushright}
    \end{minipage}
\end{beamercolorbox}
}
\beamertemplatenavigationsymbolsempty

%% Хрень всякая
\newcommand{\alertAt}[2]{\alt<#1>{\alert{#2}}{#2}}

\begin{document}

%% Идиотский способ сделать титульную страницу
\begingroup
\setbeamercolor{background canvas}{bg=itmoblue}
\setbeamercolor{normal text}{fg=white}
\begin{frame}[plain]
\color{white}
\centering\bfseries
\includegraphics[width=0.35\textwidth]{itmo_small_blue_rus.png}

{\Large Ковшаров Антон Павлович \par}

\vspace{0pt plus 0.3filll}

{\large {<<}Исследование зависимости вероятности ошибки на блок от спектра 
            графа Таннера для МППЧ-кодов{>>}}

\vspace{0pt plus 0.3filll}

{\small Научный руководитель:\par
 канд. техн. наук, доцент, Буздалов Максим Викторович}

\vspace{0pt plus 0.3filll}

{\small Кафедра КТ \hfill } 

\vspace{0pt plus 1filll}
\end{frame}
\endgroup

\begin{frame}{Магистерская диссертация}
\begin{block}{Оглавление}
\begin{enumerate}
    \item Цели работы
    \item Общие понятия 
    \item Алгоритм подсчета спектра
    \item Экспериментальные исследования
    \item Выводы
\end{enumerate}
\end{block}
\end{frame}

\begin{frame}{Цели работы}
\begin{itemize}
  \item Проверить гипотезу зависимости эффективности итеративного декодирования от числа коротких циклов
  \item Разработать алгоритм подсчета спектра (числа циклов определенной длины) графа Таннера
\end{itemize}
\end{frame}

\begin{frame}{Общие понятия \\ Линейный код}
  Линейный $(n, k)$ код \\
  $G$ -- порождающая матрица \\
  $H$ -- проверочная матрица \\
  $G \cdot H^T = 0$ -- проверка на четность \\ МППЧ-код -- код с малой плотностью проверок на четность (мало единиц в $H$) \\ 

\begin{exampleblock}{Пример}
\begin{align*}
    G&=
    \begin{pmatrix}
        1&0&0&   1\\
        0&1&1& 0\\
    \end{pmatrix} \text{  } \\
    H&=
    \begin{pmatrix}
        0&1&1&0 \\
        1&0&0&1 \\
    \end{pmatrix} \text{  } 
\end{align*}
\end{exampleblock}
\end{frame}
\newcommand\colorBox[2]{\setlength{\fboxsep}{2pt}\colorbox{#1!10}{#2}}
\begin{frame}{Общие понятия \\ Графическое представление}
\begin{itemize}
  \item Символьный узел -- кодовый символ \tikz{\filldraw[thick,draw=black!50,fill=red!10] circle(1ex);}
  \item Проверочный узел -- проверка на четность \tikz{\filldraw[thick,draw=black!50,fill=green!10] rectangle(2ex,2ex);}
  \item Линия между если символ состоит в проверке
\end{itemize}
\begin{minipage}[c]{.3\textwidth}
  \[
    \begin{blockarray}{ccccc}
        \colorBox{red}{1} & \colorBox{red}{2} & \colorBox{red}{3} & \colorBox{red}{4} \\
        \begin{block}{(cccc)c}
            1&1&0&1&\colorBox{green}{5}\\
            1&1&1&0&\colorBox{green}{6}\\
            1&0&1&1&\colorBox{green}{7} \\
        \end{block}
    \end{blockarray}
  \]
\end{minipage}
\begin{minipage}[c]{.68\textwidth}
\begin{figure}[!h]
  \centering
  \inputTikZ{../tikz/ex_graph1}
  \label{fig1}
\end{figure}
\end{minipage}
\end{frame}

\begin{frame}{Общие понятия \\ Декодирование}
\begin{figure}[c]
  \only<1>{\inputTikZ{../tikz/decoding1}}
  \only<2>{\inputTikZ{../tikz/decoding2}}
  \only<3>{\inputTikZ{../tikz/decoding3}}
  \only<4>{\inputTikZ{../tikz/decoding4}}
\end{figure}
\end{frame}

\begin{frame}{Общие понятия \\ Цикл}
\begin{figure}[c]
  \only<1>{\inputTikZ{../tikz/cycle1}}
  \only<2>{\inputTikZ{../tikz/cycle2}}
  \only<3>{\inputTikZ{../tikz/cycle3}}
\end{figure}
\end{frame}

\begin{frame}{Алгоритм подсчета спектра \\ Циклы}
  \begin{minipage}{0.8\textwidth}
    \centering
    \only<1>{\input{../tikz/cycleLen1.tikz}}
    \only<2>{\input{../tikz/cycleLen2.tikz}}
    \only<3>{\input{../tikz/cycleLen3.tikz}}
    \only<4>{\input{../tikz/cycleLen4.tikz}}
    \only<5>{\begin{tikzpicture}
[
scale=1.5,transform shape,
state/.style={rectangle,draw=black!50,fill=green!10,thick,minimum size=5mm},
oper/.style={circle,draw=black!50,fill=red!10,thick, minimum size=5mm, font=\small},
arr/.style={-,auto,>=stealth},
arrBright/.style={-,auto,>=stealth,draw=red,very thick}
]
\small

% Input staff 
%\node (a0) at (0,0) [oper ] {$+$};
%\draw [arr](-1,0) to node [left,xshift=-3mm]{$x_i$} node{}(a0);
%\draw (1,0) circle (0.3mm) [fill=black!];
%\node (mp0) at (1,1) [oper,label=left:$f_0$ ] {$\times$};
%\draw [arr] (1,0) -- (mp0);
%\draw (mp0) -- (1,2.5);
%\draw [arr] (0,-2.5) -- (a0);
% Filter section

\node (s6) at (0,0) [state] {6};
\node (s5) at (2,2) [state] {5};
\node (s7) at (2,-2) [state] {7};

\node (s2) at (0,2) [oper] {2};
\node (s3) at (0,-2) [oper] {3};
\node (s1) at (2,0) [oper] {1};
\node (s4) at (4,0) [oper] {4};

\draw [arr](s4)[] to node [right]{} node{}(s5);
\draw [arr](s4)[] to node [right]{} node{}(s7);
\draw [arrBright](s2)[] to node [above]{} node{}(s5);
\draw [arrBright](s2)[] to node [right]{} node{}(s6);
\draw [arrBright](s1)[] to node [right]{} node{}(s5);
\draw [arr](s1)[] to node [above]{} node{}(s6);
\draw [arrBright](s1)[] to node [right]{} node{}(s7);
\draw [arrBright](s3)[] to node [right]{} node{}(s6);
\draw [arrBright](s3)[] to node [above]{} node{}(s7);


\end{tikzpicture}

}
    \only<6>{\input{../tikz/cycleLen6.tikz}}
    \only<7>{\input{../tikz/cycleLen7.tikz}}
    \only<8>{\input{../tikz/cycleLen8.tikz}}
    \only<9>{\input{../tikz/cycleLen1.tikz}}
  \end{minipage}\hfill
  \begin{minipage}{0.15\textwidth}
    \only<1>{4: 0}
    \only<2>{4: 1}
    \only<3>{4: 2}
    \only<4->{4: 3}
    \\
    \only<1-4>{6: 0}
    \only<5>{6: 1}
    \only<6>{6: 2}
    \only<7>{6: 3}
    \only<8->{6: 4}
    \\
    \only<1-8>{
        8: 0 \\
        10: 0 \\
        12: 0 \\
        14: 0 
    }
    \only<9->{
    8: 6 \\
    10: 12 \\
    12: 29 \\
    14: 48 
    }
    \\
    ....
  \end{minipage}
\end{frame}

\begin{frame}{Алгоритм подсчета спектра \\ Состояние, переходы}
  Состояние - направленное ребро $w_i$\\
  \begin{minipage}{0.35\textwidth}
    \centering
    \scalebox{0.7}{\begin{tikzpicture}
[
scale=1,transform shape,
state/.style={rectangle,draw=black!50,fill=white!,thick,minimum size=5mm},
oper/.style={circle,draw=black!50,fill=white!,thick, minimum size=5mm, font=\small},
arr/.style={->,auto,>=stealth}
]
\small

% Input staff 
%\node (a0) at (0,0) [oper ] {$+$};
%\draw [arr](-1,0) to node [left,xshift=-3mm]{$x_i$} node{}(a0);
%\draw (1,0) circle (0.3mm) [fill=black!];
%\node (mp0) at (1,1) [oper,label=left:$f_0$ ] {$\times$};
%\draw [arr] (1,0) -- (mp0);
%\draw (mp0) -- (1,2.5);
%\draw [arr] (0,-2.5) -- (a0);
% Filter section

\node (s6) at (0,0) [state] {6};
\node (s5) at (2,2) [state] {5};
\node (s7) at (2,-2) [state] {7};

\node (s2) at (0,2) [oper] {2};
\node (s3) at (0,-2) [oper] {3};
\node (s1) at (2,0) [oper] {1};
\node (s4) at (4,0) [oper] {4};

\draw [arr](s4)[] to node [right]{$w_8$} node{}(s5);
\draw [arr](s4)[] to node [right]{$w_9$} node{}(s7);
\draw [arr](s2)[] to node [above]{$w_4$} node{}(s5);
\draw [arr](s2)[] to node [right]{$w_5$} node{}(s6);
\draw [arr](s1)[] to node [right]{$w_1$} node{}(s5);
\draw [arr](s1)[] to node [above]{$w_2$} node{}(s6);
\draw [arr](s1)[] to node [right]{$w_3$} node{}(s7);
\draw [arr](s3)[] to node [right]{$w_6$} node{}(s6);
\draw [arr](s3)[] to node [above]{$w_7$} node{}(s7);


\end{tikzpicture}

}
  \end{minipage}
  \begin{minipage}{0.60\textwidth}
    \resizebox{\linewidth}{!}{
      $\displaystyle
    A=A_+A_{-}=\begin{pmatrix}
        0    & 0 &    0 &    0 &    \omega_{45}   &  0 &    0 &    0 &    \omega_{89}\\
        0&     0&     0&     \omega_{54} &    0  &   0  &   \omega_{67}  &   0  &   0 \\
        0&     0&     0&     0&     0&     \omega_{76} &     0&      \omega_{98}   &   0\\
        0&     \omega_{12}      & \omega_{13}  &    0 &    0  &   0 &    0  &   0      & \omega_{89} \\
    \omega_{21}     &0     & \omega_{23}     &0   &0   &0     & \omega_{67}     &0 &   0\\
    \omega_{21}     &0     & \omega_{23}      & \omega_{54}    &0   &0   &0   &0   & 0\\
    \omega_{31}     & \omega_{32}  &0    &0   &0   &0   &0     & \omega_{98}     & 0\\
        0     & \omega_{12}      & \omega_{13}     &0    & \omega_{45}     &0   &0   &0   & 0\\
    \omega_{31}      & \omega_{32}     &0   &0   &0     & \omega_{76}     &0   &0   & 0\\
    \end{pmatrix}
    $
}
  \end{minipage}
\[
a_0^i=(\underbrace{0,0,...,0}_{i \text{ раз}},1,0,...,0)
\]
\[
a_{2L}^i=a_0^i \cdot A^L
\]
\end{frame}

\begin{frame}{Алгоритм подсчета спектра \\ Симметрии, обращение Мебиуса}
  \begin{minipage}{0.45\textwidth}
    \centering
    \input{../tikz/symmetry1.tikz}
  \end{minipage}\hfill
  \begin{minipage}{0.45\textwidth}
    $w_1w_4w_5w_2w_1w_4w_5w_2$ - 4 раза \\
    $w_1w_4w_5w_2w_1w_4w_6w_3$ - 8 раз
  \end{minipage}
\end{frame}

\begin{frame}{Алгоритм подсчета спектра \\ Лифтинг, веса}
    \begin{minipage}{0.2\textwidth}
        \centering
        \begin{tikzpicture}
[
scale=1,transform shape,
state/.style={rectangle,draw=black!50,fill=green!10,thick,minimum size=5mm},
stateBr/.style={rectangle,draw=black,fill=green!20,thick,minimum size=5mm},
oper/.style={circle,draw=black!50,fill=red!10,thick,minimum size=5mm, font=\small},
operBr/.style={circle,draw=black,fill=red!20,thick,minimum size=5mm, font=\small},
arr/.style={-,auto,>=stealth},
arrAct/.style={->,auto,>=stealth,thick,draw=red}
]
\small

\node (sA) at (0,2) [oper] {a};
\node (sB) at (0,-2) [state] {b};

\draw[green] [arr](sA)[] to[out=-50,in=50] node [right]{3} node{}(sB);
\draw[red] [arr](sA)[] to node [left]{1} node{}(sB);
\draw [arr](sA)[] to[out=230,in=130] node [left]{0} node{}(sB);

\end{tikzpicture}


    \end{minipage}\hfill
    \begin{minipage}{0.75\textwidth}
        \centering
        \begin{tikzpicture}
[
scale=1,transform shape,
state/.style={rectangle,draw=black!50,fill=green!10,thick,minimum size=5mm},
stateBr/.style={rectangle,draw=black,fill=green!20,thick,minimum size=5mm},
oper/.style={circle,draw=black!50,fill=red!10,thick,minimum size=5mm, font=\small},
operBr/.style={circle,draw=black,fill=red!20,thick,minimum size=5mm, font=\small},
arr/.style={-,auto,>=stealth},
arrAct/.style={->,auto,>=stealth,thick,draw=red}
]
\small

\foreach \x in {0,1,2,3,4,5,6} {
   \node (s\x) at (\x,2) [oper] {$a_{\x}$};
   \node (t\x) at (\x,-2) [state] {$b_{\x}$};
}
\foreach \x in {0,1,2,3,4,5,6} {
   \draw [arr](s\x)[] to node [above]{} node{}(t\x);
   \pgfmathparse{int(mod(\x+1, 7))}
   \draw[red] (s\x)[] to node [above]{} node{}(t\pgfmathresult);
   \pgfmathparse{int(mod(\x+3, 7))}
   \draw[green] [arr](s\x)[] to node [above]{} node{}(t\pgfmathresult);
}

\end{tikzpicture}


    \end{minipage}
    \vfill
    $O(M \cdot S \cdot E^3)$
\end{frame}

\newcommand{\plotstandard}[1]{\centerline{\includegraphics[height=3in]{#1}}}
\newcommand{\plotsmall}[1]{\includegraphics[height=1.3in]{#1}}

\begin{frame}{Экспериментальные исследования \\ Ричардсон-Урбанке 4x8 576}
  \plotstandard{../images/r4_576.pdf}
\end{frame}

\begin{frame}{Экспериментальные исследования \\ Ричардсон-Урбанке 4x8 2304}
  \plotstandard{../images/r4_2304.pdf}
\end{frame}

\begin{frame}{Экспериментальные исследования}
\begin{table}[!t]
  \begin{tabular}{ccc}
    \plotsmall{../images/g3_576.pdf}&
    \plotsmall{../images/g3_2304.pdf}&
    \plotsmall{../images/q3_576.pdf}\\
    \plotsmall{../images/q3_2304.pdf}&
    \plotsmall{../images/q4_576.pdf}&
    \plotsmall{../images/g4_576.pdf}\\
  \end{tabular}
\end{table}
\end{frame}

\begin{frame}{Выводы}
  \begin{itemize}
    \item Получены экспериментальные доказательства прямой зависимости между количеством коротких циклов 
      и вероятностью ошибки на блок для различных ансамблей кодов
    \item Разработан вычислительно эффективный алгоритм подсчета спектра графа Таннера, который может
      быть использован для ускорения отбора кодов
  \end{itemize}
\end{frame}

\begingroup
\setbeamercolor{background canvas}{bg=itmoblue}
\setbeamercolor{normal text}{fg=white}
\begin{frame}[plain]
\color{white}
\centering\bfseries
\includegraphics[width=0.35\textwidth]{itmo_small_blue_rus.png}

\vspace{0pt plus 0.3filll}

{\Large Спасибо за внимание \par}

\vspace{0pt plus 0.3filll}

{\Large Вопросы?}

\vspace{0pt plus 0.3filll}
\vspace{0pt plus 1filll}
\end{frame}
\endgroup


\end{document}
%%% Local Variables:
%%% mode: latex
%%% TeX-master: t
%%% End:
