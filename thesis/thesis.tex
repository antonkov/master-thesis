% -*-coding: utf-8-*-
% This is an AMS-LaTeX v. 1.2 File.

\documentclass{report}

%\usepackage{pscyr}
%\renewcommand{\rmdefault}{fjn}
%\renewcommand{\ttdefault}{fcr}

%\usepackage{showkeys}
\usepackage{cmap}
\usepackage[T2A]{fontenc}
\usepackage[utf8x]{inputenc}
\usepackage[english,russian]{babel}
\usepackage{expdlist}
\usepackage[pdftex]{graphicx}
\usepackage{amsmath}
\usepackage{amssymb}
\usepackage{amsthm}
\usepackage{amsfonts}
\usepackage{amsxtra} 
\usepackage{array}
\usepackage{wrapfig}
\usepackage{sty/dbl12}
\usepackage{srcltx}
\usepackage{epsfig}
\usepackage{verbatim}
\usepackage{sty/rac}
%\usepackage[russian]{sty/ralg}
\usepackage{listings}
\usepackage{placeins}
\usepackage{caption}
\usepackage{subfigure}
\usepackage{easytable}
%\usepackage{floatrow}
\captionsetup[table]{position=t,justification=raggedright,slc=off}

%\usepackage[
%    top    = 2.00cm,
%    bottom = 2.00cm,
%    left   = 3.00cm,
%    right  = 1.50cm]{geometry}
\hoffset = -10mm
\voffset = -20mm
\textheight = 230mm
\textwidth = 165mm

%%%%%%%%%%%%%%%%%%%%%%%%%%%%%%%%%%%%%%%%%%%%%%%%%%%%%%%%%%%%%%%%%%%%%%%%%%%%%%

% Redefine margins and other page formatting

%\setlength{\oddsidemargin}{0.5in}

% Various theorem environments. All of the following have the same numbering
% system as theorem.

\theoremstyle{plain}
\newtheorem{theorem}{Теорема}
\newtheorem{prop}[theorem]{Утверждение}
\newtheorem{corollary}[theorem]{Следствие}
\newtheorem{lemma}[theorem]{Лемма}
\newtheorem{question}[theorem]{Вопрос}
\newtheorem{conjecture}[theorem]{Гипотеза}
\newtheorem{assumption}[theorem]{Предположение}

\theoremstyle{definition}
\newtheorem{definition}[theorem]{Определение}
\newtheorem{notation}[theorem]{Обозначение}
\newtheorem{condition}[theorem]{Условие}
\newtheorem{example}[theorem]{Пример}
\newtheorem{algorithm}[theorem]{Алгоритм}
%\newtheorem{introduction}[theorem]{Introduction}

\renewcommand{\proof}{\\\textbf{Доказательство.}~}

%\def\startprog{\begin{lstlisting}[language=Java,basicstyle=\normalsize\ttfamily]}

%\theoremstyle{remark}
%\newtheorem{remark}[theorem]{Remark}
%\include{header}
%%%%%%%%%%%%%%%%%%%%%%%%%%%%%%%%%%%%%%%%%%%%%%%%%%%%%%%%%%%%%%%%%%%%%%%%%%%%%%%

\numberwithin{theorem}{chapter}        % Numbers theorems "x.y" where x
                                        % is the section number, y is the
                                        % theorem number

%\renewcommand{\thetheorem}{\arabic{chapter}.\arabic{theorem}}

%\makeatletter                          % This sequence of commands will
%\let\c@equation\c@theorem              % incorporate equation numbering
%\makeatother                           % into the theorem numbering scheme

%\renewcommand{\theenumi}{(\roman{enumi})}

%%%%%%%%%%%%%%%%%%%%%%%%%%%%%%%%%%%%%%%%%%%%%%%%%%%%%%%%%%%%%%%%%%%%%%%%%%%%%%


%%%%%%%%%%%%%%%%%%%%%%%%%%%%%%%%%%%%%%%%%%%%%%%%%%%%%%%%%%%%%%%%%%%%%%%%%%%%%%%

%This command creates a box marked ``To Do'' around text.
%To use type \todo{  insert text here  }.

\newcommand{\todo}[1]{\vspace{5 mm}\par \noindent
\marginpar{\textsc{ToDo}}
\framebox{\begin{minipage}[c]{0.95 \textwidth}
\tt #1 \end{minipage}}\vspace{5 mm}\par}

%%%%%%%%%%%%%%%%%%%%%%%%%%%%%%%%%%%%%%%%%%%%%%%%%%%%%%%%%%%%%%%%%%%%%%%%%%%%%%%

\binoppenalty=10000
\relpenalty=10000

\begin{document}

% Begin the front matter as required by Rackham dissertation guidelines

\initializefrontsections

\pagestyle{title}

\begin{center}
Санкт-Петербургский национальный исследовательский университет \\ информационных технологий, механики и оптики

\vspace{2cm}

Кафедра компьютерных технологий

\vspace{3cm}

{\Large А. П. Ковшаров}

\vspace{2cm}

\vbox{\LARGE\bfseries
Оптимизация параметров стратегий поиска объектов на море}

\vspace{4cm}

Бакалаврская работа 

\vspace{1cm}

{\Large Научный руководитель: А. С. Ковалев}

\vspace{5cm}

Санкт-Петербург\\ 2015
\end{center}

\newpage

\setcounter{page}{4}
\pagestyle{plain}

%\dedicationpage{Put a dedication here}
% Dedication page

%\startacknowledgementspage
% Acknowledgements page
%{Put Acknowledgements here}

% Table of contents, list of figures, etc.
\tableofcontents
%\listoffigures


\def\t#1{\mbox{\texttt{\hbox{#1}}}}
\def\b#1{\textbf{#1}}
\def\tb#1{\t{\b{#1}}}

\def\cln#1{\t{#1}}
\def\pcn#1{\t{#1}}
\newcommand{\p}{\par Здесь будет текст...}
\def\putImgx#1#2{
  \includegraphics[width=#1]{../img/#2}
}

\def\putImg#1{
  \includegraphics{../img/#1}
}

\def\drawfigure#1#2{
        \begin{figure}
        \putImg{#1}
        \caption{#2}
        \label{#1}
        \end{figure}
}
\def\drawfigurex#1#2#3#4{
        \begin{figure}[ht]
          \begin{center}
            \putImgx{#4}{#1}
            \caption{#2}
            \label{#3}
          \end{center}
        \end{figure}
}
 
% Chapters
\startthechapters
% -*-coding: utf-8-*-
\startprefacepage

Основной целью данной работы является разработка эффективного метода построения маршрутов поиска
объектов согласно общепринятым стратегиям поиска. Метод должен работать с большим количеством
различных моделей изменения распределения вероятности обнаружения объекта со временем. %Цель
 
 Существующие подходы строят весь маршрут исходя из данных в начальный момент
времени --- информации о начальном распределении и модели его изменения.
Таким образом главным недостатком существующих подходов является необходимость
разработки нового алгоритма для каждой модели изменения распределения.
Учитывая тот факт, что подобрать правильную модель, хорошо приближающую реальность,
 крайне непросто, возникает необходимость разработки алгоритма,
 работающего единообразно на широком классе различных моделей.
 Основная идея рассматриваемого подхода --- использование симулятора для получения %Новизна
 информации о распределении в любой момент времени при построении маршрута.
 Таким образом при планировании пользователь в первую очередь выберет модель,
 как можно лучше приближающую реальность в данном случае, а после запустит
 алгоритм построения маршрута.

 Таким образом задачами данной работы являются разработка инструмента симуляции изменения
вероятности во время прохождения маршрута и алгоритма построения маршрута при заданных условиях.
В данной работе будет рассмотрен алгоритм построения маршрута согласно
 стратегии ``Параллельное галсирование''.

Расчетная задача может быть использована оператором как вспомогательное средство
при планировании маршрутов поиска потерпевших бедствие объектов в реальной жизни.
Визуализированный процесс изменения распределения может пригодиться при решении какие области
следует исследовать более подробно, а какие можно пропустить.
%Практическое значение

Концепция данной задачи был продемонстрирована на одной из выставок. К задаче был проявлен 
интерес, и было решено внедрить ее в комплекс расчетных морских задач. %Актуальность

В главе 1 решаемая задача будет рассмотрена более подробно.
 Описаны классы маршрутов, получаемые при использовании стратегии ``Параллельное галсирование''. 
Будут приведены особенности задачи, которые отличают ее от классической задачи коммивояжера
 и делают невозможным использование ранее разработанных методов решения
 для решения исходной задачи в общем случае.

В главе 2 будут рассмотрены вопросы, связанные с разработкой симулятора на CUDA.
 Обозначены предоставляемые симулятором сервисы.

В главе 3 будет описан алгоритм построения маршрутов согласно стратегии ``Параллельное галсирование''.
В первую очередь будет рассмотрен подход с динамическим программирования для определения
маршрута при фиксированном распределении. Далее будет рассмотрена корректировка маршрута
согласно изменениям распределения.

В главе 4 будет осуществлено сравнение с существующим алгоритмом ``Кронштадт технологии''.
\FloatBarrier

%-*-coding: utf-8-*-
\chapter{Постановка задачи}
\label{chapSVD}

\section{Задача построения маршрута поиска в oбщeм случае}


\subsection{Расширения задачи коммивояжера}
Соответственно выделяют вершины в которых более вероятно обнаружить объект.
 Сопоставим вершине $v$ величину $p_v$ --- вероятность обнаружить объект в этой вершине.
 $p_{path}=\sum\limits_{v\in path}p_v$. На практике длина путей с $p_{path} \ge 0.99$ может превышать
длину путей с $p_{path} \ge 0.9$ в десятки раз. То есть длина пути может расти экспоненциально
в зависимости от $p_{path}$. Следовательно необходимым параметром задачи становится максимальная длина
пути (или время поиска с физической точки зрения). Известно обобщение Profit Based TSP \cite{dew13}:
каждой вершине сопоставляется значение $p_v$, при посещении вершины к сумме призов 
добавляется $(p_v-t_v)$, где $t_v$---время посещения, необходимо составить маршрут с
наибольшей суммой призов. К сожалению наша задача и здесь сравнительно более общая, так как
величины призов могут изменяться нелинейно.}

\FloatBarrier
%%% Local Variables:
%%% mode: latex
%%% TeX-master: "../thesis"
%%% End:

%-*-coding: utf-8-*-

\chapter{Симуляция эволюции распределения}
\section{Примеры моделей и распределений}
\begin{figure}[ht]
  \begin{center}
    \subfigure[Композиция стандартных распределений для задания начального распределения]{%
        \putImgx{0.4\textwidth}{pic03-1}
        \label{simul:init}
    }
    \subfigure[Изменения распределения согласно модели случайного блуждания]{%
        \putImgx{0.4\textwidth}{pic03-4}
        \label{simul:after}
    }
  \end{center}
  \caption{Пример симуляции изменения распределения без маршрута}
\end{figure}

\FloatBarrier
\subsection{Разновидности начальных распределений}
Начальное распределение частиц фактически может быть представлено любой двумерной функцией,
интеграл которой по всей плоскости равен единице. Однако для большинства применений кажется
достаточным задание начального распределения как композиции стандартных двумерных распределений
(рис. \ref{simul:init}), таких как равномерное распределение в произвольной области или
 нормальное распределение задаваемое эллипсом, содержащим $3\sigma$ вероятности и тому подобное.
 Инструмент симуляции предоставляет удобный инструмент их задания.

На протяжении всей симуляции распределение хранится в текстуре (матрице). Значение, которое
записано в каждом пикселе, означает суммарный вес частиц, находящихся под этим пикселем,
если его нарисовать в мире. Пиксель текстуры имеет достаточно малый размер при рисовании в мире,
чтобы достаточно точно различать положения различных частиц, при этом достаточно большой,
чтобы текстура покрывающая всю вероятность влезла в память GPU, и ее обработка занимала
приемлемое время. \FloatBarrier
\subsection{Разновидности моделей изменения распределения}
Изменения распределений происходят с фиксированным временем дискретизации $\Delta t$.
Для симуляции изменений в большинстве моделей используется свертка с ядром с периодом $\Delta t$.

Процесс свертки с ядром осуществляется следующим образом, пусть $D_t, D_{t+\Delta t}$ ---
последовательные распределения, а $K$ --- ядро размера $(2k+1) \times (2k+1)$
(Пусть обращение к несуществующим элементам возвращает 0):

\begin{equation}
  D_{t+\Delta t}[i][j] = \sum\limits_{di=-k}^k \sum\limits_{dj=-k}^k D_t[i+di][j+dj] \cdot K[di][dj]
\end{equation}

Размер матрицы свертки выбирается такого размера, чтобы дать возможность симулировать
передвижение в произвольном направлении с достаточной точностью. На данный момент выбрано
значение $k = 5$. $n = 5$

Время $\Delta t$ выбирается исходя из размера ядра и физического размера пикселя.
Если размер пикселя $h \times h$, максимальная скорость перемещения объекта $v_{max}$,
тогда (\ref{eq:deltat}). 
\begin{equation}\label{eq:deltat}
  \Delta t = \frac {k \cdot h} {v_{max}}
\end{equation}

\begin{itemize}
\item{
\textbf{Модель случайных блужданий}

Простейший пример модели изменения распределения --- модель случайных блужданий
(рис. \ref{simul:after}). Ядро для модели случайных блужданий представляет собой матрицу со
значениями равными площади пересечения текущей ячейки с кругом, вписанным в квадрат размера
$2kh \times 2kh$. Ячейки соответствующим образом нормированы для равенства суммы единице.

Посредством изменения ядра (замены круга на кольца), можно изменять интенсивность блужданий.
}
\item{
\textbf{Модель движения в одном направлении}

Пусть заданы последовательности $\alpha_i, p_i, 1 \le i \le n$ --- углов и вероятностей
движения в направлении каждого из них. Для симулирования заданной модели можно
использовать $n$ экземпляров ядер и текстур для симуляции движения распределений
соответственно в каждом из направлений. На выходе для получения итогового распределения
достаточно скомбинировать полученные результаты с соответствующими весами $p_i$.
}
\item{
\textbf{Модель притяжения(отталкивания) к заданным точкам}

Иногда уместной оказывается модель избегания некоторых точек (например точек вблизи
текущей позиции поискового средства, если объект является вражеским и не желает быть
обнаруженным) или, напротив, приближения к ним (например в случае если объект знает ближайшее
положение суши, он может стремиться двигаться в ее направлении). Здесь 
удобно воспользоваться аналогией с электрическими зарядами, однако, действие электрических
сил для создания ускорения не кажется естественным в данной задаче. Значение функции
аналогичной силе Кулона будет задавать направление и скорость (вместо ускорения) движения
 при нахождении в данной точке. Таким образом каждая из точек имеет свой заряд $q_i$,
 позицию $pos_i$, радиус действия $R_i$ и закон убывания влияния, для примера используем
 Кулоновские $r^{-2}$. Пусть максимальная скорость перемещения объекта $v_{max}$.
Таким образом вектор скорости объекта в точке $p$ равен (\ref{eq:vqulon}).
\begin{equation}
  \chi_i(p) = 
  \left\{
    \begin{array}{ll}
      1 & \mbox{если } {(p - pos_i)^2 \le R_i} \\
      0 & \mbox{иначе }
    \end{array}
  \right.
\end{equation}

\begin{equation}
  v' = \sum\limits_{i=1}^n\frac{\chi_i(p) \cdot q_i \cdot (pos_i-p)} {|pos_i-p|^3} 
\end{equation}

\begin{equation}\label{eq:vqulon}
  v = min(|v'|, v_{max}) \cdot \frac {v'} {|v'|}
\end{equation}

При реализации данной модели, во время свертки ядро в каждой конкретной ячейке будет различным.
Таким образом приходится работать с текстурой скоростей из глобальной памяти вместо ядра в константной
памяти как в модели со случайными блужданиями, что в несколько раз замедляет время симуляции.
}
\end{itemize}
\FloatBarrier

\section{Процесс симуляции изменения распределения
 и прохождения маршрута}
\FloatBarrier
\subsection{Симуляция изменения распределения}
\drawfigurex{keyframe.png}{Схема применения частичных ядер}{pic:keyframe}{\textwidth}

Симуляция изменения распределения производится путем применения свертки текущего
распределения с ядром с промежутком времени $\Delta t_{frame}$. Если
$\frac {1} {\Delta t} \ge 60 FPS$, где $\Delta t$ --- время с которым необходимо применять
заданное ядро, тогда $\Delta t_{frame} = \Delta t$. Иначе время применения ядра слишком велико
и не обеспечивает необходимой динамичности.

Для обеспечения актуальной информации в промежутках между ключевыми кадрами, идущими с 
интервалом $\Delta t$, необходимо добавить $m$ промежуточных, так чтобы
$\Delta t_{frame} = \frac {\Delta t} {m + 1} \ge 60 FPS$. Таким образом из оригинального ядра
 $K_1$ нам необходимо получить $m$ промежуточных с некоторой точностью приближающих реальность 
$K_{\frac {1} {m+1}}, K_{\frac {2} {m + 1}}, ... , K_{\frac {m} {m + 1}}$. Посредством применения
заданных ядер к последнему ключевому кадру, будут получены приближенные распределения в 
промежуточные моменты времени (рис. \ref{pic:keyframe}), однако, полученные с помощью этих
 ядер текстуры не будут использованы для построения следующих, что избавляет симулятор 
от накопления погрешности от применения приближенных ядер.

Далее описан один из вариантов получения ядра $K_{\frac {i} {m + 1}}$, которой достаточно
хорошо демонстрирует себя на практике. Необходимо элементы ядра $K_{1}$ домножить на
$\alpha = 1 + c_0 - c_0^{\frac {i} {m+1}}$, где $c_0 = K_{1}[k][k]$, после чего добавить 
остаток $1 - \alpha$ к стационарной ячейке получившегося ядра $K_{\frac {i} {m + 1}}[k][k]$.
\FloatBarrier
\subsection{Симуляция сбора частиц средством поиска}
Сбор частиц производится с интервалом $\Delta t_{frame}$ в текстуре, соответствующей последнему
на данный момент ключевому кадру. После сбора к ключевому кадру применяются ядра,
рассчитывающие перемещение оставшихся частиц, для получения кадра, соответствующего
текущему времени.

Сбор частиц осуществляется посредством умножения значения каждого пикселя на долю его площади,
не покрытой зоной видимости ни в один из моментов времени в течении текущего интервала
$\Delta t_{frame}$. То есть подразумевается, что вес частицы распределен равномерно по площади
пикселя.

 Для получения суммарной зоны видимости во все моменты времени текущего интервала маршрут
средства поиска делится на прямолинейные участки. После чего на каждом участке берется объединение
$n = \beta \frac{dist}{r}$, взятых через равные промежутки, зон видимости, где $dist$ --- длина
прямолинейного участка, $r$ --- радиус видимости средства поиска, $beta$ --- коэффициент
точности края. Выбрано $\beta = 100$, что для всех разумных размеров текстур дает достаточно точные
края объединения зон видимости.

\FloatBarrier
\section{Корректировка области симуляции с течением времени}
\begin{figure}[ht]
  \centering
\begin{minipage}[t]{.48\textwidth}
  \centering
  \putImgx{\textwidth}{pic07-1}
  \captionof{figure}{Начальная область симуляции}
  \label{corr:1}
\end{minipage}
\begin{minipage}[t]{.48\textwidth}
  \centering
  \putImgx{\textwidth}{pic07-2}
  \captionof{figure}{Выход распределения за границы области, без корректировки}
  \label{corr:2}
\end{minipage}
\end{figure}
Одна из сторон прямоугольника $rect$, охватывающего интересующий район поиска, параллельна прямой
галсирования (в случае построения маршрутов стратегией ``Параллельное галсирование''). 
В начальный момент времени район задается вручную пользователем (рис. \ref{corr:1}).
Частицы, оказавшиеся снаружи начального района учитываться не будут.

Разрешение текстуры выбирается таким образом, чтобы размер пикселя был квадратным
(при необходимости немного изменяется выбранный район для симуляции), а количество пикселей
было максимально возможным, при котором можно обеспечить приемлемую производительность.
Сейчас выбрано количество пикселей порядка $800^2 = 6.4 \cdot 10^5$.

С течением времени рассматриваемые частицы могут выходить за пределы текущего рассматриваемого
района. Без корректировки района информация о местоположении этих частиц будет потеряна
(рис. \ref{corr:2}).

\begin{figure}[ht]
  \centering
\begin{minipage}[t]{.48\textwidth}
  \centering
  \putImgx{\textwidth}{pic07-3}
  \captionof{figure}{Приближение распределения к границе области симуляции}
  \label{corr:3}
\end{minipage}
\begin{minipage}[t]{.48\textwidth}
  \centering
  \putImgx{\textwidth}{pic07-4}
  \captionof{figure}{Изменение центра области симуляции согласно новому положению распределения}
  \label{corr:4}
\end{minipage}
\end{figure}

Корректировка рассматриваемого района осуществляется, когда в одном из пикселей на границе
шириной $2k$, где $k$ --- параметр ядра, находятся частицы с суммарным весом более $\epsilon$.
(рис. \ref{corr:3}) В реализации выбрано $\epsilon = 5 \cdot 10^{-3}$.

Для корректировки находится прямоугольник $rect$, охватывающий все пиксели суммарный вес в которых
больше $\epsilon$, причем длина и ширина прямоугольника содержат четное число пикселей.
После чего выделяется новая текстура с центром, совпадающим с центром полученного прямоугольника
(рис. \ref{corr:4}) (благодаря четности, стороны пиксели новой и старой текстуры выровнены), 
после чего данные копируются в новую текстуру.

Так как разрешение новой текстуры всегда остается прежним, возможно необходимо увеличить
прямоугольник покрываемый текстурой для покрытия прямоугольника $rect$. Размер покрываемого
прямоугольника последовательно увеличивается в два раза, равномерно расширяясь относительно центра,
до тех пор пока не покроет прямоугольник $rect$, (рис. \ref{corr:5}) после чего его положение
немного корректируется, чтобы в каждый новый пиксель (если увеличили в $2^m$ раз) попало ровно
$2^{2m}$ целых старых пикселей.

\begin{figure}[ht]
  \centering
    \putImgx{0.8\textwidth}{pic07-5}
    \caption{Расширение и изменение центра области симуляции}
    \label{corr:5}
\end{figure}

\section{Статистика прохождения маршрута}
\begin{figure}[ht]
  \centering
    \putImgx{\textwidth}{pic08-2}
    \caption{Окно с отображением статистики}
    \label{stat:0}
\end{figure}

Симулятор предоставляет статистику о прохождении маршрута, по которой строятся графики
прогресса поиска и поисковой производительности. Оба графика отображают зависимости
от времени, поэтому изображены в одном окне с разными шкалами по оси ординат. (рис. \ref{stat:0})

Прогресс поиска (красный график) --- отображает текущий суммарный вес собранных частиц
в процентах от общего веса в начальный момент времени.

Поисковая производительность (синий график) --- отображает текущую эффективность поиска.
Фактически является производной прогресса поиска по времени, однако также измеряется
в процентах. Поисковая производительность $x\%$ --- означает, что если мы будет искать
с такой поисковой производительностью в течении всего времени поиска $t_{search}$,
то сможем собрать $x\%$ веса частиц. 

\FloatBarrier

%-*-coding: utf-8-*-
\chapter{Алгоритм построения маршрутов согласно стратегии
``Параллельное галсирование''}
\section{Алгоритм построения маршрута при
 фиксированном распределении}

\def\dprule{ \rule[-2ex]{0pt}{4ex} }

Для построения маршрута в случае сохранения частицами их положений будет использован метод
динамического программирования. Далее будут описаны состояния алгоритма, его переходы и
вспомогательные значения, посчитанные на матрице распределения.

Прежде всего введем сетку на прямоугольнике на котором рассматривается распределение.
Пусть размер сетки будет $H \times W$, размеры ячеек будут одинаковыми, но не обязаны
быть квадратными или выровненными по пикселям текстуры. В частности мы будем стремиться
к очень высокому разрешению по строчкам для более точного определения расстояния между
соседними галсами и нам будет достаточно небольшого разрешения по столбцам. Например,
если размеры текстуры $T_H \times T_W$, а размеры прямоугольника в мире $R_H \times R_W$,
то мы хотим иметь размеры сетки $H = T_H$ и $W \approx \frac {R_w} {1000}$.

\subsection{Состояния алгоритма}
Значением динамического программирования $dp(state)$ для определенного состояния будет
суммарный вес собранных частиц к моменту времени определяемому состоянием. На самом деле
храниться будет весь суммарный весь за исключением покрытого лишь передним полукругом
текущего положения. 

Состоянием динамического программирования является кортеж $(cntHor, row, col, move, notCleared)$.
\begin{itemize}
\item $(row, col)$ --- строка и столбец ячейки в которой закончен маршрут к текущему моменту времени
соответственно.
\item $cntHor$ --- количество горизонтальных перемещений (между столбцами). Если обозначить
за $time_W$ --- время перемещения между двумя соседними ячейками по горизонтали, а за $time_H$ ---
по вертикали, то текущее время можно посчитать как $curTime = time_H \cdot row + time_w \cdot cntHor$.
Здесь использованы предположения, что средство поиска никогда не возвращается назад по строчками
(исходя из выбранной стратегии поиска), всегда двигаемся с максимальной скоростью и в начальный
момент времени находится в нулевой строке.
\item $move$ --- тип текущего хода (который завершится в этом состоянии). Принимает значение 
из множества $\{L, R, LU, RU\}$. $L/R$ --- означает, что мы галсируем в текущей строчке влево или
вправо соответственно (галсировать  в обе стороны в одной строчке нельзя). 
$LU/RU$ --- означает, что мы поднимаемся по строчкам, причем в последней строчке в которой мы
галсировали мы двигались влево или вправо соответственно. Уточнение в какую сторону мы двигались
последний раз необходимо для более точного учета собранных нами частиц во время подъема по строчкам.
\item $notCleared$ --- количество пропущенных строк без галсирования. Главное предположение,
позволяющее значительно сократить количество состояний, состоит в том, что для понимания
какие из частиц еще не были собраны в текущей полосе галсирования,
нам достаточно знать когда был предыдущий галс или другими словами сколько строк мы пропустили без
галсирования. Зная какие из частиц не собраны, мы сможем рассчитать сколько частиц мы соберем
в текущий момент прохождения маршрута. Данный параметр принимает $N$ значений. $0$ означает,
что мы галсировали прямо в предыдущей строчке, $(N-1)$ --- последний раз мы галсировали $N$ строк
назад либо еще дальше. $N$ выбирается таким образом, чтобы галсирование $N$ строк назад
не оказывало значительного влияния на текущее.
\end{itemize}

\subsection{Вспомогательные матрицы с частичными суммами весов}
Обозначим радиус зоны видимости средства поиска за $r$. Обозначим функции, переводящие
из системы координат построенной сетки в мировую систему координат, за $w_x(col), w_y(row)$.

Предположим, что вес частиц в каждом пикселе распределен равномерно. Тогда с помощью
сумм на префиксе, построенных на текстуре за время $O(T_W \cdot T_H)$, можно за $O(1)$
отвечать на запросы вида сумма на прямоугольнике $[x_l ; x_r] \times [y_l ; y_r]$.
При вычислении суммы используется линейная интерполяция на пикселях частично попавших
в прямоугольник запроса.

\drawfigurex{dpzones}{Типы зон при прохождении маршрута}{dpzones}{0.5\textwidth}

Для подсчета веса частиц, собранных при прохождении маршрута, маршрут будет разделен на
части разного типа (рис. \ref{dpzones}). Как видно из картинки некоторые части могут
быть частично учтены ранее, поэтому важно знать значение $notCleared$ --- сколько строк
было пропущено без галсирования. Таким образом все предпосчитанные матрицы будут возвращать
значения с учетом $notCleared$. Самое большое значение $notCleared$ равное $N-1$ будет
использовано также в случаях, когда требуется считать, что последний галс был
достаточно далеко, чтобы не оказывать никакого влияния в текущем положении.

Для примера рассмотрен переход между двумя галсами, содержащий части всех возможных типов.
В примере ширина ячейки в два раза больше ее высоты. Примеры ячеек сетки динамики 
изображены красными прямоугольниками. Красная точка в центре 
ячейки сетки обозначает положения средства поиска при нахождении в этом состоянии динамики.
Для упрощения пусть точка с координатами $(row, col)$ совпадает с центром соответствующей ячейки.
Тогда область определения $w_x \in [-0.5; M-0.5]$, а $w_y \in [-0.5; N-0.5]$.
Желтым отмечен маршрут. Разные типы частей помечены различными цветами. Черная сетка на
фоне может интерпретироваться как сетка пикселей.

Рассмотрим вертикальные части синего цвета. Значение матрицы $vl[row][col][notCleared]$
будет содержать сумму в прямоугольнике
$[w_x(col-1), w_x(col)] \times [w_y(row)-r, w_y(row)+r]$. Без 
учета частиц собранных ниже прямой $w_y(row-1-notCleared) + r$. А $vr[row][col][notCleared]$
будет содержать $[w_x(col), w_x(col+1)] \times [w_y(row)-r, w_y(row)+r]$.

Розовый и оранжевый тип отличаются лишь положением относительно маршрута, справа или слева
соответственно. Матрица для оранжевого 
$hl[row][col][notCleared]$ содержит сумму в
$[w_x(col)-r, w_x(col)] \times [w_y(row-1), w_y(row)]$. Матрица для розового
$hr[row][col][notCleared]$ содержит $[w_x(col), w_x(col)+r] \times [w_y(row-1), w_y(row)]$.

Обратим внимание на область, покрытую одновременно оранжевым и синим цветом в верхней части картинки.
До тех пор пока мы не повернули налево для прохождения нового галса --- эта область будет
учтена как область оранжевого типа. После поворота она будет переучтена как область синего.
Для исключения первого, ошибочно посчитанного раза необходимо предпосчитать
матрицу $ol[row][col][notCleared]$ --- область $[w_x(col)-r, w_x(col)] \times [w_y(row)-r, w_y(row)]$
и матрицу $or[row][col][notCleared]$ --- область $[w_x(col),w_x(col)+r]\times [w_y(row)-r, w_y(row)]$.

Последним типом являются четверти круга (зеленый цвет), которые должны быть учтены при поворотах.
Суммарный вес под ними будет хранить матрица $cs[row][col][quarter][notCleared]$.
$quarter$ --- обозначает четверть и будет принимать значения $Q_{RU}, Q_{LU}, Q_{LD}, Q_{RD}$
соответственно для четвертей в порядке их стандартной нумерации. Для вычисления суммарного
веса, находящегося под четвертью круга, используется численное интегрирование посредством
вычисления сумм на прямоугольниках.
\FloatBarrier
\subsection{Переходы между состояниями}
Переходы из состояния динамики \[state=(cntHor, notCleared, row, col, curMove)\] будут
описаны в таблице \ref{table:dp} следующим образом. Текущий ход
$curMove$ будет зафиксирован в первом столбце, следующий выбранный во втором.
$newState$ --- состояние в которое мы перейдем, выбрав такой ход, указано
в четвертом столбце. Все величины, которые нужно добавить при переходе
к величине $dp(state)$ перед релаксацией значения $dp(newState)$ указаны
в третьем столбце. Если обозначить эти величина за $a_i, i \in 1..k$, то более
формально $dp(newState) = min(dp(newState), dp(state) + \sum_{i=1}^{k} a_i)$.
\begin{table}[ht]
  \centering
\begin{tabular}{|l|l|l|l|}
  \hline
  Текущий & Следующий & Собрано на текущем ходу & Новое состояние  \\
  ход & ход & & \\
\hline 
\dprule $L$&$L$&$vr[row][col-1][notCleared]$&$(cntHor+1,notCleared,row,col-1,L)$\\
\hline 
\dprule $L$&$LU$&$hl[row+1][col][N-1]$&$(cntHor,0,row+1,col,LU)$\\
\dprule  && $hr[row+1][col][0]$ & \\
\dprule  && $cs[row][col][Q_{LD}][notCleared]$ & \\
\hline 
\dprule $R$&$R$&$vl[row][col+1][notCleared]$&$(cntHor+1,notCleared,row,col+1,R)$\\
\hline 
\dprule $R$&$RU$&$hr[row+1][col][N-1]$&$(cntHor,0,row+1,col,RU)$\\
\dprule  && $hl[row+1][col][0]$ & \\
\dprule  && $cs[row][col][Q_{RD}][notCleared]$ & \\
\hline 
\dprule $LU$&$LU$&$hl[row+1][col][N-1]$&$(cntHor,notCleared+1,row+1,col,LU)$\\
\dprule && $hr[row+1][col][notCleared+1]$ & \\
\hline 
\dprule $LU$&$L$&$vr[row][col-1][notCleared]$&$(cntHor+1,notCleared,row,col-1,L)$\\
\dprule && $cs[row][col][Q_{RU}][notCleared]$ & \\
\dprule && $-ol[row][col][notCleared]$ & \\
\hline 
\dprule $LU$&$R$&$vl[row][col+1][notCleared]$&$(cntHor+1,notCleared,row,col+1,R)$\\
\dprule  && $cs[row][col][Q_{LU}][N-1]$ & \\
\dprule && $-or[row][col][notCleared]$ & \\
\hline 
\dprule $RU$&$RU$&$hr[row+1][col][N-1]$&$(cntHor,notCleared+1,row+1,col,RU)$\\
\dprule && $hl[row+1][col][notCleared+1]$ & \\
\hline 
\dprule $RU$&$L$&$vr[row][col-1][notCleared]$&$(cntHor+1,notCleared,row,col-1,L)$\\
\dprule && $cs[row][col][Q_{RU}][N-1]$ & \\
\dprule && $-ol[row][col][notCleared]$ & \\
\hline 
\dprule $RU$&$R$&$vl[row][col+1][notCleared]$&$(cntHor+1,notCleared,row,col+1,R)$\\
\dprule && $cs[row][col][Q_{LU}][notCleared]$ & \\
\dprule && $-or[row][col][notCleared]$ & \\
\hline 
\end{tabular}
\captionsetup{justification=centering}
\caption{Переходы из состояния $(cntHor,notCleared,row,col,curMove)$}
\label{table:dp}
\end{table}
\FloatBarrier
\subsection{Оценка времени работы}
Оценим количество операций для построения вспомогательных матриц. Подсчет значений в матрицах
$hl, hr, ol, or, vl, vr$ можно осуществить за $O(1)$ операций, если предпосчитать префикс
суммы на текстуре за $O(T_W \cdot T_H)$. Количество значений в каждой из этих матриц
$O(W \cdot H \cdot N)$ или при грубой оценке $N$ --- $O(W \cdot H^2)$.

Из каждого состояния динамики осуществляется $O(1)$ переходов, таким образом для оценки
времени работы алгоритма достаточно посчитать количество состояний. Если обозначить
за $T = \frac {t_{search}} {time_W}$, то количество состояний составит $O(T \cdot W \cdot H \cdot N)$.
Однако в большинстве случаев время достаточное для обследования всей территории с вероятностью
близкой к 100\% составляет $W \cdot \frac {H} {N}$.  Таким образом количество состояний
составит $O(W^2 \cdot H^2)$.

Максимальную точность позволяет получить значение $H \approx T_H$, однако на практике оказывается
достаточным $H = \frac {T_H} {4}$, что также позволяет алгоритму работать достаточно быстро.

На практике достижимых состояний оказывается втрое меньше их общего числа. Одна из причин
этого то, что не имеет смысла проводить больше времени чем нужно на каком-либо префиксе
пути. Если все частицы на этом префиксе можно собрать за определенное время $\tau$, состояния,
которые тратят больше времени, оказываются лишними. Вторая причина противоположная --- до некоторых
положений невозможно дойти за время меньше некоторого определенного.
\section{Корректировка маршрута}
\begin{figure}[ht]
  \centering
\begin{minipage}[t]{.4\textwidth}
  \centering
  \putImgx{\textwidth}{pic02-init}
  \captionof{figure}{Маршрут построенный на начальном распределении}
  \label{algo:init}
\end{minipage}
\begin{minipage}[t]{.4\textwidth}
  \centering
  \putImgx{\textwidth}{pic02-before}
  \captionof{figure}{Маршрут оказался неудовлетворительным с течением времени}
  \label{algo:before}
\end{minipage}
\end{figure}

Вернемся к изначальной постановке задачи, когда частицы могут изменять свое местоположение
согласно некоторой модели. На рис. \ref{algo:init} изображен маршрут, построенный
рассмотренным алгоритмом. Изменение распределения следует комбинации моделей случайного
блуждания и приближения к вертикальной прямой (берегу), находящейся слева от распределения.
Таким образом к середине прохождения маршрута становится заметно (рис. \ref{algo:before}), что
положения частиц изменились настолько, что текущий планируемый маршрут оказался неудовлетворительным.

Предлагаемым решением является перестройка оставшейся части маршрута (рис. \ref{pic02-after})
в момент времени когда становится понятно, что текущий маршрут не получит достаточного
суммарного веса.

\drawfigurex{pic02-after}{Оставшаяся часть маршрута была перестроена}{pic02-after}{0.4\textwidth}

Рассмотрим критерий необходимости перестройки маршрута в момент времени $t_i$, если последний
раз маршрут перестраивался в момент времени $t_{i-1}$. Разделим построенный
маршрут на две части: $path_{t_{i-1} \le t \le t_{i}}$ --- уже пройденная часть маршрута, $path_{\ge t_{i}}$
--- запланированная часть маршрута. Далее рассмотрим два симулятора прохождения маршрута в которых
частицы не изменяют своего местоположения со временем. Симулятор $D_0$ содержит распределение
в момент времени $t_{i-1}$ на котором был просимулирован маршрут $path_{t_{i-1} \le t \le t_{i}}$.
Симулятор $D_1$ содержит распределение в момент времени $t_i$. На каждом из симуляторов оценивается
суммарный вес, который будет собран на маршруте $path_{\ge t_{i}}$. Пусть это будут величины
$sum_0$ и $sum_1$ соответственно. Маршрут должен быть перестроен в случае, если
 $sum_1 \le p \cdot sum_0$. В реализации выбрано значение $p = 0.98$.

Данный критерий корректировки, как будет показано далее, позволяет достаточно успешно корректировать
маршрут для сбора достаточного веса частиц при активно изменяющемся положении частиц.

\FloatBarrier

%%% Local Variables:
%%% mode: latex
%%% TeX-master: t
%%% End:

\chapter{Сравнение с существующими решениями}
\section{Обзор существующих решений}
Был разработан алгоритм, строящий маршруты поиска, которые удовлетворяют фиксированным паттернам
стратегии ``Параллельное галсирование''. Задача построения маршрутов имеющих определенную
структуру достаточно специфична. К сожалению, методы, решающие поставленную задачу (или
ее частные случаи) не были найдены в открытых источниках.

 Однако существует комплекс инструментов, ранее разработанный в ``Кронштадт Технологии'',
который предоставляет возможность строить маршруты разными стратегиями поиска, несколькими
поисковыми средствами, оптимизировать параметры совместного поиска и некоторые другие возможности.
Рассмотренный в данной работе метод должен прийти на замену существующему в данном комплексе,
поэтому в первую очередь должно быть осуществлено сравнение с этим методом.

Далее будет произведено сравнение с существующим алгоритмом построения маршрутов стратегией
``Параллельное галсирование''.

\FloatBarrier
\section{Сравнение со старым алгоритмом построения маршрутов ``Кронштадт Технологии''}
Старый алгоритм имеет немного другой интерфейс. На вход принимается район в котором необходимо
осуществить галсирование. Единственная поддерживаемая модель изменения распределения ---
случайное блуждание с определенной интенсивностью $I$. Интенсивность $I$ определяет сколько
времени потребуется частицам, чтобы не менее $p\%$ отдалились на единицу длины. Исходя из
интенсивности рассчитывается расстояние между галсами --- одинаковое на всем протяжении маршрута.
Причем допускается пропуск частиц суммарного веса $\epsilon$ на каждом галсе.

Сравнение будет происходить следующим образом. Тест 1 продемонстрирует несостоятельность старого
метода в условиях другой модели изменения распределения. Тест 2 покажет преимущества нового метода
при неравномерных распределениях. Все последующие тесты будут иметь равномерное распределение
и модель изменения случайное блуждание. Тест 3 покажет важность корректировки маршрута со
временем, так как район расширяется и ширина галса со временем должны быть увеличена.
Тест 4 проверит способность нового метода подстраивать расстояние между галсами в простейшем случае,
когда частицы не передвигаются. Кроме того тест 4 покажет способность нового метода учитывать зоны,
досматриваемые во время движения вверх.

В старом методе время поиска не фиксируется, а возвращается по результату работы алгоритма.
Таким образом для каждого теста новым алгоритм будет построено два маршрута. Первый маршрут
будет демонстрировать лучший результат, занимая столько же времени сколько маршрут
старого алгоритма. Если времени данного старым алгоритмом недостаточно для построения
хорошего маршрута в данном случае --- будет построен маршрут с большим временем поиска.
Второй маршрут будет демонстрировать лучшее время, при этом не теряя в результативности
по сравнению со старым алгоритмом.

Для каждого маршрута будет показана статистика --- прогресс и поисковая производительность
на протяжении всего времени поиска. Синий график поисковой производительности имеет
относительный масштаб и может быть использован только для сравнения поисковой производительности
в рамках одного поиска (если его абсолютная величина на одном из двух графиков меньше --- это
не значит, что поисковая производительность в этом запуске хуже чем во втором).
\subsection{Teст 1}

Воспользуемся моделью изменения распределения --- притяжение к определенным точкам. Пусть
частицы стремятся приблизиться к берегу, расположенному слева от района поиска. Новый алгоритм
скорректирует маршрут, в то время как старый продолжит обследовать область их начального
местоположения.

 Старый алгоритм запросил 3.6 часа и собрал 85.7\% суммарного веса.
За это же время новый алгоритм собрал 94.7\%, а для сбора 85.7\% ему потребовалось
менее 2.9 часа.

\subsection{Teст 2}
Модель изменения распределения --- случайное блуждание. В левом нижнем и правом верхнем
углу поля расположены нормальные распределения. Вес частиц значительно убывает при удалении
от центров распределений, следовательно поиск в тех местах не является целесообразным.
Старый алгоритм этого не учитывает.

Таким образом старый алгоритм по прежнему тратит 3.6 часа, собирая при этом 83.3\%.
За это же время более аккуратный новый алгоритм собирает 92.6\%. Для сбора старого
результата новому будет достаточно почти вдвое меньшего времени 1.9 часа.

\subsection{Teст 3}
Модель изменения распределения --- интенсивное случайное блуждание. Таким образом
район увеличивается со временем и построенный изначально маршрут будет покрывать все меньше
частиц с течением времени.

Старый алгоритм за 3.6 часа собирает 83.3\%, проходя по наиболее вероятным местам. Новый алгоритм
за то же время может собрать такой же вес. Однако, если увеличить время поиска для нового алгоритма,
он начнет собирать частицы, отдаляющиеся от начального района.

\subsection{Teст 4}
В данном тесте частицы не передвигаются со временем. Старый алгоритм, ожидаемо, рассчитывает
верное расстояние между галсами и собирает все частицы за время 3.6 часа. За это же время
новый алгоритм при разрешении по строкам $T_H$, правильно выбрав расстояние, также собирает
все частицы. Однако с незначительными потерями (0.3\%) новый алгоритм может сэкономить
много времени на поворотах (до четверти часа), учитывая то, что на поворотах частицы также собираются.

 
\def\testPics#1#2{
\begin{figure}[ht]
  \begin{center}
    \subfigure[Старый алгоритм]{%
        \putImgx{0.4\textwidth}{t#1op}
        \label{t#1op}
    }
    \subfigure[Статистика]{%
        \putImgx{0.4\textwidth}{t#1os}
        \label{t#1os}
    } \\
    \subfigure[Новый --- сравнение по результату]{%
        \putImgx{0.4\textwidth}{t#1np}
        \label{t#1np}
    }
    \subfigure[Статистика]{%
        \putImgx{0.4\textwidth}{t#1ns}
        \label{t#1ns}
    } \\
    \subfigure[Новый --- сравнение по времени]{%
        \putImgx{0.4\textwidth}{t#1sp}
        \label{t#1sp}
    } 
    \subfigure[Статистика]{%
        \putImgx{0.4\textwidth}{t#1ss}
        \label{t#1ss}
    } 
\end{center}
  \caption{#2}
\end{figure}
}

\testPics{1}{Тест 1. Модель изменения --- приближение к прямой слева}
\testPics{2}{Тест 2. Начальное распределение --- композиция нормальных}
\testPics{3}{Тест 3. Случайное блуждание --- район увеличивается со временем}
\testPics{4}{Тест 4. Частицы не передвигаются с течением времени}

\FloatBarrier

\subsection{Результаты}
Далее приведена сводная таблица результатов тестов. Во втором столбце указано время,
которое занимает прохождение маршрута, построенного старым алгоритмом. В третьем столбце указан
результат, получаемый при прохождении старого алгоритма. В четвертом столбце указан
результат нового алгоритма при оптимизации по результату. Если не указано значения времени
в скобках --- значит данный результат получен за то же время, что и результат старого алгоритма.
В последнем столбце указан результат нового алгоритма при минимизации времени. Если не указано
значения в скобках --- значит за данное время получен результат не хуже результата старого
алгоритма.

\begin{table}[ht]
  \centering
  \begin{tabular}{|l|l|l|l|l|}
\hline
    Номер & Время старого & Результат старого & Сравнение по & Сравнение по \\
    теста     & & & результату & времени \\
\hline
    1 & 3.6 & 87.5\% & 94.7\% & 2.9 \\ 
    2 & 3.6 & 83.3\% & 92.6\% & 1.9 \\
    3 & 3.6 & 83.3\% & 86.1\% (4.5) & 3.6 \\
    4 & 3.6 & 100.0\%& 100.0\%& 3.3 (99.7\%) \\
\hline

  \end{tabular}
  \captionsetup{justification=centering}
  \caption{Сводная таблица результатов работы на тестах}
  \label{results}
\end{table}

\FloatBarrier

\startconclusionpage

\begin{enumerate}
	\item Получены экспериментальные доказательства прямой зависимости между количеством коротких циклов и вероятностью ошибки на блок для различных ансамблей кодов
	\item Разработан вычислительно эффективный алгоритм подсчета спектра графа Таннера, который может быть использован для ускорения отбора кодов.
\end{enumerate}


%\startappendices
%\label{appendix}
%\input{appendix}

\bibliographystyle{sty/utf8gost705u}
\bibliography{thesis}

\end{document}

%%% Local Variables:
%%% mode: latex
%%% TeX-master: t
%%% End:
