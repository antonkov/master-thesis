\documentclass[specification,annotation,times]{itmo-student-thesis}

%% Опции пакета:
%% - specification - если есть, генерируется задание, иначе не генерируется
%% - annotation - если есть, генерируется аннотация, иначе не генерируется
%% - times - делает все шрифтом Times New Roman, требует пакета pscyr.

%% Делает запятую в формулах более интеллектуальной, например:
%% $1,5x$ будет читаться как полтора икса, а не один запятая пять иксов.
%% Однако если написать $1, 5x$, то все будет как прежде.
\usepackage{icomma}

%% Данные пакеты необязательны к использованию в бакалаврских/магистерских
%% Они нужны для иллюстративных целей
%% Начало
\usepackage{tikz}
\usetikzlibrary{arrows}
\newcommand{\inputTikZ}[1]{\input{../tikz/#1.tikz}}

\newcommand{\bs}[1]{\ensuremath{\boldsymbol{#1}}}

\usepackage{caption}
\usepackage{subcaption}
\usepackage{pgfplots}
\usepackage{mathtools}
\usepackage{multirow}
\usepackage{enumitem}
\usepackage{blkarray}

\DeclarePairedDelimiter\floor{\lfloor}{\rfloor}
%% Конец

%% Указываем файл с библиографией.
\addbibresource{thesis.bib}

\begin{document}

\studygroup{M4239}
\title{Исследование зависимости вероятности ошибки на блок от спектра графа Таннера для МППЧ-кодов}
\author{Ковшаров Антон Павлович}{Ковшаров А.П.}
\supervisor{Буздалов Максим Викторович}{Буздалов М.В.}{канд. техн. наук}{научный сотрудник Университета ИТМО}
\publishyear{2017}
%% Дата выдачи задания. Можно не указывать, тогда надо будет заполнить от руки.
\startdate{01}{сентября}{2015}
%% Срок сдачи студентом работы. Можно не указывать, тогда надо будет заполнить от руки.
\finishdate{31}{мая}{2017}
%% Дата защиты. Можно не указывать, тогда надо будет заполнить от руки.
\defencedate{15}{июня}{2017}

\addconsultant{Кудряшов Б.Д.}{докт. техн. наук, профессор}
\addconsultant{Бочарова И.Е.}{канд. техн. наук, доцент}

%% Задание
%%% Техническое задание и исходные данные к работе
\technicalspec{В рамках работы требуется исследовать зависимость между спектром графа Таннера МППЧ-кода и его
 эффективностью при декодировании. Для проведения исследования необходимо разработать 
 эффективный алгоритм вычисления спектра, позволяющий провести отбор кодов с хорошим спектром среди широкого спектра
 сгенерированных кодов. Также необходимо разработать итеративный декодер, позволяющий измерить эффективность кода 
 посредством симуляции передачи кодовых слов через канал с шумом.}

%%% Содержание выпускной квалификационной работы (перечень подлежащих разработке вопросов)
\plannedcontents{\begin{enumerate}
	\item Обоснование важности установления зависимости между спектром и эффективностью для исследования
	и отбора МППЧ-кодов;
	\item Разработка и реализация итеративного декодера для быстрой оценки эффективности кода посредством
	моделирования;
	\item Разработка и реализация алгоритма подсчета спектра графа Таннера;
	\item Описание плана исследования. Порядок отбора кодов для проведения тестирования;
	\item Результаты исследования.
\end{enumerate}}

%%% Исходные материалы и пособия 
\plannedsources{\begin{enumerate}
    \item Б.Д.Кудряшов. Основы теории кодирования;
    \item М.Холл. Комбинаторика;
    \item D.J.C.MacKay. Encyclopedia of Sparse Graph Codes. http://www.inference.phy.cam.ac.uk/mackay/codes/data.html.
\end{enumerate}}

%%% Календарный план
\addstage{Ознакомление с основами теории кодирования}{12.2015}
\addstage{Ознакомление с имеющимся набором программ для исследования и отбора МППЧ-кодов}{05.2016}
\addstage{Ознакомление с существующими итеративными декодерами}{07.2016}
\addstage{Разработка и реализация итеративного декодера заточенного под нужды исследования}{09.2016}
\addstage{Ознакомление с существующими подходами подсчета спектра кода}{11.2016}
\addstage{Разработка и реализация алгоритма подсчета спектра графа Таннера МППЧ-кода}{12.2016}
\addstage{Проведение исследования зависимости эффективности кода от спектра}{03.2017}
\addstage{Написание пояснительной записки}{05.2017}

%%% Цель исследования
\researchaim{Установить существование критерия оценки эффективности МППЧ-кода, основанного
 на спектре графа Таннера.}

%%% Задачи, решаемые в ВКР
\researchtargets{\begin{enumerate}
    \item разработка и реализация эффективного алгоритма вычисления спектра графа Таннера;
    \item проведение исследования зависимости эффективности кода от спектра графа Таннера;
    \item формулировка критерия оценки эффективности МППЧ-кода.
\end{enumerate}}

%%% Использование современных пакетов компьютерных программ и технологий
\advancedtechnologyusage{C++, CUDA С --- для создания итеративного декодера. Python и zsh скрипты для
автоматизация исследования. Java --- алгоритм подсчета спектра. Python, matplotlib, pandas ---
обработка и визуализация результатов. \LaTeX, Git.}

%%% Краткая характеристика полученных результатов 
\researchsummary{В результате была продемонстрирована зависимость между спектром графа Таннера и
эффективностью кода. Разработан вычислительно эффективный алгоритм подсчета спектра графа Таннера.
 Результаты могут быть использованы для ускорения поиска эффективных МППЧ-кодов.}

%%% Гранты, полученные при выполнении работы 
\researchfunding{Грантов или других форм государственной поддержи и субсидирования
 в процессе работы не предусматривалось.}

%%% Наличие публикаций и выступлений на конференциях по теме выпускной работы
\researchpublications{
\begin{refsection}
\nocite{kmu-russian,kmu-english}
\printannobibliography
\end{refsection}
}

%% Эта команда генерирует титульный лист и аннотацию.
\maketitle{Магистр}

%% Оглавление
\tableofcontents

% -*-coding: utf-8-*-
\startprefacepage

Основной целью данной работы является разработка эффективного метода построения маршрутов поиска
объектов согласно общепринятым стратегиям поиска. Метод должен работать с большим количеством
различных моделей изменения распределения вероятности обнаружения объекта со временем. %Цель
 
 Существующие подходы строят весь маршрут исходя из данных в начальный момент
времени --- информации о начальном распределении и модели его изменения.
Таким образом главным недостатком существующих подходов является необходимость
разработки нового алгоритма для каждой модели изменения распределения.
Учитывая тот факт, что подобрать правильную модель, хорошо приближающую реальность,
 крайне непросто, возникает необходимость разработки алгоритма,
 работающего единообразно на широком классе различных моделей.
 Основная идея рассматриваемого подхода --- использование симулятора для получения %Новизна
 информации о распределении в любой момент времени при построении маршрута.
 Таким образом при планировании пользователь в первую очередь выберет модель,
 как можно лучше приближающую реальность в данном случае, а после запустит
 алгоритм построения маршрута.

 Таким образом задачами данной работы являются разработка инструмента симуляции изменения
вероятности во время прохождения маршрута и алгоритма построения маршрута при заданных условиях.
В данной работе будет рассмотрен алгоритм построения маршрута согласно
 стратегии ``Параллельное галсирование''.

Расчетная задача может быть использована оператором как вспомогательное средство
при планировании маршрутов поиска потерпевших бедствие объектов в реальной жизни.
Визуализированный процесс изменения распределения может пригодиться при решении какие области
следует исследовать более подробно, а какие можно пропустить.
%Практическое значение

Концепция данной задачи был продемонстрирована на одной из выставок. К задаче был проявлен 
интерес, и было решено внедрить ее в комплекс расчетных морских задач. %Актуальность

В главе 1 решаемая задача будет рассмотрена более подробно.
 Описаны классы маршрутов, получаемые при использовании стратегии ``Параллельное галсирование''. 
Будут приведены особенности задачи, которые отличают ее от классической задачи коммивояжера
 и делают невозможным использование ранее разработанных методов решения
 для решения исходной задачи в общем случае.

В главе 2 будут рассмотрены вопросы, связанные с разработкой симулятора на CUDA.
 Обозначены предоставляемые симулятором сервисы.

В главе 3 будет описан алгоритм построения маршрутов согласно стратегии ``Параллельное галсирование''.
В первую очередь будет рассмотрен подход с динамическим программирования для определения
маршрута при фиксированном распределении. Далее будет рассмотрена корректировка маршрута
согласно изменениям распределения.

В главе 4 будет осуществлено сравнение с существующим алгоритмом ``Кронштадт технологии''.
\FloatBarrier

\chapter{Общие сведения}

\section{Линейные коды}

\definition{\textit{Линейным двоичным $(n,k)$---кодом} называется любое $k$---мерное подпространство пространства
всевозможных двоичных векторов длины $n$.}

\definition{Отношение $R=k/n$ называется \textit{скоростью линейного $(n,k)$ кода}.}

\definition{\textit{Порождающей матрицей линейного $(n,k)$---кода} называется матрица 
размера $k\times n$, строки которой его базисные вектора.}

\example{
\[
G=
\begin{pmatrix}
	1 & 0 & 1 \\
	1 & 0 & 0
\end{pmatrix}
\]

задает код ${000, 101, 100, 001}$.
}

\definition{Двоичный вектор $\bs{h}$ длины $n$ для которого все кодовые слова некоторого $(n,k)$ кода
 $C=\{\bs{c}_i\},i=0,\ldots,2^k-1$ удовлетворяют тождеству
\[
(\bs{c}_i,\bs{h})=0, i=0,\ldots,2^k-1
\]
называется \textit{проверкой} по отношению к коду $C$.}

Заметим, что предыдущее определение проверки эквивалентно
\[
G \cdot \bs{h}^T=0
\]
так как для выполнения тождества для любого кодового слова достаточно выполнения тождества для 
базисных векторов.

\definition{Пространство проверок называется пространством, ортогональным линейному коду, или \textit{проверочным
пространством}.}

\theorem{Размерность проверочного пространства линейного $(n,k)$---кода равна $r=n-k$.}

\definition{\textit{Проверочной матрицей линейного $(n,k$)---кода} называется матрица размера $r\times n$,
строки которой составляют базис проверочного пространства}

Для проверочной и порождающей матрицы выполнено следующее соотношение
\[
G \cdot H^T=0
\]

Если же принятая последовательность $\bs{y}$ из-за шума в канале перестала быть кодовым
словом то соответсвенно
\[
\bs{s}=\bs{y}\cdot H^T \neq 0
\]
и \bs{s} называется \textit{синдромом} вектора \bs{y}, неравенство нулевому вектору которого
указывает на ошибки в принятой последовательности \bs{y}.

\section{МППЧ-коды}

Линейный код может быть задан проверочной матрицей $H$.

Галлагер \cite{gallager} предложил идею выбора матрицы $H$ разряженной (с малой плотностью) для уменьшения
 сложности кодирования и декодирования: в матрице должно быть мало единиц, строки и столбцы не должны 
 иметь большое число  общих элементов. Он также подкрепил свою идею анализом с использованием 
 метода случайного декодирования.
 
 Для интуитивного понимания почему малое число единиц приводит к более эффективному декодированию следует
 заметить, что в случае когда строки проверок мало между собой зависят, декодирование может производиться 
 методом проб и ошибок, пытаясь подобрать последовательность символов, исправление которых будет уменьшать
 вес синдрома, с каждой следующей попыткой.
 
 Используемый алгоритм декодирования, описанный далее, существенно опирается на факт, что влияние конкретного
 столбца на синдром не сильно зависит от остальных столбцов.
 


\definition{МППЧ-код называется \textit{$(J,K)$ регулярным} если его проверочная матрица $H$ содержит ровно
$J$ единиц в каждом столбце и  ровно $K$ единиц в каждой строке. Иначе МППЧ-код называется \textit{иррегулярным}.}

\section{Квазициклические МППЧ-коды}

$P^i_M$ --- квадратная матрица порядка $M$, полученная из единичной сдвигом строк вправо $i$ раз. Например:
\[
P^0_3=
\begin{pmatrix}
	1 & 0 & 0 \\
	0 & 1 & 0 \\
	0 & 0 & 1
\end{pmatrix}
\quad
P^1_3=
\begin{pmatrix}
	0 & 1 & 0 \\
	0 & 0 & 1 \\
	1 & 0 & 0
\end{pmatrix}
\quad
P^2_3=
\begin{pmatrix}
	0 & 0 & 1 \\
	1 & 0 & 0 \\
	0 & 1 & 0
\end{pmatrix}
\] 

Матрицы $P^i_M$ содержат одинаковые элементы на всех диагоналях параллельных главной --- такие матрицы
называются \textit{циркулянтными матрицами} или \textit{циркулянтами}.

МППЧ-код называется квазициклическим если его проверочная матрица может быть представлена в виде блочной
матрицы из блоков циркулянтов единичной матрицы.

Для построения б

\begin{figure}[h!]
\centering
\begin{subfigure}{0.2\textwidth}
  \centering
  \scalebox{1.2}{\inputTikZ{liftingBase}}
  \caption{Протограф}
  \label{liftingBase}
\end{subfigure}%
\begin{subfigure}{.8\textwidth}
  \centering
  \scalebox{1.2}{\inputTikZ{liftingExpanded}}
  \caption{Расширенный граф}
  \label{liftingExpanded}
\end{subfigure}
\caption{Пример лифтинга}
\end{figure}

\section{Граф Таннера}


\newcommand\colorBox[2]{\setlength{\fboxsep}{2pt}\colorbox{#1!10}{#2}}

\begin{figure}[h!]
\centering
\begin{subfigure}{.3\textwidth}
  \centering
  \[
    \begin{blockarray}{ccccc}
        \colorBox{red}{1} & \colorBox{red}{2} & \colorBox{red}{3} & \colorBox{red}{4} \\
        \begin{block}{(cccc)c}
            1&1&0&1&\colorBox{green}{5}\\
            1&1&1&0&\colorBox{green}{6}\\
            1&0&1&1&\colorBox{green}{7} \\
        \end{block}
    \end{blockarray}
  \]
  \caption{Проверочная матрица $H$}
  \label{checkMatrix}
\end{subfigure}%
\begin{subfigure}{.7\textwidth}
  \centering
   \scalebox{1.2}{\inputTikZ{ex_graph1}}
  \caption{Граф Таннера}
  \label{checkMatrix}
\end{subfigure}
\caption{Пример графа Таннера}
\end{figure}























































\chapter{Декодер}

Пример ссылок на литературные источники: \cite{hall-combinatorics,kudryashov-codingtheory,finding-and-counting-given-length-cycles,counting-short-cycles-of-quasi-cyclic-protograph-ldpc-codes,message-passing-algorithm-for-counting-short-cycles-in-graph,how-to-find-long-paths-efficiently,color-coding,algorithm-for-counting-for-counting-short-cycles-in-bipartite-graphs,on-the-number-of-cycles-in-a-graph,understanding-belief-propogation,mackay-codes}.


Проведение экспериментального исследования зависимости эффективности итеративного декодирования от некоторого
критерия подразумевает непосредственное моделирование передачи информации через канал с шумом посредством 
кодирования информации определенным МППЧ-кодом и последующим декодированием с помощью итеративного декодера.

Критерий, основанный на спектре циклов графа Таннера, побуждает к проведению исследования на иррегулярных
МППЧ-кодах. Еще одним фактором не в пользу скорости декодирования является размер кода --- который
должен быть выбран в соотвествии с длинами кодов, используемыми на практике, которые достаточно велики.
Кроме того, количество переданных кодовых слов при моделировании должны быть достаточно большим, чтобы оценка
отношения ошибочно-переданных блоком к общему числу переданных блоков (FER - frame error rate) была достаточно точна.

Моделирование передачи достаточного количества кодовых слов занимает минуты на современных 
компьютерах, используя наивную реализацию декодера на CPU или реализацию из доступных библиотек 
(таких как например реализация из Communication System Toolbox в Matlab). 
С учетом того, что каждый код должен быть протестирован на десятках различных отношений сигнал-шум (SNR ---
signal noise ratio), подход с использованием наивной реализации при имеющихся вычислительных ресурсах займет
недопустимо большое количество времени. 

Альтернативным подходом являются реализации итеративных декодеров с использованием
GPU.
Многие из доступных реализаций заточены под МППЧ-коды из стандартов. Большая часть использует параллелизм
на уровне декодирования одного кодового слова, что позволяет ускорить время между получением 
битовой последовательности и декодированного кодового слова
\cite{stressing-the-ber-simulation-of-ldpc-codes-in-the-error-floor-region-using-gpu-clusters}, 
но не использует тот факт что при оценке вероятности ошибки на блок передается большое количество кодовых
слов. В данном случае значительно более существенный выйгрыш позволяет получить параллелизм
на уровне набора кодовых слов, где вычислительные узлы разделены на группы, каждая из которых занимается
декодированием различных кодовых слов
\cite{opencl-cuda-algorithms-for-parallel-decoding-of-any-irregular-ldpc-code-using-gpu}.

К сожалению, итеративного GPU декодера, оптимизированного для получения оценки вероятности ошибки на блок,
не было найдено в открытых источниках сети Internet. Что побудило к созданию простого итеративного декодера
с использованием архитектуры CUDA, который позволил получить 60-70 кратное ускорение при моделировании
передачи большого числа кодовых слов иррегулярного МППЧ-кода.

\section{Количество кодовых слов при моделировании}

В первую очередь следует провести анализ количества кодовых слов, которое следует передать, чтобы 
статистически достоверно оценить вероятность ошибки на блок.

Пусть $P$ --- вероятность ошибки на блок и передача производилась до достижения $k$ ошибочно переданных блоков.
Тогда суммарно передано $N\approx \frac{k}{P}$ кодовых слов. Оценим дисперсию среднего арифметического при 
$N$ опытах. $X_i$ --- индикаторная случайная величина ошибки при передаче $i$-го кодового слова. 
\newcommand{\Expect}{\mathsf{E}}
\[
	\begin{split}
	\Variance \left({\frac{\sum_i{X_i}}{N}}\right) & =
	\Expect \left(\frac{\sum_i{X_i}}{N}\right)^2-\left(\Expect \left(\frac{\sum_i{X_i}}{N}\right)\right)^2 \\
	& = \frac{\Expect(\sum_i{X_i})^2}{N^2})-\left(\frac{\sum_i{\Expect(X_i)}}{N}\right)^2 \\
	& = \frac{\Expect(\sum_i{X_i})^2}{N^2})-\left(\frac{N \cdot P}{N}\right)^2 \\
	& = \frac{\sum_{i \neq j}{\Expect(X_i \cdot X_j)} + \sum_i{\Expect(X_i^2)}}{N^2}-P^2 \\
	& = \frac{N \cdot (N - 1) \cdot P^2 + N \cdot P}{N^2}-P^2 \\
	& = \frac{-N \cdot P^2 + N \cdot P}{N^2} \\
	& = \frac{P \cdot (1 - P)}{N}  = \frac{P^2 \cdot (1 - P)}{k}
	\end{split}
\]

Применяя Гауссовскую аппрокимацию биномиального распределения, можем говорить что пределы за $3\cdot \sigma$
маловероятны (вероятность порядка 0.001). Поэтому погрешность находится в пределах (учитывая что в реальности
$P \ll 1$):
\[
\pm \delta = 3 \cdot P \sqrt{\frac{1 - P}{k}} \approx P \cdot \frac{3}{\sqrt{k}} 
\]	 

Соответственно, например при $k=50$ получаем относительную погрешность $\approx 42\%$, 
а при $k=100$ --- $\approx 30\%$. При увеличении $k$ погрешность убывает обратно пропорционально корню.
Для проведения дальнейших исследований остановимся на $k=100$.

\chapter{Алгоритм подсчета спектра графа Таннера}

Эта задача порождена проблемой анализа и оптимизации МППЧ кодов. 
Пусть $B$  -- (двоичная) базовая матрица кода.
Для нее определен двудольный граф Таннера $T=\{V, E \}$ с множеством вершин 
$V= V_s \cup V_c$, где  $V_s$ и  $V_c$ -- множества символьных и проверочных вершин, 
соответственно. Единицам  матрицы $B$ соответствуют ребра графа $T$. 


\begin{example} \label{ex1}
Рассмотрим  базовую матрицу $B$ 
\begin{eqnarray}%{C} %\label{H0}
B(D)&=&
\begin{pmatrix}
1&1&0&   1\\
1&1&1& 0\\
1&0         & 1& 1
\end{pmatrix},       
\end{eqnarray}
для задающей квазициклический МППЧ код полиномиальной проверочной матрицы 
\begin{eqnarray}%{C} %\label{H0}
H(D)&=&
\begin{pmatrix}
D^{w_{11}}&D^{w_{12}}&0&   D^{w_{14}}\\
D^{w_{21}}&D^{w_{22}}& D^{w_{23}}& 0\\
D^{w_{31}}&0         &  D^{w_{33}}& D^{w_{34}}
\end{pmatrix}                          \label{H0}                                                                                                                  
\end{eqnarray}
Для удобства переназначим веса переходов
\begin{equation} \label{H}
H(D)=
\begin{pmatrix}
D^{w_{1}}&D^{w_{4}}&0                   &   D^{w_{8}}\\
D^{w_{2}}&D^{w_{5}}& D^{w_{6}}& 0\\
D^{w_{3}}&0                &  D^{w_{7}}& D^{w_{9}}
\end{pmatrix}                                                                                                                                           
\end{equation}
Соответствующая матрица инцидентности графа Таннера 
\[
\arraycolsep=1.5pt \def\arraystretch{1}
T(D)=
\begin{pmatrix}
1   &        1 &  1& 0 &0 & 0 &   0 &  0 &  0 \\
0   &        0 &  0& 1 &1 & 0 &   0 &  0 &  0 \\
0   &        0 &  0& 0&0 & 1 &   1 &  0 &  0 \\
0   &        0 &  0& 0 &0 & 0 &   0 &  1 &  1 \\
D^{w_{1}}&   0   &  0  & D^{w_{4}}&0 &  0 & 0                   &   D^{w_{8}} & 0\\
0          &D^{w_{2}} &0  &0&D^{w_{5}}&  D^{w_{6}}          & 0          & 0&0\\
0 & 0& D^{w_{3}}      &0         &0      &  0   &  D^{w_{7}}& 0 &D^{w_{9}}
\end{pmatrix}                                                                                                                                           
\]
Сам граф показан на рис. \ref{fig1}. 
Кружками и квадратами показаны символьные и проверочные узлы. 
\begin{figure}[!h]
  \centering
  \inputTikZ{ex_graph}
   \caption{Граф Таннера для кода из примера \ref{ex1}}
  \label{fig1}
\end{figure}

\end{example}

Задача состоит в подсчете числа циклов заданной длины в расширенном 
графе с заданной степенью расширения $M$. Длина цикла равна числу переходов в 
пути, начинающемся и заканчивающемся  в одном и том же узле и таком, что 
сумма весов переходов по модулю $M$ равна нулю. 
Веса суммируются с учетом знаков, зависящих от направления 
перехода (см. рис. \ref{fig1}).

Такой тип циклов часто называют замкнутым обходом, однако при подсчете
рассматриваемых объектов необходимо учесть следующие дополнительные ограничения.
\begin{itemize}
\item
Запрещено двигаться обратно по последнему пройденному ребру. Например,
путь  6$\to$1$\to$6 веса 0 запрещен.
\item
Циклические сдвиги путей и инверсии путей должны  учитываться как один цикл. Например, путь
$p=6\to 1 \to 7\to 3 \to 6$  образует цикл при условии $w_2-w_6+w_7-w_3=0$. При этом пути
$6 \to 3 \to 7\to 1 \to 6$ и  $1\to 7 \to 3\to 6 \to 1$ тоже циклы, но они уже не вносят
вклад в число циклов длины 4,
если цикл $p$ учтен. 
\end{itemize}

Хотелось бы применить стандартные методы, используемые при анализе систем на основе 
конечных автоматов. Данный граф не является конечным автоматом, поскольку перемещение
из состояния в состояние зависит от предыдущего состояния.  Например, на рис. \ref{fig1}
после состояния 4 возможно только 5, если предыдущим было 7 и, наоборот,
только 7, если предыдущим было 5. 

Чтобы свести задачу к анализу конечных автоматов, введем новое множество состояний 
$U=\{e,\xi\}$, где  $e$ задает ребро исходного графа, а $\xi$ -- направление перехода. 
Сокращенно будем записывать пары в виде $+e$  и $-e$.   
 
Из графа на рис. \ref{fig1} получится  граф с 18 состояниями $\{ \pm 1, \pm 2,..., \pm 9\}$.
Заметим, однако, что после отрицательного  ребра следуют только положительные и после 
положительного отрицательные. Это позволит записать матрицу переходов компактно 
в виде двух матриц, матрицы положительных и матрицы отрицательных переходов.
Например, следующими состояниями (ребрами графа Таннера) после положительного перехода 
$+1$  возможны отрицательные $-4, -8$. После отрицательного перехода
$-1$  возможны положительные $2,3$. 

В нашем примере две матрицы переходов  имеют вид
 \[
 \setlength{\arraycolsep}{2.pt}
A_{-}=
\begin{pmatrix}
  0           &0          & 0     &D^{-w_4}& 0              & 0&0 & D^{-w_8}&0 \\
   0          & 0        & 0     & 0    &D^{-w_5}&D^{-w_6}&   0 &    0 &  0  \\ 
    0         &  0       &  0    & 0    &   0  &   0  &D^{-w_7}&   0  &D^{-w_9}\\ 
D^{-w_1}&  0       &    0  &   0  &   0  &   0  & 0   &D^{-w_8}&  0  \\ 
  0          &D^{-w_2}&  0    & 0    &  0   &D^{-w_6}&  0  & 0    &0    \\ 
   0         &D^{-w_2}&   0   &  0   & D^{-w_5}&   0  &  0  &  0   & 0   \\ 
   0         &0         &D^{-w_3}& 0    &  0   &   0  &0    &0     &D^{-w_9} \\ 
D^{-w_1}&0         &   0   &D^{-w_4}&  0   & 0    & 0   &   0  &0    \\ 
   0        & 0        &D^{-w_3} &   0  &    0 &  0   &D^{-w_7}    & 0    &0     
               \end{pmatrix}
\]
\[
A_{+}=
\begin{pmatrix}
0              &D^{w_2}&D^{w_3}&  0  &      0         & 0&0 &0 &0 \\
D^{w_1}& 0              &D^{w_3}&0    &      0         &0 &0 &0 &0 \\
D^{w_1}&D^{w_2}&0               & 0   &       0        &0 &0 &0 &0 \\
0              & 0             &0                &0    &D^{w_5}&0 &0 & 0& 0 \\
 0             &  0            &0                & D^{w_4}& 0 &0 &0  &0 & 0 \\
  0            &   0           & 0               &  0& 0 & 0&D^{w_7}  &0 &0  \\
   0           &    0          &  0              & 0 & 0 &D^{w_6}&0  &0 &0  \\
    0          &      0        &   0             & 0 & 0 & 0& 0& 0&  D^{w_9} \\
     0         &     0         &    0            & 0 & 0&0 & 0&  D^{w_8} &0
\end{pmatrix}
\]

Заметим, что все циклы имеют четную длину и состоят из пар переходов
(по стрелке, против стрелки). Можно построить матрицу переходов за
два шага, как произведение матриц $A_+, A_-$
\[
A=A_+A_{-}=\begin{pmatrix}
     0    & 0 &    0 &    0 &    \omega_{45}   &  0 &    0 &    0 &    \omega_{89}\\
     0&     0&     0&     \omega_{54} &    0  &   0  &   \omega_{67}  &   0  &   0 \\
     0&     0&     0&     0&     0&     \omega_{76} &     0&      \omega_{98}   &   0\\
     0&     \omega_{12}      & \omega_{13}  &    0 &    0  &   0 &    0  &   0      & \omega_{89} \\
\omega_{21}     &0     & \omega_{23}     &0   &0   &0     & \omega_{67}     &0 &   0\\
\omega_{21}     &0     & \omega_{23}      & \omega_{54}    &0   &0   &0   &0   & 0\\
\omega_{31}     & \omega_{32}  &0    &0   &0   &0   &0     & \omega_{98}     & 0\\
    0     & \omega_{12}      & \omega_{13}     &0    & \omega_{45}     &0   &0   &0   & 0\\
\omega_{31}      & \omega_{32}     &0   &0   &0     & \omega_{76}     &0   &0   & 0\\
\end{pmatrix}
\]
где
$\omega_{ij}=D^{-w_i+w_j}$.
%
%     0     0     0     0     1     0     0     0     1
%     0     0     0     1     0     0     1     0     0
%     0     0     0     0     0     1     0     1     0
%     0     1     1     0     0     0     0     0     1
%     1     0     1     0     0     0     1     0     0
%     1     0     1     1     0     0     0     0     0
%     1     1     0     0     0     0     0     1     0
%     0     1     1     0     1     0     0     0     0
%     1     1     0     0     0     1     0     0     0

В общем случае матрица переходов за 2 шага будет иметь размер, равный числу единиц в 
базовой матрице $B$. 

Для подсчета производящей функции числа путей длины $2L$, начинающихся с ребра 1 в положительном направлении
нужно начальный вектор $a_0=(1,0,...,0)$ умножить на $A^L$. Для нашего примера при
 \begin{equation}
\bs a_2=\bs a_0 A =
\begin{pmatrix}
     0    & 0 &    0 &    0 &    \omega_{45}   &  0 &    0 &    0 &    \omega_{89}
 \end{pmatrix}
\end{equation} 
\begin{equation}\label{eq4}
 \bs a_4=\bs a_2A=
 \begin{pmatrix}    
    \omega_{4521,8931}  &\omega_{8932} & \omega_{4523 } &0 &    0 &    \omega_{8976}   & \omega_{4567 }&    0 &    0 
 \end{pmatrix}
\end{equation} 
где первая компонента является сокращенной записью полинома
\[
 \omega_{4521,8931}=D^{-w_4+w_5-w_2+w_1}+D^{-w_8+w_9-w_3+w_1}
\]
Кроме того необходимо отметить:
\begin{itemize}
\item 
Соблюдение условия неповторения ребра на стыке цикла также соблюдается.
\item
Хотя мы пишем $+, -$  в выражениях типа $D^{-w_4+w_5-w_2+w_1}$,
$w_i$ не коммутируют между собой так как неупорядоченный набор ребер не задает однозначно цикл (\ref{ex2}),
 таким образом до подстановки конкретных значений $w_i$ не могут быть сложены.
\end{itemize}

\begin{figure}[!h]
  \centering
  \inputTikZ{ex_graph2}
   \caption{Граф Таннера для кода из примеров \ref{ex2}, \ref{ex3}}
  \label{fig1}
\end{figure}

\begin{example} \label{ex2}
  Обозначим пути $c_1=1 \to -4 \to 5 \to -2$, $c_2=1 \to -4 \to 6 \to -3$, $c_3=2 \to -5 \to 6 \to -3$.
 Обратные к этим путям по направлению соответственно $c_{-1}, c_{-2}, c_{-3}$.
 Тогда набор ребер как объединением $\{c_1,c_{-2},c_{-3}\}$ может обозначать два пути -- $c_1c_{-3}c_{-2}$ и 
$c_1c_{-2}c_{-3}$, которые не могут быть получены друг из друга с помощью циклического сдвига и инверсии.
\end{example}

Из (\ref{eq4}) видим, что существуют 2 потенциальных цикла длины 4. 
Цикл образуется в том случае, когда сумма в показателе степени равна нулю. 
Например, если $-w_4+w_5-w_2+w_1=0$, а  $-w_8+w_9-w_3+w_1\neq 0$  то 
\[
a_{41}(D)=1+D^{-w_8+w_9-w_3+w_1}, 
\]
и $a_{41}(0)=1$.
Пусть $a_{2L,i}^{j}$ -- обозначает коэффициент при $D^j$ в $i$-ой компоненте $a_{2L}$.

(здесь и далее считаем, что $D^0=1$, несмотря на то что $D$ может быть $0$. Таким образом
свободный член полинома равен количеству циклов веса $0$ без учета эквивалентных циклов)
В общем случае все циклы длины $2L$, проходящие через ребро +1 учтены в $a_{2L,1}(0)$, однако
некоторые из них могут быть учтены многократно (\ref{ex3}). Кроме того необходимо аналогично учесть
циклы, проходящие через ребро +2, +3 и так далее. Такое суммарное рассмотрение неизбежно
будет учитывать циклы повторно, однако количество повторений не обязательно равно длине цикла и
зависит от его состава.
\begin{example} \label{ex3}

 Пути $1 \to -4 \to 5 \to -2 \to 1 \to -4 \to 6 \to -3$ и $1 \to -4 \to 6 \to -3 \to 1 \to -4 \to 5 \to -2$,
начинающиеся с $1$ эквивалентны, так как являются циклическим сдвигом друг друга, но будут учтены дважды. 
\end{example}

Дальнейшие рассуждения можно проводить двумя способами:
\begin{itemize}
  \item
  Рассматривать $w_i$ в качестве символьных переменных. 
Можно выписать все возможные комбинации весов, приводящие к потенциальным 
циклам длины $2L$ и затем вычислять спектры циклов, подставляя разные 
наборы разметок в полученные уравнения. Число циклов равно числу нулей в уравнениях.  
\item
  При заданных (фиксированных) весах ребер подсчитать спектр длин циклов в заданном диапазоне.
\end{itemize}

\section{Веса как символьные переменные}
В случае рассмотрения $w_i$ как символьных переменных необходимо избавиться от эквивалентных
относительно сдвига и инверсии комбинаций. Для этого каждый путь приводится к минимальному 
лексикографическому виду после чего убираются дубликаты. 

Заметим, что цикл в обратном направлении
к рассматриваемому циклу не может быть эквивалентен относительно сдвига исходному.
Обозначим $\bar{p}$ путь в обратном направлении к пути $p=e_1,e_2,...,e_n$. 
Предположим $p$ и $\bar{p}$ эквивалентны относительно сдвига -- тогда найдется индекс $i$, такой что
$\bar{e}_i, \bar{e}_{i-1}, ..., \bar{e}_1 = e_1, e_2, ..., e_i$.  Если $i$ нечетно, тогда $\bar{e}_{(i + 1)/2} = e_{(i+1)/2}$ -- противоречие.
 Если $i$ четно, то $e_{i/2 + 1}=\bar{e}_{i/2}$, что противоречит ограничению что путь не может
идти обратно по последнему пройденному ребру.

Таким образом каждому циклу соответствует ровно два пути в разных направлениях. Непосредственно
инвертируя каждый путь несложно оставить ровно один из каждой пары.

По причине того, что число различных путей растет
экспоненциально при построении всех возможных символьных комбинаций, лучше воспользоваться подходом
meet-in-the-middle. Это позволяет для нахождения всех возможных циклов длины $2L$ рассматривать
только пути длины $L$, в то время как описанный алгоритм рассматривает пути длины $2L$.

\section{Фиксированные веса}
В случае когда веса изначально зафиксированы можно считать, что веса коммутируют, поэтому достаточно
считать не более $M$ членов в каждом полиноме при умножении на матрицу $A$,
 по одному значению для каждой возможной суммы весов. После чего устранить дубликаты из результата с 
помощью обращения Мебиуса.

Рассмотрим для цикла $p$ сколько раз он был учтен. Порядком цикла назовем максимальное $r$, такое что
$p$ можно представить как 
\[
p=\underbrace{ss \ldots s}_{r \text{ раз}}
\]
Периодом цикла $p$ назовем длину $|s|=\frac{|p|}{r}$. Таким образом цикл $p$ 
периода $l$ имеет $l$ различных циклических сдвигов -- следовательно будет учтен $2l$ раз, с учетом
последовательностей в обратном порядке.

Последовательно для каждого $i$ зафиксируем
\[
a_0=(\underbrace{0,0,...,0}_{i \text{ раз}},1,0,...,0)
\]
и вычислим
\[
a_{2L}=a_0A^L
\]

проводя все вычисления в кольце многочленов по модулю $D^M$.

Запомним количество циклов (всех возможных весов) проходящих через +i, содержащееся в 
многочлене $a_{2L,i}$ как $b_{2L,i}$.

Суммируя по всем возможным $a_0$ получаем 
\[
b_{2L}=\sum_{\substack{i}}b_{2L,i}
\]
 - многочлен, c коэффициентами при $D^w$ соответсвенно равными количеству циклов длины $2L$ и веса $w$,
 где каждый цикл учтен дважды столько, сколько он имеет различных циклических
 сдвигов (в обоих направлениях).  

Далее временно забудем об ограничении инверсии, так как для устранения путей эквивалентных
относительно разворота достаточно разделить результирующий спектр на два.

Обозначим за $g(l)$ -- многочлен, коэффициент при $D^w$ которого равен числу циклов длины и периода $l$ веса $w$.

\begin{definition} \label{def1}
Введем операцию $T_d(p(D))$ где $p(D)$ -- многочлен, а $d$ -- натуральное число как:
\[
  T_d(c_0+c_1D+c_2D^2+...+c_nD^n) = c_0+c_1D^{d}+c_2D^{2d}+...+c_nD^{nd} \pmod{D^M}
\]
\end{definition}

Тогда при фиксированной длине $L$ имеем равенство:
\[
  \sum_{\substack{d | L}}d \cdot T_{\frac{L}{d}}(g(d)) = b_L \label{eqWithT}
\]

Равенство справедливо, так как каждый цикл длины $L$ периода $d$ веса $W$ состоит из повторенного $\frac{L}{d}$ раз
цикла длины и периода $d$ веса $w$, такого что $w\cdot \frac{L}{d} = W \pmod M$ и однозначно им задается.
Каждый такой цикл имеет $d$ различных циклических сдвигов, поэтому учтен в $b_L$ $d$ раз.
Таким образом суммирование ведется по всем возможных длинам периодов, после чего благодаря $T_{\frac{L}{d}}$
вес каждого цикла из $g(d)$ домножается на число повторений цикла для достижения длины $L$ и каждый
из циклов учитывается с коэффициентов $d$ так как входит в $b_L$ в виде $d$ различных линейных 
последовательностей ребер.

В равенствах вида
\[
  f(n)=\sum_{\substack{d|n}}g(d)
\]

$g(d)$ может быть выражено с помощью формулы обращения Мебиуса \cite{hall-combinatorics}:
\[
  g(n) = \sum_{\substack{d|n}}\mu(d)f(n/d)
\]

где\[
  n = p_1^{e_1} \cdot p_2^{e_2} \cdot \ldots p_r^{e_r}
\]
- разложение $n$ на простые множители

\[
  \mu(n)=
\left\{
  \begin{array}{lll}
    1 & \mbox{при } n=1, \\
    0 & \mbox{если } \exists i \quad e_i > 1, \\
    (-1)^r & \mbox{если } e_1 = e_2 = \ldots = e_r = 1
  \end{array}
\right.
\]

В рассматриваемом случае можно ввести и использовать следующее обобщение формулы Мебиуса:

\begin{theorem} \label{th1}
Если
\[
  f(n) = \sum_{\substack{d|n}}T_{\frac{n}{d}}(g(d))
\]
где $T_{k}$ удовлетворяет свойствам:
\begin{eqnarray}
T_k(c \cdot p(D)) = c \cdot T_k(p(D)) \label{eq1}\\
T_k(p(D)+q(D)) = T_k(p(D))+T_k(q(D))\label{eq2}\\
T_{k_1}(T_{k_2}(p(D))) = T_{k_1\cdot k_2}(p(D)) \label{eq3}\\
T_1(p(D)) = p(D) \label{eq4}\
\end{eqnarray}
Тогда $g(d)$ может быть выражено:
\[
  g(n) = \sum_{\substack{d|n}}\mu(d)T_d(f(n/d))
\]
\end{theorem}

\begin{proof}
  
\begin{lemma}
\[
  \sum_{\substack{d|n}}\mu(d)=0 \quad \text{при } n > 1
\]
\end{lemma}


\begin{proof}
  \[
  n = p_1^{e_1} \cdot p_2^{e_2} \cdot \ldots p_r^{e_r}
  \]

\[
n^{\ast}=p_1 \cdot p_2 \ldots p_r
\]

\[
\sum_{\substack{d|n}}\mu(d)=\sum_{\substack{d|n^{\ast}}}\mu(d)
\]
 так как любое $\mu(d)$ где $p_i^2|d$ равно $0$.
 
\[
\sum_{\substack{d|n^{\ast}}}\mu(d)=1-\binom{r}{1}+\binom{r}{2}-\dotsc+(-1)^k \cdot \binom{r}{k}+ \dotsc = (1-1)^r=0
\]
так как существует $\binom{r}{k}$ делителей $n^{\ast}$ состоящих из $k$ простых множителей,
каждый из которых внесет вклад $\mu(d)=(-1)^k$.
\end{proof}


\[
  \sum_{\substack{d|n}}\mu(d) \cdot T_d(f(n/d)) = 
\]
\[
  \sum_{\substack{d|n}}\mu(d) \cdot T_d(\sum_{\substack{d'|\frac{n}{d}}}T_{\frac{n}{dd`}}(g(d'))) =
\]
\[
  \sum_{\substack{d|n}}\mu(d) \cdot \sum_{\substack{d'|\frac{n}{d}}}T_d(T_{\frac{n}{dd`}}(g(d'))) =
\]
\[
  \sum_{\substack{d|n}}\mu(d) \cdot \sum_{\substack{d'|\frac{n}{d}}}T_{\frac{n}{d`}}(g(d')) =
\]
\[
  \sum_{\substack{d|n}}\sum_{\substack{d'|\frac{n}{d}}}\mu(d) \cdot T_{\frac{n}{d`}}(g(d')) =
\]
\[
  \sum_{\substack{d'|n}}\sum_{\substack{d|\frac{n}{d'}}}\mu(d) \cdot T_{\frac{n}{d`}}(g(d')) =
\]
\[
  \sum_{\substack{d'|n}}T_{\frac{n}{d`}}(g(d')) \cdot \sum_{\substack{d|\frac{n}{d'}}}\mu(d)  =
\]
\[
  T_{\frac{n}{n}}(g(n))  = T_1(g(n)) = g(n)
\]
\end{proof}

Нетрудно заметить что операция $T_d(p(D))$ из определения \ref{def1} удовлетворяет ограничениям
из теоремы \ref{th1}. Действительно, первые два свойства следуют из линейности кольца многочленов
по модулю. Третье и четвертое свойства очевидно следуют из определения операции \ref{def1}.

Таким образом, можно воспользоваться теоремой \ref{th1} для разрешения выражения \ref{eqWithT}
относительно $g(L)$. В качестве $f(L)$ выступает $b_L$, а в качестве $g(d)$ необходимо
взять $d \cdot g(d)$, внеся множитель $d$ внутрь операции $T_d$, согласно свойству один:
\[
  L \cdot g(L) = \sum_{\substack{d|L}}\mu(d)T_d(b_{L/d})
\]

\[
  g(L) = \frac{\sum_{\substack{d|L}}\mu(d)T_d(b_{L/d})}{L}
\]

Напомним, что $g(L)$ содержит многочлен, коэффициент при $D^w$ которого равен числу циклов длины и периода $L$ веса $w$.
Обозначим результирующую величину количества циклов длины $L$ веса 0 с учетом эквивалентности относительно сдвига за $C_L$.
Для подсчета величины $C_L$ с помощью $g(L)$ необходимо произвести суммирование по всем возможным длинам периодов $d$ в циклах
длины $L$ и весам $w$ таким, что период повторенный $\frac{L}{d}$ раз приведет к циклу нулевого веса:
\[
  C_L=\sum_{\substack{d|L}}\sum_{\substack{\frac{w \cdot L}{d} = 0 \pmod{M}}}g^{(w)}(0)
\]
где $g^{(w)}$ -- $w$-ая производная $g$, используемая для получения коэффициента при $D^w$.

Наконец для устранения дубликатов относительно инверсии достаточно разделить $C_L$ на два.

\section{Алгоритм}
\begin{eqnarray}
a_0^i=(\underbrace{0,0,...,0}_{i \text{ раз}},1,0,...,0) \label{aeq1} \\
a_{2L}^i=a_0^i \cdot A^L \label{aeq2} \\
b_{2L}=\sum_{\substack{i}}a_{2L}^i \label{aeq3} \\
  g(L) = \frac{\sum_{\substack{d|L}}\mu(d)T_d(b_{L/d})}{L} \label{aeq4} \\
  C_L=\sum_{\substack{d|L}}\sum_{\substack{\frac{w \cdot L}{d} = 0 \pmod{M}}}g^{(w)}(0) \label{aeq5} \
\end{eqnarray}

\section{Оценка сложности}

 Пусть исходная базовая  матрица имела размер $b \times c$, а коэффициент расширения равен $M$.
Для простоты будем рассматривать $(J,K)$-регулярный МППЧ код с 
весом столбцов и строк $J$ и $K$ соответственно. 

Тогда число ребер в графе Таннера обозначим за $E=b \cdot K = c \cdot J$.
Ограничение на максимальную длину в спектре обозначим $S$.

Таким образом размер матрицы $A$ составляет $E \times E$, каждый из элементов которой представляет
собой многочлен степени не больше $M$. И для получения
всех интересующих $A^L$, посредством перемножения матриц,
достаточно затратить $O(S \cdot E^3 \cdot M)$ операций, так как перемножение и сложение многочленов по модулю $D^M$
требует $O(M)$ времени в отличии от $O(1)$ для обычных чисел.

При каждом фиксированном стартовом ребре $i$ и длине циклов $2L$, многочлен $a_{2L}^i=a_0^i \cdot A^L$ \ref{aeq2}
может быть получен за время $E^2 \cdot M$, посредством перемножения вектора $a_0$ длины $E$ на матрицу
$A^L$ размера $E \times E$, опять же по причине того что перемножение и сложение многочленов по модулю 
требует $O(M)$ операций. Таким образом суммарно получение всех необходимых $a_{2L}$ займет
 время $O(E \cdot S \cdot E^2 \cdot M) = O(S \cdot E^3 \cdot M)$.

Сложение всех необходимых $a_{2L}$ для получения $b_{2L}$ (\ref{aeq3}) может быть осуществлено за суммарный размер 
полиномов $a$, а именно $O(E \cdot S \cdot M)$.

Все необходимые значения функции Мебиуса $\mu(d)$ могут быть подсчитаны за время $O(S \cdot \ln S)$ с
помощью решета Эратосфена. При оценке времени суммарно затраченного на подсчет $g(L)$ заметим,
 что суммирование ведется по всем делителям чисел $L$ от $1$ до $S$. Как известно суммарное количество
делителей чисел от $1$ до $n$ имеет порядок $O(n \cdot \ln n)$. Действительно, число $d$ является 
делителем для $\floor*{\frac{n}{d}}$ чисел: $d, 2d, 3d, \ldots, \floor*{\frac{n}{d}} \cdot d$. Получаем сумму гармонического ряда
 $\sum_d \floor*{\frac{n}{d}} = O(n \cdot \ln n)$. Таким образом для подсчета всех $g(L)$ необходимо затратить $O(M \cdot S \cdot \ln S)$,
так как операция $T_d$ и сложение многочленов степени $M$ может быть осуществлено за $O(M)$.

Пользуясь оценкой суммы гармонического ряда для \ref{aeq5} с учетом суммирования по весам получаем
время необходимое для подсчета $C_L$ -- $O(M \cdot S \cdot \ln S)$, так как теперь складываются 
коэффициенты многочлена, а не целые многочлены.

Итого по всем шагам алгоритма получаем:
\[
O(S \cdot E^3 \cdot M) + O(S \cdot E^3 \cdot M) + 
O(E \cdot S \cdot M) + O(M \cdot S \cdot \ln S) + O(M \cdot S \cdot \ln S)= 
\]
\[
O(S \cdot E^3 \cdot M) + O(M \cdot S \cdot \ln S) = O(M \cdot S \cdot max(\ln S, E^3))
\]
Очевидно для всех разумных входных данных член $E^3$ мажорирует $\ln S$ таким образом итоговая сложность:
\[
  O(M \cdot S \cdot E^3)
\]

\chapter{Численный анализ влияния спектров циклов на вероятность ошибки 
БП-декодирования} 

\section{Описание ансамблей кодов}

Генерация случайных матриц заданного размера с фиксированным числом единиц в строках и столбцах для
задания регулярного МППЧ-кода может производиться различными способами. При проведении
тестирования были рассмотрены следующие ансамбли кодов.

\subsection{Ансамбль Галлагера}

Матрицы в ансамбле Галлагера состоят из полос с фиксированным числом строк в каждой. Каждый столбец
полосы содержит ровно одну единицу. Таким образом число полос равно весу столбца.

Например, рассмотрим (3,6)-код, $M=4$. Такой код состоит из 6 полос, каждая из которых состоит
из $M=4$ строк. Первая строка имеет вид
\setcounter{MaxMatrixCols}{30}
\[
\begin{pmatrix}
1 & 1 & 1 & 1 & 1 & 1 & 0 & 0 & 0 & 0 & 0 & 0 & 0 & 0 & 0 & 0 & 0 & 0 & 0 & 0 & 0 & 0 & 0 & 0 & 0
\end{pmatrix}
\]

Остальные $M-1$ строк этой полосы --- сдвиги первой строки на 6 позиций. Первая полоса построена.
Оставшиеся 2 полосы --- случайные перестановки первой полосы.

В результате получен (24,12)-код, он же (3,6)-регулярный МППЧ-код.

\subsection{Ансамбль Ричардсона-Урбанке}



\subsection{Ансамбль квазициклических кодов}

\section{Описание эксперимента}

- несколько (4-6) графиков из Матлаба или  TikZ

- выводы по графикам

\startconclusionpage

\begin{enumerate}
	\item Получены экспериментальные доказательства прямой зависимости между количеством коротких циклов и вероятностью ошибки на блок для различных ансамблей кодов
	\item Разработан вычислительно эффективный алгоритм подсчета спектра графа Таннера, который может быть использован для ускорения отбора кодов.
\end{enumerate}


%% Обратите внимание на heading. Без него тоже работает, но название будет другим.
\printmainbibliography

\end{document}

%%% Local Variables:
%%% mode: latex
%%% TeX-master: t
%%% End:
