\documentclass[specification,annotation]{itmo-student-thesis}

%% Опции пакета:
%% - specification - если есть, генерируется задание, иначе не генерируется
%% - annotation - если есть, генерируется аннотация, иначе не генерируется
%% - times - делает все шрифтом Times New Roman, требует пакета pscyr.

%% Делает запятую в формулах более интеллектуальной, например:
%% $1,5x$ будет читаться как полтора икса, а не один запятая пять иксов.
%% Однако если написать $1, 5x$, то все будет как прежде.
\usepackage{icomma}

%% Данные пакеты необязательны к использованию в бакалаврских/магистерских
%% Они нужны для иллюстративных целей
%% Начало
\usepackage{tikz}
\usetikzlibrary{arrows}
%% Конец

%% Указываем файл с библиографией.
\addbibresource{master-thesis.bib}

\begin{document}

\studygroup{M4239}
\title{Исследование зависимости вероятности ошибки на блок от спектра графа Таннера для МППЧ-кодов}
\author{Ковшаров Антон Павлович}{Ковшаров А.П.}
\supervisor{Буздалов Максим Викторович}{Буздалов М.В.}{канд. техн. наук}{научный сотрудник Университета ИТМО}
\publishyear{2017}
%% Дата выдачи задания. Можно не указывать, тогда надо будет заполнить от руки.
\startdate{01}{сентября}{2015}
%% Срок сдачи студентом работы. Можно не указывать, тогда надо будет заполнить от руки.
\finishdate{31}{мая}{2017}
%% Дата защиты. Можно не указывать, тогда надо будет заполнить от руки.
\defencedate{15}{июня}{2017}

\addconsultant{Кудряшов Б.Д.}{докт. техн. наук, профессор}
\addconsultant{Бочарова И.Е.}{канд. техн. наук, доцент}

%% Задание
%%% Техническое задание и исходные данные к работе
\technicalspec{В рамках работы требуется исследовать зависимость между спектром графа Таннера МППЧ-кода и его
 эффективностью при декодировании. Для проведения исследования необходимо разработать 
 эффективный алгоритм вычисления спектра, позволяющий провести отбор кодов с хорошим спектром среди широкого спектра
 сгенерированных кодов. Также необходимо разработать итеративный декодер, позволяющий измерить эффективность кода 
 посредством симуляции передачи кодовых слов через канал с шумом.}

%%% Содержание выпускной квалификационной работы (перечень подлежащих разработке вопросов)
\plannedcontents{\begin{enumerate}
	\item Обоснование важности установления зависимости между спектром и эффективностью для исследования
	и отбора МППЧ-кодов;
	\item Разработка и реализация итеративного декодера для быстрой оценки эффективности кода посредством
	моделирования;
	\item Разработка и реализация алгоритма подсчета спектра графа Таннера;
	\item Описание плана исследования. Порядок отбора кодов для проведения тестирования;
	\item Результаты исследования.
\end{enumerate}}

%%% Исходные материалы и пособия 
\plannedsources{\begin{enumerate}
    \item Б.Д.Кудряшов. Основы теории кодирования;
    \item М.Холл. Комбинаторика;
    \item D.J.C.MacKay. Encyclopedia of Sparse Graph Codes. http://www.inference.phy.cam.ac.uk/mackay/codes/data.html.
\end{enumerate}}

%%% Календарный план
\addstage{Ознакомление с основами теории кодирования}{12.2015}
\addstage{Ознакомление с имеющимся набором программ для исследования и отбора МППЧ-кодов}{05.2016}
\addstage{Ознакомление с существующими итеративными декодерами}{07.2016}
\addstage{Разработка и реализация итеративного декодера заточенного под нужды исследования}{09.2016}
\addstage{Ознакомление с существующими подходами подсчета спектра кода}{11.2016}
\addstage{Разработка и реализация алгоритма подсчета спектра графа Таннера МППЧ-кода}{12.2016}
\addstage{Проведение исследования зависимости эффективности кода от спектра}{03.2017}
\addstage{Написание пояснительной записки}{05.2017}

%%% Цель исследования
\researchaim{Установить существование критерия оценки эффективности МППЧ-кода, основанного
 на спектре графа Таннера.}

%%% Задачи, решаемые в ВКР
\researchtargets{\begin{enumerate}
    \item разработка и реализация эффективного алгоритма вычисления спектра графа Таннера;
    \item проведение исследования зависимости эффективности кода от спектра графа Таннера;
    \item формулировка критерия оценки эффективности МППЧ-кода.
\end{enumerate}}

%%% Использование современных пакетов компьютерных программ и технологий
\advancedtechnologyusage{C++, CUDA С --- для создания итеративного декодера. Python и zsh скрипты для
автоматизация исследования. Java --- алгоритм подсчета спектра. Python, matplotlib, pandas ---
обработка и визуализация результатов. \LaTeX, Git.}

%%% Краткая характеристика полученных результатов 
\researchsummary{В результате была продемонстрирована зависимость между спектром графа Таннера и
эффективностью кода. Разработан вычислительно эффективный алгоритм подсчета спектра графа Таннера.
 Результаты могут быть использованы для ускорения поиска эффективных МППЧ-кодов.}

%%% Гранты, полученные при выполнении работы 
\researchfunding{Грантов или других форм государственной поддержи и субсидирования
 в процессе работы не предусматривалось.}

%%% Наличие публикаций и выступлений на конференциях по теме выпускной работы
\researchpublications{
\begin{refsection}
\nocite{kmu-russian,kmu-english}
\printannobibliography
\end{refsection}
}

%% Эта команда генерирует титульный лист и аннотацию.
\maketitle{Магистр}

%% Оглавление
\tableofcontents

%% Макрос для введения. Совместим со старым стилевиком.
\startprefacepage

В данном разделе размещается введение.

%% Начало содержательной части.
\chapter{Первая глава}

Пример ссылок на литературные источники: \cite{hall-combinatorics,kudryashov-codingtheory,finding-and-counting-given-length-cycles,counting-short-cycles-of-quasi-cyclic-protograph-ldpc-codes,message-passing-algorithm-for-counting-short-cycles-in-graph,how-to-find-long-paths-efficiently,color-coding,algorithm-for-counting-for-counting-short-cycles-in-bipartite-graphs,opencl-cuda-algorithms-for-parallel-decoding-of-any-irregular-ldpc-code-using-gpu,stressing-the-ber-simulation-of-ldpc-codes-in-the-error-floor-region-using-gpu-clusters,on-the-number-of-cycles-in-a-graph,understanding-belief-propogation,mackay-codes}.

\startconclusionpage

В данном разделе размещается заключение

%% Обратите внимание на heading. Без него тоже работает, но название будет другим.
\printmainbibliography

\end{document}

%%% Local Variables:
%%% mode: latex
%%% TeX-master: t
%%% End:
