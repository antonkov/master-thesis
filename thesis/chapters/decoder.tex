\chapter{Декодер}

Пример ссылок на литературные источники: \cite{hall-combinatorics,kudryashov-codingtheory,finding-and-counting-given-length-cycles,counting-short-cycles-of-quasi-cyclic-protograph-ldpc-codes,message-passing-algorithm-for-counting-short-cycles-in-graph,how-to-find-long-paths-efficiently,color-coding,algorithm-for-counting-for-counting-short-cycles-in-bipartite-graphs,on-the-number-of-cycles-in-a-graph,understanding-belief-propogation,mackay-codes}.


Проведение экспериментального исследования зависимости эффективности итеративного декодирования от некоторого
критерия подразумевает непосредственное моделирование передачи информации через канал с шумом посредством 
кодирования информации определенным МППЧ-кодом и последующим декодированием с помощью итеративного декодера.

Критерий, основанный на спектре циклов графа Таннера, побуждает к проведению исследования на иррегулярных
МППЧ-кодах. Еще одним фактором не в пользу скорости декодирования является размер кода --- который
должен быть выбран в соотвествии с длинами кодов, используемыми на практике, которые достаточно велики.
Кроме того, количество переданных кодовых слов при моделировании должны быть достаточно большим, чтобы оценка
отношения ошибочно-переданных блоком к общему числу переданных блоков (FER - frame error rate) была достаточно точна.

Моделирование передачи достаточного количества кодовых слов занимает минуты на современных 
компьютерах, используя наивную реализацию декодера на CPU или реализацию из доступных библиотек 
(таких как например реализация из Communication System Toolbox в Matlab). 
С учетом того, что каждый код должен быть протестирован на десятках различных отношений сигнал-шум (SNR ---
signal noise ratio), подход с использованием наивной реализации при имеющихся вычислительных ресурсах займет
недопустимо большое количество времени. 

Альтернативным подходом являются реализации итеративных декодеров с использованием
GPU.
Многие из доступных реализаций заточены под МППЧ-коды из стандартов. Большая часть использует параллелизм
на уровне декодирования одного кодового слова, что позволяет ускорить время между получением 
битовой последовательности и декодированного кодового слова
\cite{stressing-the-ber-simulation-of-ldpc-codes-in-the-error-floor-region-using-gpu-clusters}, 
но не использует тот факт что при оценке вероятности ошибки на блок передается большое количество кодовых
слов. В данном случае значительно более существенный выигрыш позволяет получить параллелизм
на уровне набора кодовых слов, где вычислительные узлы разделены на группы, каждая из которых занимается
декодированием различных кодовых слов
\cite{opencl-cuda-algorithms-for-parallel-decoding-of-any-irregular-ldpc-code-using-gpu}.

К сожалению, итеративного GPU декодера, оптимизированного для получения оценки вероятности ошибки на блок,
не было найдено в открытых источниках сети Internet. Что побудило к созданию простого итеративного декодера
с использованием архитектуры CUDA, который позволил получить 60-70 кратное ускорение при моделировании
передачи большого числа кодовых слов иррегулярного МППЧ-кода.

\section{Количество кодовых слов при моделировании}

В первую очередь следует провести анализ количества кодовых слов, которое следует передать, чтобы 
статистически достоверно оценить вероятность ошибки на блок.

Пусть $P$ --- вероятность ошибки на блок и передача производилась до достижения $k$ ошибочно переданных блоков.
Тогда суммарно передано $N\approx \frac{k}{P}$ кодовых слов. Оценим дисперсию среднего арифметического при 
$N$ опытах. $X_i$ --- индикаторная случайная величина ошибки при передаче $i$-го кодового слова. 
\newcommand{\Expect}{\mathsf{E}}
\[
	\begin{split}
	\Variance \left({\frac{\sum_i{X_i}}{N}}\right) & =
	\Expect \left(\frac{\sum_i{X_i}}{N}\right)^2-\left(\Expect \left(\frac{\sum_i{X_i}}{N}\right)\right)^2 \\
	& = \frac{\Expect(\sum_i{X_i})^2}{N^2})-\left(\frac{\sum_i{\Expect(X_i)}}{N}\right)^2 \\
	& = \frac{\Expect(\sum_i{X_i})^2}{N^2})-\left(\frac{N \cdot P}{N}\right)^2 \\
	& = \frac{\sum_{i \neq j}{\Expect(X_i \cdot X_j)} + \sum_i{\Expect(X_i^2)}}{N^2}-P^2 \\
	& = \frac{N \cdot (N - 1) \cdot P^2 + N \cdot P}{N^2}-P^2 \\
	& = \frac{-N \cdot P^2 + N \cdot P}{N^2} \\
	& = \frac{P \cdot (1 - P)}{N}  = \frac{P^2 \cdot (1 - P)}{k}
	\end{split}
\]

Применяя Гауссовскую аппрокимацию биномиального распределения, можем говорить что пределы за $3\cdot \sigma$
маловероятны (вероятность порядка 0.001). Поэтому погрешность находится в пределах (учитывая что в реальности
$P \ll 1$):
\[
\pm \delta = 3 \cdot P \sqrt{\frac{1 - P}{k}} \approx P \cdot \frac{3}{\sqrt{k}} 
\]	 

Соответственно, например при $k=50$ получаем относительную погрешность $\approx 42\%$, 
а при $k=100$ --- $\approx 30\%$. При увеличении $k$ погрешность убывает обратно пропорционально корню.
Для проведения дальнейших исследований остановимся на $k=100$.

\section{Построение порождающей матрицы по проверочной}

Многие исследователи при моделировании передачи кодовых слов через канал останавливаются на многократной отправке
нулевого кодового слова и его последующем декодировании. Однако, при таком подходе оценка эффективности
МППЧ-кода оказывается существенно завышена относительно передачи различных кодовых слов при практическом использовании
кода.

Для генерации большого числа различных кодовых слов необходимо иметь представление кода в виде
 порождающей матрицы. Имея представление $(n,k)$ кода в виде проверочной матрицы $H$ размера $r \times n$ и
 ранга $p=n-k$, порождающую матрицу кода $G$ можно получить следующим образом:
 \begin{enumerate}
 	\item С помощью двух проходов алгоритма Гаусса преобразуем матрицу $H$ к матрице $J$ того же размера,
 	 которая выглядит как:
 	\[
 		J = \begin{pmatrix}
 			I_p & 0 \\
 			0 & 0
 		\end{pmatrix}
 	\] 
 	Действительно, с помощью первого прохода получаем матрицу с нулями ниже главной диагонали,
 	 записав линейное преобразование как матрицу $P$ размера $r \times r$:
 	\[
 		H_2 = P \cdot H = \begin{pmatrix}
 			1 & x_{1,2} & x_{1,3} & \ldots & x_{1,p} & x_{1,p+1} & \ldots & x_{1,n} \\
 			0 & 1 & x_{2,3} & \ldots & x_{2,p} & x_{2,p+1} & \ldots & x_{2,n} \\
 			\ldots & \ldots & \ldots & \ldots & \ldots & \ldots & \ldots & \ldots \\
 			0 & 0 & 0 & \ldots & 1 & x_{p,p+1} & \ldots & x_{p,n} \\ 
 			0 & 0 & 0 & \ldots & 0 & 0 & \ldots & 0 \\ 
 			\ldots & \ldots & \ldots & \ldots & \ldots & \ldots & \ldots & \ldots \\
 			0 & 0 & 0 & \ldots & 0 & 0 & \ldots & 0
 		\end{pmatrix}
 	\]
 	Затем с помощью еще одного прохода $H_2^T$ может быть приведен к $J^T$ записав линейное преобразование
 	в матрицу $Q$ размера $n \times n$:
 	\[
 	\begin{split}
 		J^T & = Q \cdot H_2^T \\
 		& = Q \cdot \begin{pmatrix}
 			1 & 0 & \ldots & 0 & 0 & \ldots & 0 \\
 			x_{1,2} & 1 & \ldots & 0 & 0 & \ldots & 0 \\
 			x_{1,3} & x_{2,3} & \ldots & 0 & 0 & \ldots & 0 \\
 			\ldots & \ldots & \ldots & \ldots & \ldots & \ldots & \ldots \\
 			x_{1,p} & x_{2,p} & \ldots & 1 & 0 & \ldots & 0 \\
 			x_{1,p+1} & x_{2,p+1} & \ldots & x_{p,p+1} & 0 & \ldots & 0 \\
 			\ldots & \ldots & \ldots & \ldots & \ldots & \ldots & \ldots \\
 			x_{1,n} & x_{2,n} & \ldots & x_{p,n} & 0 & \ldots & 0
 		\end{pmatrix}  = \begin{pmatrix}
 			I_p & 0 & \ldots & 0 \\
 			0 & 0 & \ldots & 0 \\
 			\ldots & \ldots & \ldots & \ldots \\
 			0 & 0 & \ldots & 0 \\
 		\end{pmatrix}
 	\end{split}
 	\]
 	Итого:
 	\[
 		J = (Q \cdot H_2^T)^T = H_2 \cdot Q^T = P \cdot H \cdot Q^T
 	\]
 	
 	\item Запишем отношение между порождающей и проверочной матрицей:
 	 \[
 	 	G \cdot H^T = 0
 	 \]
 	 или тоже самое:
 	 \[
 	 	H \cdot G^T = 0
 	 \]
 	 Выражая $H$ через $J$:
 	 \[
 	 	P^{-1} \cdot J \cdot (Q^T)^{-1} \cdot G^T = 0
 	 \]
 	 где $P^{-1}$ и $(Q^T)^{-1}$ обратные матрицы для $P$ и $Q^T$ соответственно. Обратные матрицы существуют
 	 так как проходы Гаусса сохраняли ранг матрицы, соответственно преобразование обратимо.
 	 
 	 \item Обозначим $(Q^T)^{-1} \cdot G^T$ за $Y$ ($Y$ имеет размер $n\times k$), тогда:
 	\[
 		J \cdot Y = 0
 	\]
 	откуда следует, что $Y$ имеет форму:
 	\[
 		Y=\begin{pmatrix}
 			0 & 0 & \ldots & 0 \\
 			\ldots & \ldots & \ldots & \ldots \\
 			0 & 0 & \ldots & 0 \\
 			y_{p+1,1} & y_{p+1,2} & \ldots & y_{p+1,k} \\
 			\ldots & \ldots & \ldots & \ldots \\
 			y_{n,1} & y_{n,2} & \ldots & y_{n,k} \\
 		\end{pmatrix}
 	\]
 	а
 	\[
 	G^T = Q^T \cdot Y
 	\]
 	
 	Выбирая различные коэффициенты $y_{i,j}$ в $G$ будут получатся различные наборы кодовых слов,
 	не все из которых будут являться базисами. Для получения базисного набора векторов в $G$, необходимо
 	и достаточно чтобы ранг $G$ был равен $k$, так как $\det Q \neq 0$ соответственно
 	 ранг $Y$ тоже должен быть равен $k$.
 	
 	Одним из подходящих вариантов $Y$ является следующий, ранг которого равен, 
 	очевидно, $k$ (так как $n - k = p$):
 	\[
 		Y=\begin{pmatrix}
 			0 \\
 			I_k
 		\end{pmatrix}
 	\]
 	
 	Итого:
 	\[
 	G=\begin{pmatrix}
 		0 & I_k
 	\end{pmatrix} \cdot Q
 	\] 
 \end{enumerate}

\section{Алгоритм декодирования}

\section{Реализация алгоритма декодирования на GPU}

\section{Подсчет эффективности кода}

Стандартный способ оценки эффективности кода --- построения
графика вероятности ошибки на блок от отношения сигнал-шум.

При фиксированной величине отношения сигнал-шум для оценки вероятности
ошибки на блок необходимо отсылать кодовые слова до достижения некоторого порогового
значения количества ошибок на блок --- в нашем случае $k=100$, как было обсуждено ранее.
В силу того, что разработанный декодер обрабатывает несколько полученных
кодовых последовательностей одновременно, декодирование будет производиться
блоками по $D=100$ кодовых слов (наиболее оптимальное количество кодовых слов в блоке для 
одновременного декодирования установлено экспериментальным путем).
Проверка достижения порогового количества ошибок будет происходить после
обработки каждого блока.

Таким образом при фиксированном значении отношения сигнал-шум подсчет эффективности
производится следующим образом:

\begin{enumerate}
	\item Вычислить порождающую матрицу $G$ по проверочной матрице $H$.
	\item Сгенерировать достаточное количество кодовых слов для подсчета эффективности.
	Было решено генерировать $D=100$ различных кодовых слов, то есть количество равное числу слов в
	блоке. Таким образом набор кодовых слов в блоках будет одинаковым, но переданные последовательности
	будут разными за счет различных шумовых данных.
	Каждое случайно сгенерированное кодовое слово является линейной комбинацией базисных кодовых
	слов, где каждое базисное кодовое слово взято с вероятностью $\frac{1}{2}$. Генератор случайных 
	чисел инициализируется одинаковым случайным значением, соответственно подсчет эффективности
	проводится на одинаковой последовательности кодовых слов для фиксированного кода на разных
	отношениях сигнал-шум.
	\item Сгенерировать шум и прибавить шумовые данные к кодовым словам. Исходя из определения
	отношения сигнал-шум:
	\[
		SNR(dB)=10\log_{10}\left(\frac{A_{signal}^2}{A_{noise}^2}\right)
	\]
	где $A_{signal},A_{noise}$ --- среднеквадратичное значение амплитуды сигнала и шума соответственно. 
	
	Принимая амплитуды сигнала за $\pm1$ как у Галлагера считаем, что $A_{signal}=1$. Соответственно:
	\[
		A_{noise}^2 = 10^{-\frac{SNR}{10}} 
	\]
	Таким образом для моделирования шума с заданным отношением SNR необходимо сгенерировать
	данные с нормальным распределением со средним равным $0$ и дисперсией $\sigma^2=A_{noise}^2$.
	\item Производить декодирование по блокам с последующим измерением числа ошибок до достижения
	порогового значения количества ошибок.
\end{enumerate}

В данном исследовании нас будет интересовать интервал 
отношения сигнал-шум от 1 дБ до 3.5 дБ, так как большинство
случайных кодов достигают вероятности ошибки на блок $10^{-3}$
для исследуемых размеров ранее 3.5 дБ. Кроме того заметим,
что нет необходимости измерять вероятность ошибки
для кода при большем отношении
сигнал-шум, если при текущем отношении код настолько хорош,
что ошибка оказывается менее некоторой пороговой, например
$10^{-3}$. Более того это еще и очень долго,
так как вероятность ошибки при увеличении отношения сигнал-шум
может уменьшаться экспоненциально.

Шаг между отношениями сигнал шум выбран равным $0.1$,
опять же из соображений того, что вероятность может убывать
очень быстро и при большом шаге будет настолько мала, что
достижение достаточного числа ошибок займет существенное время.
Использование более малого шага нецелесообразно, так как
вероятность уменьшается не так существенно.

Иногда представление эффективности кода в виде графика зависимости
от отношения сигнал-шум неудобно, так как в таком представлении различные
коды могут быть несравнимы. Значение отношения сигнал-шум $S_0$ при котором вероятность
ошибки на блок впервые становится менее некоторого заданного значения, например $10^{-3}$,
дает гораздо более компактное представление об эффективности кода и может быть использовано
для сравнения эффективности различных кодов.

Подсчет величины $S_0$ можно организовать на основе построения графика эффективности
кода от отношения сигнал-шум до достижения необходимой вероятности, однако подсчет
может быть организован гораздо эффективнее.

Заметим, что вероятность ошибки монотонно убывает при росте отношения сигнал-шум,
соответственно можно использовать бинарный поиск по отношению сигнал-шум для нахождения
$S_0$. Однако при данном подходе подсчет вероятности при значениях больше $S_0$ может
занимать существенное время, так как мы знаем что вероятность ошибки может убывать экспоненциально.
Для экономии времени при обработке слишком больших отношении сигнал-шум ограничим количество
кодовых слов для отправки. 

Обозначим текущее отношение сигнал-шум за $S$, вероятность ошибки
при $S_0$ за $FER_{target}$ (где FER --- Frame Error Rate), полученное количество ошибок на блок
за $k$, пороговое значение количества ошибок
на блок за $k_0$, количество переданных слов за $N$. Тогда нет необходимости пересылать
более $N$ кодовых слов:
\[
	N = \frac{2 \cdot k_0}{FER_{target}}
\]

Оценим вероятность ошибочного определения положения $S$ относительно $S_0$.

Действительно, при условии что $S > S_0$ ограничение на количество переданных кодовых слов
при недостижении заданного порога количества ошибок только увеличивает вероятность правильного определения
предиката нахождения $S$ относительно $S_0$. 

При условии $S < S_0$ и передаче $N$ кодовых слов 
матожидание $k$ не менее $2 \cdot k_0$. Вспомним, что при $k_0=100$ среднеквадратичное отклонение 
составляет $\sigma=0.1$ 
и соответственно вероятность, что $k > 2 \cdot k_0 - 3 \cdot \sigma \cdot 2 \cdot k_0 = 1.4 \cdot k_0$
 более $0.999$, а вероятность, что $k > 2 \cdot k_0 - 5 \cdot \sigma \cdot 2 \cdot k_0 = k_0$ более $0.999999997$.
Таким образом вероятность, протестировав $N$ слов, получить менее $k_0$ ошибок при отношении сигнал-шум
$S < S_0$ составляет менее $3 \cdot 10^{-9}$, что достаточно мало.

В итоге для нахождения $S_0$ с помощью бинарного поиска достаточно порядка $10$ итераций при начальных
границах от 1 дБ до 5 дБ.




















































































