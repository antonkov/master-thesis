% -*-coding: utf-8-*-
\startprefacepage

Основной целью данной работы является разработка эффективного метода построения маршрутов поиска
объектов согласно общепринятым стратегиям поиска. Метод должен работать с большим количеством
различных моделей изменения распределения вероятности обнаружения объекта со временем. %Цель
 
 Существующие подходы строят весь маршрут исходя из данных в начальный момент
времени --- информации о начальном распределении и модели его изменения.
Таким образом главным недостатком существующих подходов является необходимость
разработки нового алгоритма для каждой модели изменения распределения.
Учитывая тот факт, что подобрать правильную модель, хорошо приближающую реальность,
 крайне непросто, возникает необходимость разработки алгоритма,
 работающего единообразно на широком классе различных моделей.
 Основная идея рассматриваемого подхода --- использование симулятора для получения %Новизна
 информации о распределении в любой момент времени при построении маршрута.
 Таким образом при планировании пользователь в первую очередь выберет модель,
 как можно лучше приближающую реальность в данном случае, а после запустит
 алгоритм построения маршрута.

 Таким образом задачами данной работы являются разработка инструмента симуляции изменения
вероятности во время прохождения маршрута и алгоритма построения маршрута при заданных условиях.
В данной работе будет рассмотрен алгоритм построения маршрута согласно
 стратегии ``Параллельное галсирование''.

Расчетная задача может быть использована оператором как вспомогательное средство
при планировании маршрутов поиска потерпевших бедствие объектов в реальной жизни.
Визуализированный процесс изменения распределения может пригодиться при решении какие области
следует исследовать более подробно, а какие можно пропустить.
%Практическое значение

Концепция данной задачи был продемонстрирована на одной из выставок. К задаче был проявлен 
интерес, и было решено внедрить ее в комплекс расчетных морских задач. %Актуальность

В главе 1 решаемая задача будет рассмотрена более подробно.
 Описаны классы маршрутов, получаемые при использовании стратегии ``Параллельное галсирование''. 
Будут приведены особенности задачи, которые отличают ее от классической задачи коммивояжера
 и делают невозможным использование ранее разработанных методов решения
 для решения исходной задачи в общем случае.

В главе 2 будут рассмотрены вопросы, связанные с разработкой симулятора на CUDA.
 Обозначены предоставляемые симулятором сервисы.

В главе 3 будет описан алгоритм построения маршрутов согласно стратегии ``Параллельное галсирование''.
В первую очередь будет рассмотрен подход с динамическим программирования для определения
маршрута при фиксированном распределении. Далее будет рассмотрена корректировка маршрута
согласно изменениям распределения.

В главе 4 будет осуществлено сравнение с существующим алгоритмом ``Кронштадт технологии''.
\FloatBarrier
