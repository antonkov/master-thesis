\startprefacepage

Коды с малой плотностью проверок на четность (МППЧ-коды) в последнее время стали одной из
самых популярных конструкций для построения эффективных кодов. В первую очередь из-за 
простого декодирования по принципу распространения доверия \cite{understanding-belief-propogation}, которое
хорошо поддается распараллеливанию. Таким образом, МППЧ-коды приходится рассматривать только вместе с
алгоритмом декодирования.

Однако, использование декодирования по принципу распространения доверия приводит к трудности оценки
эффективности определенного кода. В случае большинства кодов главной характеристикой оценки эффективности
является минимальное расстояние, ведь именно оно определяет сколько ошибок может быть исправлено кодом.
В случае МППЧ-кода --- это не так, так как минимальное расстояние кода не говорит о том сколько ошибок может быть исправлено алгоритмом декодирования и лучше ли код с большим минимальным расстоянием
чем код у которого это расстояние меньше.

Удобным представлением для анализа и описания МППЧ-кодов, в частности записи алгоритма декодирования
как алгоритма из семейства алгоритмов передачи сообщений, является граф Таннера \cite{kudryashov-codingtheory}.

Исходя из конструкции алгоритма декодирования, интуиция подсказывает, что большой обхват и 
малое число коротких циклов в графе Таннера положительно сказываются на эффективности 
итеративного декодирования. Утверждение о том, что большая величина обхвата 
(минимальная длина цикла в графе) соответствует более эффективным кодам является общеизвестным фактом \cite{kudryashov-codingtheory}. Хочется предложить обобщенный критерий оценки
эффективности, основанный не только на величине обхвата, но и на количестве циклов различных длин (спектре)
в графе Таннера.

Существует множество подходов подсчета циклов определенной длины в графе
\cite{finding-and-counting-given-length-cycles,on-the-number-of-cycles-in-a-graph}.
Однако большинство алгоритмов подсчета циклов имеют ad-hoc структуру и способны находить 
только количество циклов
с малой длиной. Нахождение циклов с большей длиной является вычислительно сложной задачей
\cite{how-to-find-long-paths-efficiently,color-coding}. 
Например, задача существования гамильтонова цикла --- является подзадачей нахождения числа циклов
определенной длины. Некоторые из подходов ограничиваются нахождением циклов длины менее двух обхватов 
\cite{message-passing-algorithm-for-counting-short-cycles-in-graph,counting-short-cycles-of-quasi-cyclic-protograph-ldpc-codes}.

В рассмотренном подходе, предлагается производить подсчет схожего циклам комбинаторного объекта,
который равен числу циклов при длине до двух обхватов в графе, но может быть посчитан гораздо эффективнее.
Новый подход имеет простое описание и низкую вычислительную сложность.

Посчитанный спектр предлагается использовать в качестве критерия поиска хороших МППЧ-кодов.
В \cite{algorithm-for-counting-for-counting-short-cycles-in-bipartite-graphs} были приведены два примера:
в одном из которых код, имеющий меньше циклов минимальной длины значительно превосходит второй код в
эффективности, а во
втором код имеющий больший обхват оказывается существенно хуже кода с меньшим, за счет регулярности
расположения циклов. Разумеется данный критерий не может быть использован безусловно
в отрыве от остальных критериев. В данной работе будет выдвинута и доказана гипотеза, что в среднем согласно
критерию лексикографического сравнения спектров, код с меньшим спектром оказывается эффективнее кода с большим.

Существуют и другие критерии для поиска хороших МППЧ кодов, например, ACE \cite{ace}. 
В любом случае, все сводится к анализу структуры циклов графа. 
Поиск эффективных МППЧ-кодов -- вычислительно сложная задача.
Наша задача --- посредством разработки различных критериев оценки эффективности ускорить поиск. 
В данной работе предложен новый критерий оценки эффективности, обусловлена его состоятельность и предложен
алгоритм для его подсчета.









































