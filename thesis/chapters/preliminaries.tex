\chapter{Общие сведения}

\section{Линейные коды}

\definition{\textit{Линейным двоичным $(n,k)$---кодом} называется любое $k$---мерное подпространство пространства
всевозможных двоичных векторов длины $n$.}

\definition{Отношение $R=k/n$ называется \textit{скоростью линейного $(n,k)$ кода}.}

\definition{\textit{Порождающей матрицей линейного $(n,k)$---кода} называется матрица 
размера $k\times n$, строки которой его базисные вектора.}

\example{
\[
G=
\begin{pmatrix}
	1 & 0 & 1 \\
	1 & 0 & 0
\end{pmatrix}
\]

задает код ${000, 101, 100, 001}$.
}

\definition{Двоичный вектор $\bs{h}$ длины $n$ для которого все кодовые слова некоторого $(n,k)$ кода
 $C=\{\bs{c}_i\},i=0,\ldots,2^k-1$ удовлетворяют тождеству
\[
(\bs{c}_i,\bs{h})=0, i=0,\ldots,2^k-1
\]
называется \textit{проверкой} по отношению к коду $C$.}

Заметим, что предыдущее определение проверки эквивалентно
\[
G \cdot \bs{h}^T=0
\]
так как для выполнения тождества для любого кодового слова достаточно выполнения тождества для 
базисных векторов.

\definition{Пространство проверок называется пространством, ортогональным линейному коду, или \textit{проверочным
пространством}.}

\theorem{Размерность проверочного пространства линейного $(n,k)$---кода равна $r=n-k$.}

\definition{\textit{Проверочной матрицей линейного $(n,k$)---кода} называется матрица размера $r\times n$,
строки которой составляют базис проверочного пространства}

Для проверочной и порождающей матрицы выполнено следующее соотношение
\[
G \cdot H^T=0
\]

Если же принятая последовательность $\bs{y}$ из-за шума в канале перестала быть кодовым
словом то соответсвенно
\[
\bs{s}=\bs{y}\cdot H^T \neq 0
\]
и \bs{s} называется \textit{синдромом} вектора \bs{y}, неравенство нулевому вектору которого
указывает на ошибки в принятой последовательности \bs{y}.

\section{МППЧ-коды}

Линейный код может быть задан проверочной матрицей $H$.

Галлагер \cite{gallager} предложил идею выбора матрицы $H$ разряженной (с малой плотностью) для уменьшения
 сложности кодирования и декодирования: в матрице должно быть мало единиц, строки и столбцы не должны 
 иметь большое число  общих элементов. Он также подкрепил свою идею анализом с использованием 
 метода случайного декодирования.
 
 Для интуитивного понимания почему малое число единиц приводит к более эффективному декодированию следует
 заметить, что в случае когда строки проверок мало между собой зависят, декодирование может производиться 
 методом проб и ошибок, пытаясь подобрать последовательность символов, исправление которых будет уменьшать
 вес синдрома, с каждой следующей попыткой.
 
 Используемый алгоритм декодирования, описанный далее, существенно опирается на факт, что влияние конкретного
 столбца на синдром не сильно зависит от остальных столбцов.
 


\definition{МППЧ-код называется \textit{$(J,K)$ регулярным} если его проверочная матрица $H$ содержит ровно
$J$ единиц в каждом столбце и  ровно $K$ единиц в каждой строке. Иначе МППЧ-код называется \textit{иррегулярным}.}

\section{Граф Таннера}

\newcommand\colorBox[2]{\setlength{\fboxsep}{2pt}\colorbox{#1!10}{#2}}

\begin{figure}[h!]
\centering
\begin{subfigure}{.3\textwidth}
  \centering
  \[
    \begin{blockarray}{ccccc}
        \colorBox{red}{1} & \colorBox{red}{2} & \colorBox{red}{3} & \colorBox{red}{4} \\
        \begin{block}{(cccc)c}
            1&1&0&1&\colorBox{green}{5}\\
            1&1&1&0&\colorBox{green}{6}\\
            1&0&1&1&\colorBox{green}{7} \\
        \end{block}
    \end{blockarray}
  \]
  \caption{Проверочная матрица $H$}
  \label{checkMatrix}
\end{subfigure}%
\begin{subfigure}{.7\textwidth}
  \centering
   \scalebox{1.2}{\inputTikZ{ex_graph1}}
  \caption{Граф Таннера}
  \label{checkMatrix}
\end{subfigure}
\caption{Пример графа Таннера}
\end{figure}

МППЧ-коды представляют интерес только в совокупности с алгоритмом декодирования --- 
алгоритмом распространения доверия. Основной характеристикой эффективности линейного кода является
минимальное  расстояние. Из-за используемого алгоритма декодирования минимальное расстояние МППЧ-кода
не может служить критерием сравнения эффективности различных кодов.

Еще Галлагер заметил связь МППЧ-кодов с циклами определенного графа, после чего большая работа в этом
направлении была проделана Таннером.

\definition{(Определение 11.2 из книги \cite{kudryashov-codingtheory}) Графом Таннера линейного кода, заданного проверочной матрицей 
$H=\{h_{ij}\}$; $i=1,\ldots,r$; $j=1,\ldots,n$ называется двудольный граф, одно множество узлов которого
соответствует множеству проверочных соотношений (проверочные узлы), другое --- множеству кодовых символов
(символьные узлы). 
Символьный узел $c_i$ \tikz{\filldraw[thick,draw=black!50,fill=red!10] circle(1ex);} 
связан ребром с некоторым
 проверочным узлом $s_j$ \tikz{\filldraw[thick,draw=black!50,fill=green!10] rectangle(2ex,2ex);},
  если $i$-й
кодовый символ входит в $j$-ю проверку, иными словами, если элемент $h_{ij}$ 
проверочной матрицы не равен нулю.}

\section{Квазициклические МППЧ-коды}

$P^i_M$ --- квадратная матрица порядка $M$, полученная из единичной сдвигом строк вправо $i$ раз. Например:
\[
P^0_3=
\begin{pmatrix}
	1 & 0 & 0 \\
	0 & 1 & 0 \\
	0 & 0 & 1
\end{pmatrix}
\quad
P^1_3=
\begin{pmatrix}
	0 & 1 & 0 \\
	0 & 0 & 1 \\
	1 & 0 & 0
\end{pmatrix}
\quad
P^2_3=
\begin{pmatrix}
	0 & 0 & 1 \\
	1 & 0 & 0 \\
	0 & 1 & 0
\end{pmatrix}
\] 

Матрицы $P^i_M$ содержат одинаковые элементы на всех диагоналях параллельных главной --- такие матрицы
называются \textit{циркулянтными матрицами} или \textit{циркулянтами}.

\definition{МППЧ-код называется квазициклическим если для некоторого натурального числа $M$ циклический сдвиг
любого кодового слова на $M$ также является кодовым словом.}

С помощью перестановки строк и столбцов порождающая и проверочная матрицы квазициклического МППЧ-кода
могут быть представлены в виде блочной матрицы состоящей из нулевых матриц и матриц циркулянтов размера
$M \times M$.

\begin{equation} \label{exlift}
	H_B=\begin{pmatrix}
		1 & 0 & 0 \\
		1 & 0 & 1 \\
		0 & 1 & 0
	\end{pmatrix}
	\quad
	H=\begin{pmatrix}
		P^{\mu_{11}}_M & 0 & 0 \\
		P^{\mu_{21}}_M & 0 & P^{\mu_{23}}_M \\
		0 & P^{\mu_{32}}_M & 0
	\end{pmatrix}
\end{equation}

Таким образом квазициклический код с проверочной матрицей размера $(rM,nM)$ может быть построен из проверочной
матрицы размера $(r,n)$ меньшего кода с помощью подстановки вместо элементов базовой матрицы нулевых блоков
 и блоков вида $P^i$ (см. рис. \ref{exlift}). 
 
  Если рассматривать графы Таннера, соответствующие этим кодам --- такой метод построения
 большего графа называется методом \textit{лифтинга} или \textit{расширения}. Граф соответствующий 
 исходному коду называется \textit{базовым графом} или \textit{протогорафом}.

Пусть $G_B=\{V_B, E_B\}$ --- базовый граф с множеством вершин $\{V_B\}$ и ребер $\{E_B\}$.
$\Gamma=\{\gamma\}$ --- множество вычетов по модулю положительного целого числа $M$, который также 
называется \textit{коэффициентом расширения}. Сопоставим каждому ребру базового графа элемент из $\Gamma$.
Тем самым получим \textit{размеченный граф} или \textit{граф напряжений} (voltage graph).

\definition{(Определение 11.4 из книги \cite{kudryashov-codingtheory})
 Расширенным графом называется граф $G=\{E \in E_b \times \Gamma, V_B \times \Gamma\}$,
в котором две вершины $(v,\gamma)$ и $(v',\gamma')$ связаны ребром $e$ в том случае, если вершины
$v$ и $v'$ связаны ребром в базовом графе $G_B=\{E_B,V_B\}$ и этому ребру присвоена метка (напряжение)
$\gamma-\gamma'$}

\begin{figure}[h!]
\centering
\begin{subfigure}{0.2\textwidth}
  \centering
  \scalebox{1.2}{\inputTikZ{liftingBase}}
  \caption{Протограф}
  \label{liftingBase}
\end{subfigure}%
\begin{subfigure}{.8\textwidth}
  \centering
  \scalebox{1.2}{\inputTikZ{liftingExpanded}}
  \caption{Расширенный граф}
  \label{liftingExpanded}
\end{subfigure}
\caption{Пример лифтинга}
\label{exliftgraph}
\end{figure}

На рис. \ref{exliftgraph} представлен пример протографа и расширенного графа для матриц из выражения \ref{exlift2}.

\begin{equation} \label{exlift2}
	H_B=\begin{pmatrix}
		1 & 1 & 1 \\
		1 & 1 & 1
	\end{pmatrix}
	\quad
	H=\begin{pmatrix}
		P^0_7 & P^0_7 & P^0_7 \\
		P^0_7 & P^1_7 & P^3_7
	\end{pmatrix}
\end{equation}

\section{Описание ансамблей МППЧ-кодов}

Генерация случайных матриц заданного размера с фиксированным числом единиц в строках и столбцах для
задания регулярного МППЧ-кода может производиться различными способами. В данной работе были рассмотрены следующие ансамбли кодов.

\subsection{Ансамбль Галлагера}

Матрицы в ансамбле Галлагера состоят из полос с фиксированным числом строк в каждой. Каждый столбец
полосы содержит ровно одну единицу. Таким образом число полос равно весу столбца.

Например, рассмотрим (3,6)-код, $M=4$. Такой код состоит из 6 полос, каждая из которых состоит
из $M=4$ строк. Первая строка имеет вид
\setcounter{MaxMatrixCols}{30}
\[
\begin{pmatrix}
1 & 1 & 1 & 1 & 1 & 1 & 0 & 0 & 0 & 0 & 0 & 0 & 0 & 0 & 0 & 0 & 0 & 0 & 0 & 0 & 0 & 0 & 0 & 0 & 0
\end{pmatrix}
\]

Остальные $M-1$ строк этой полосы --- сдвиги первой строки на 6 позиций. Таким образом строится первая
полоса. Оставшиеся 2 полосы --- случайные перестановки первой полосы.

В результате получен (24,12)-код, он же (3,6)-регулярный МППЧ-код.

\subsection{Ансамбль Ричардсона-Урбанке}

В ансамбле Ричардсона-Урбанке все (3,6)-регулярные коды равновероятные.

Рассмотрим способ на примере того же (24,12)-кода.
Возьмем последовательность (номера строк единиц):
\[
\begin{pmatrix}
	1 & 1 & 1 & 1 & 1 & 1 & 2 & 2 & 2 & 2 & 2 & 2 & 3 & 3 & 3 & 3 & 3 & 3 & ... & 12 & 12 & 12 & 12 & 12 & 12
\end{pmatrix}
\]

Возьмем случайную перестановку чисел этой последовательности.
Берем первые $J=3$ числа, скажем 7, 3, 11. Они указывают номера строк единиц
первого столбца. Берем еще 3 числа и находим второй столбец и так далее.

Этот способ не гарантирует регулярность кода, так как на одной позиции может оказаться две единицы
и таким образом столбец будет содержать меньше чем $J$ строк с единицами. Это оказалось существенно при
проведении экспериментов. Матрицы в которых столбец содержит меньше $J$ строк с единицами
более разряженные и соответственно их граф Таннера содержит меньше циклов. Таким образом такие матрицы
становятся лучшими согласно спектру, но вполне ожидаемо показывают довольно плохие результаты при моделировании.

Таким образом был рассмотрен немного модифицированный ансамбль Ричардсона-Урбанке, который контролировал
регулярность кода, отбрасывая иррегулярные МППЧ-коды. Так как регулярные коды составляют значительную
часть кодов из ансамбля Ричардсона-Урбанке, данная модификация незначительно увеличивает время генерации.
\subsection{Ансамбль квазициклических кодов}

Построение кодов из ансамбля квазициклических кодов осуществляется с помощью той же разметки
некоторой <<базовой>> матрицы и последующей замены элементов матрицы блоками циркулянтами, как было
описано в разделе о квазициклических МППЧ-кодах.

В качестве базовой матрицы берется матрица из всех единиц, например матрица размера $(3,6)$,
после чего к ней применяется случайная разметка числами $\mu_{ij}$ из кольца по модулю $M$,
 где $M$ коэффициент
расширения:

\[
	H_1 = \begin{pmatrix}
		1 & 1 & 1 & 1 & 1 & 1 \\
		1 & 1 & 1 & 1 & 1 & 1 \\
		1 & 1 & 1 & 1 & 1 & 1
	\end{pmatrix}
	\quad
	\begin{pmatrix}
		\mu_{11} & \mu_{12} & \mu_{13} & \mu_{14} & \mu_{15} & \mu_{16} \\
		\mu_{21} & \mu_{22} & \mu_{23} & \mu_{24} & \mu_{25} & \mu_{26} \\
		\mu_{31} & \mu_{32} & \mu_{33} & \mu_{34} & \mu_{35} & \mu_{36}
	\end{pmatrix}
\]

После элементы $H_1$ заменяются на соответствующие блоки циркулянты и получается матрица из ансамбля
квазициклических кодов $H_B$:

\[
H_B=\begin{pmatrix}
		P^{\mu_{11}}_M & P^{\mu_{12}}_M & P^{\mu_{13}}_M & P^{\mu_{14}}_M & P^{\mu_{15}}_M & P^{\mu_{16}}_M \\
		P^{\mu_{21}}_M & P^{\mu_{22}}_M & P^{\mu_{23}}_M & P^{\mu_{24}}_M & P^{\mu_{25}}_M & P^{\mu_{26}}_M \\
		P^{\mu_{31}}_M & P^{\mu_{32}}_M & P^{\mu_{33}}_M & P^{\mu_{34}}_M & P^{\mu_{35}}_M & P^{\mu_{36}}_M
	\end{pmatrix}
\]

Таким образом при построении матриц заданного размера из ансамбля квазициклических кодов 
процедура разметки и лифтинга
применяется дважды: для создания базовой матрицы и матрицы заданного размера из базовой.

















































