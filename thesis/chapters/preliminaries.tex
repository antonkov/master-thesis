\chapter{Общие сведения}

\section{Линейные коды}

\definition{\textit{Линейным двоичным $(n,k)$---кодом} называется любое $k$---мерное подпространство пространства
всевозможных двоичных векторов длины $n$.}

\definition{Отношение $R=k/n$ называется \textit{скоростью линейного $(n,k)$ кода}.}

\definition{\textit{Порождающей матрицей линейного $(n,k)$---кода} называется матрица 
размера $k\times n$, строки которой его базисные вектора.}

\example{
\[
G=
\begin{pmatrix}
	1 & 0 & 1 \\
	1 & 0 & 0
\end{pmatrix}
\]

задает код ${000, 101, 100, 001}$.
}

\definition{Двоичный вектор $\bs{h}$ длины $n$ для которого все кодовые слова некоторого $(n,k)$ кода
 $C=\{\bs{c}_i\},i=0,\ldots,2^k-1$ удовлетворяют тождеству
\[
(\bs{c}_i,\bs{h})=0, i=0,\ldots,2^k-1
\]
называется \textit{проверкой} по отношению к коду $C$.}

Заметим, что предыдущее определение проверки эквивалентно
\[
G \cdot \bs{h}^T=0
\]
так как для выполнения тождества для любого кодового слова достаточно выполнения тождества для 
базисных векторов.

\definition{Пространство проверок называется пространством, ортогональным линейному коду, или \textit{проверочным
пространством}.}

\theorem{Размерность проверочного пространства линейного $(n,k)$---кода равна $r=n-k$.}

\definition{\textit{Проверочной матрицей линейного $(n,k$)---кода} называется матрица размера $r\times n$,
строки которой составляют базис проверочного пространства}

Для проверочной и порождающей матрицы выполнено следующее соотношение
\[
G \cdot H^T=0
\]

Если же принятая последовательность $\bs{y}$ из-за шума в канале перестала быть кодовым
словом то соответсвенно
\[
\bs{s}=\bs{y}\cdot H^T \neq 0
\]
и \bs{s} называется \textit{синдромом} вектора \bs{y}, неравенство нулевому вектору которого
указывает на ошибки в принятой последовательности \bs{y}.

\section{МППЧ-коды}

Линейный код может быть задан проверочной матрицей $H$.

Галлагер \cite{gallager} предложил идею выбора матрицы $H$ разряженной (с малой плотностью) для уменьшения
 сложности кодирования и декодирования: в матрице должно быть мало единиц, строки и столбцы не должны 
 иметь большое число  общих элементов. Он также подкрепил свою идею анализом с использованием 
 метода случайного декодирования.
 
 Для интуитивного понимания почему малое число единиц приводит к более эффективному декодированию следует
 заметить, что в случае когда строки проверок мало между собой зависят, декодирование может производиться 
 методом проб и ошибок, пытаясь подобрать последовательность символов, исправление которых будет уменьшать
 вес синдрома, с каждой следующей попыткой.
 
 Используемый алгоритм декодирования, описанный далее, существенно опирается на факт, что влияние конкретного
 столбца на синдром не сильно зависит от остальных столбцов.
 


\definition{МППЧ-код называется \textit{$(J,K)$ регулярным} если его проверочная матрица $H$ содержит ровно
$J$ единиц в каждом столбце и  ровно $K$ единиц в каждой строке. Иначе МППЧ-код называется \textit{иррегулярным}.}

\section{Квазициклические МППЧ-коды}

$P^i_M$ --- квадратная матрица порядка $M$, полученная из единичной сдвигом строк вправо $i$ раз. Например:
\[
P^0_3=
\begin{pmatrix}
	1 & 0 & 0 \\
	0 & 1 & 0 \\
	0 & 0 & 1
\end{pmatrix}
\quad
P^1_3=
\begin{pmatrix}
	0 & 1 & 0 \\
	0 & 0 & 1 \\
	1 & 0 & 0
\end{pmatrix}
\quad
P^2_3=
\begin{pmatrix}
	0 & 0 & 1 \\
	1 & 0 & 0 \\
	0 & 1 & 0
\end{pmatrix}
\] 

Матрицы $P^i_M$ содержат одинаковые элементы на всех диагоналях параллельных главной --- такие матрицы
называются \textit{циркулянтными матрицами} или \textit{циркулянтами}.

МППЧ-код называется квазициклическим если его проверочная матрица может быть представлена в виде блочной
матрицы из блоков циркулянтов единичной матрицы.

Для построения б

\begin{figure}[h!]
\centering
\begin{subfigure}{0.2\textwidth}
  \centering
  \scalebox{1.2}{\inputTikZ{liftingBase}}
  \caption{Протограф}
  \label{liftingBase}
\end{subfigure}%
\begin{subfigure}{.8\textwidth}
  \centering
  \scalebox{1.2}{\inputTikZ{liftingExpanded}}
  \caption{Расширенный граф}
  \label{liftingExpanded}
\end{subfigure}
\caption{Пример лифтинга}
\end{figure}

\section{Граф Таннера}


\newcommand\colorBox[2]{\setlength{\fboxsep}{2pt}\colorbox{#1!10}{#2}}

\begin{figure}[h!]
\centering
\begin{subfigure}{.3\textwidth}
  \centering
  \[
    \begin{blockarray}{ccccc}
        \colorBox{red}{1} & \colorBox{red}{2} & \colorBox{red}{3} & \colorBox{red}{4} \\
        \begin{block}{(cccc)c}
            1&1&0&1&\colorBox{green}{5}\\
            1&1&1&0&\colorBox{green}{6}\\
            1&0&1&1&\colorBox{green}{7} \\
        \end{block}
    \end{blockarray}
  \]
  \caption{Проверочная матрица $H$}
  \label{checkMatrix}
\end{subfigure}%
\begin{subfigure}{.7\textwidth}
  \centering
   \scalebox{1.2}{\inputTikZ{ex_graph1}}
  \caption{Граф Таннера}
  \label{checkMatrix}
\end{subfigure}
\caption{Пример графа Таннера}
\end{figure}






















































