%-*-coding: utf-8-*-
\chapter{Постановка задачи}
\label{chapSVD}

\section{Задача построения маршрута поиска в oбщeм случае}


\subsection{Расширения задачи коммивояжера}
Соответственно выделяют вершины в которых более вероятно обнаружить объект.
 Сопоставим вершине $v$ величину $p_v$ --- вероятность обнаружить объект в этой вершине.
 $p_{path}=\sum\limits_{v\in path}p_v$. На практике длина путей с $p_{path} \ge 0.99$ может превышать
длину путей с $p_{path} \ge 0.9$ в десятки раз. То есть длина пути может расти экспоненциально
в зависимости от $p_{path}$. Следовательно необходимым параметром задачи становится максимальная длина
пути (или время поиска с физической точки зрения). Известно обобщение Profit Based TSP \cite{dew13}:
каждой вершине сопоставляется значение $p_v$, при посещении вершины к сумме призов 
добавляется $(p_v-t_v)$, где $t_v$---время посещения, необходимо составить маршрут с
наибольшей суммой призов. К сожалению наша задача и здесь сравнительно более общая, так как
величины призов могут изменяться нелинейно.}

\FloatBarrier
%%% Local Variables:
%%% mode: latex
%%% TeX-master: "../thesis"
%%% End:
