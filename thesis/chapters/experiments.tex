\chapter{Численный анализ влияния спектров циклов на вероятность ошибки 
БП-декодирования} 

\section{Описание ансамблей кодов}

Генерация случайных матриц заданного размера с фиксированным числом единиц в строках и столбцах для
задания регулярного МППЧ-кода может производиться различными способами. При проведении
тестирования были рассмотрены следующие ансамбли кодов.

\subsection{Ансамбль Галлагера}

Матрицы в ансамбле Галлагера состоят из полос с фиксированным числом строк в каждой. Каждый столбец
полосы содержит ровно одну единицу. Таким образом число полос равно весу столбца.

Например, рассмотрим (3,6)-код, $M=4$. Такой код состоит из 6 полос, каждая из которых состоит
из $M=4$ строк. Первая строка имеет вид
\setcounter{MaxMatrixCols}{30}
\[
\begin{pmatrix}
1 & 1 & 1 & 1 & 1 & 1 & 0 & 0 & 0 & 0 & 0 & 0 & 0 & 0 & 0 & 0 & 0 & 0 & 0 & 0 & 0 & 0 & 0 & 0 & 0
\end{pmatrix}
\]

Остальные $M-1$ строк этой полосы --- сдвиги первой строки на 6 позиций. Первая полоса построена.
Оставшиеся 2 полосы --- случайные перестановки первой полосы.

В результате получен (24,12)-код, он же (3,6)-регулярный МППЧ-код.

\subsection{Ансамбль Ричардсона-Урбанке}



\subsection{Ансамбль квазициклических кодов}

\section{Описание эксперимента}

- несколько (4-6) графиков из Матлаба или  TikZ

- выводы по графикам
