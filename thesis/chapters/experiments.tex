\chapter{Численный анализ влияния спектров циклов на вероятность ошибки 
декодирования} 

Общий подход к генерации случайного регулярного МППЧ-кода состоит из следующих шагов:
\begin{enumerate}
	\item Выбор весов столбцов и строк базовой матрицы (J,K). В исследовании будут рассмотрены
	(3,6) и (4,8)-регулярные МППЧ-коды, как наиболее применяемые на практике.
	\item Выбор размера базовой матрицы $(n_b,r_b)$. Размер должен удовлетворять соотношению 
	$J \cdot n_b = K \cdot r_b$ (два способа подсчитать число единиц в матрице --- по строкам и по столбцам).
	В исследовании рассмотрен размер (24,12), который удовлетворяет (3,6) и (4,8) с коэффициентам 4 и 3
	соответственно.
	\item Выбор базовой матрицы из фиксированного ансамбля кодов. В исследовании рассмотрены ансамбли
	Галлагера, Ричардсона-Урбанке и Квазициклических кодов.
	\item Выбор размера результирующего кода после лифтинга. В исследовании рассмотрены размеры
	576 и 2304, используемые в стандарте 802.16e WiMAX. Коэффициенты лифтинга $M=24$ и $M=96$
	соответственно.
	\item Разметка базовой матрицы весами по модулю $M$. Все разметки приняты равновероятными.
\end{enumerate}

\section{Описание ансамблей кодов}

Генерация случайных матриц заданного размера с фиксированным числом единиц в строках и столбцах для
задания регулярного МППЧ-кода может производиться различными способами. При проведении
тестирования были рассмотрены следующие ансамбли кодов.

\subsection{Ансамбль Галлагера}

Матрицы в ансамбле Галлагера состоят из полос с фиксированным числом строк в каждой. Каждый столбец
полосы содержит ровно одну единицу. Таким образом число полос равно весу столбца.

Например, рассмотрим (3,6)-код, $M=4$. Такой код состоит из 6 полос, каждая из которых состоит
из $M=4$ строк. Первая строка имеет вид
\setcounter{MaxMatrixCols}{30}
\[
\begin{pmatrix}
1 & 1 & 1 & 1 & 1 & 1 & 0 & 0 & 0 & 0 & 0 & 0 & 0 & 0 & 0 & 0 & 0 & 0 & 0 & 0 & 0 & 0 & 0 & 0 & 0
\end{pmatrix}
\]

Остальные $M-1$ строк этой полосы --- сдвиги первой строки на 6 позиций. Таким образом строится первая
полоса. Оставшиеся 2 полосы --- случайные перестановки первой полосы.

В результате получен (24,12)-код, он же (3,6)-регулярный МППЧ-код.

\subsection{Ансамбль Ричардсона-Урбанке}

В ансамбле Ричардсона-Урбанке все (3,6)-регулярные коды равновероятные.

Рассмотрим способ на примере того же (24,12)-кода.
Возьмем последовательность (номера строк единиц):
\[
\begin{pmatrix}
	1 & 1 & 1 & 1 & 1 & 1 & 2 & 2 & 2 & 2 & 2 & 2 & 3 & 3 & 3 & 3 & 3 & 3 & ... & 12 & 12 & 12 & 12 & 12 & 12
\end{pmatrix}
\]

Возьмем случайную перестановку чисел этой последовательности.
Берем первые $J=3$ числа, скажем $7,3,11$. Они указывают номера строк единиц
первого столбца. Берем еще 3 числа и находим второй столбец и так далее.

Этот способ не гарантирует регулярность кода, так как на одной позиции может оказаться две единицы
и таким образом столбец будет содержать меньше чем $J$ строк с единицами. Это оказалось существенно при
проведении экспериментов. Матрицы в которых столбец содержит меньше $J$ строк с единицами
более разряженные и соответственно их граф Таннера содержит меньше циклов. Таким образом такие матрицы
становятся лучшими согласно спектру, но вполне ожидаемо показывают довольно плохие результаты при моделировании.

Таким образом был рассмотрен немного модифицированный ансамбль Ричардсона-Урбанке, который контролировал
регулярность кода, отбрасывая иррегулярные МППЧ-коды. Так как регулярные коды составляют значительную
часть кодов из ансамбля Ричардсона-Урбанке, данная модификация незначительно увеличивает время генерации.
\subsection{Ансамбль квазициклических кодов}

Построение кодов из ансамбля квазициклических кодов осуществляется с помощью той же разметки
некоторой "базовой" матрицы и последующей замены элементов матрицы блоками циркулянтами, как было
описано в разделе о квазициклических МППЧ-кодах.

В качестве базовой матрицы берется матрица из всех единиц, например матрица размера $(3,6)$,
после чего к ней применяется случайная разметка числами $\mu_{ij}$ из кольца по модулю $M$,
 где $M$ коэффициент
расширения:

\[
	H_1 = \begin{pmatrix}
		1 & 1 & 1 & 1 & 1 & 1 \\
		1 & 1 & 1 & 1 & 1 & 1 \\
		1 & 1 & 1 & 1 & 1 & 1
	\end{pmatrix}
	\quad
	\begin{pmatrix}
		\mu_{11} & \mu_{12} & \mu_{13} & \mu_{14} & \mu_{15} & \mu_{16} \\
		\mu_{21} & \mu_{22} & \mu_{23} & \mu_{24} & \mu_{25} & \mu_{26} \\
		\mu_{31} & \mu_{32} & \mu_{33} & \mu_{34} & \mu_{35} & \mu_{36}
	\end{pmatrix}
\]

После элементы $H_1$ заменяются на соответствующие блоки циркулянты и получается матрица из ансамбля
квазициклических кодов $H_B$:

\[
H_B=\begin{pmatrix}
		P^{\mu_{11}}_M & P^{\mu_{12}}_M & P^{\mu_{13}}_M & P^{\mu_{14}}_M & P^{\mu_{15}}_M & P^{\mu_{16}}_M \\
		P^{\mu_{21}}_M & P^{\mu_{22}}_M & P^{\mu_{23}}_M & P^{\mu_{24}}_M & P^{\mu_{25}}_M & P^{\mu_{26}}_M \\
		P^{\mu_{31}}_M & P^{\mu_{32}}_M & P^{\mu_{33}}_M & P^{\mu_{34}}_M & P^{\mu_{35}}_M & P^{\mu_{36}}_M
	\end{pmatrix}
\]

Таким образом при построении матриц заданного размера из ансамбля квазициклических кодов 
процедура разметки и лифтинга
применяется дважды: для создания базовой матрицы и матрицы заданного размера из базовой.

\section{Описание эксперимента}

После исследования некоторых гипотез было решено остановиться на одной из самых простых, которая состоит в
следующем:
\begin{conjecture}
	Код со спектром циклов графа Таннера лексикографически меньшим спектра циклов графа Таннера другого
	кода статистически более эффективен, имеет меньшую вероятность ошибки на блок при фиксированном отношении
	сигнал-шум. Более формально --- достигает вероятности ошибки на блок $10^{-3}$ при меньшем отношении
	сигнал-шум.
\end{conjecture}

Стандартный подход построения графика вероятности ошибки на блок от отношения
сигнал-шум для кода дает наглядное представление о эффективности кода, но не может быть формально использовано
для сравнения эффективности двух кодов.

Для сравнения эффективности вводится следующая величина $F$: отношение сигнал-шум при котором кодом
впервые достигается вероятность ошибки на блок не менее $10^{-3}$. Такой способ сравнения был использован,
например, в \cite{kudryashov-huawei}.

В первую очередь проведем статистическую проверку гипотезы без фиксированного префикса спектра циклов
графа Таннера, после чего проведем проверку гипотезы в более строгих условиях --- нескольких зафиксированных
первых значениях спектра.

\newcommand{\plotstandard}[2]{
\centerline{\includegraphics[height=3in]{#1}}
\captionof{figure}{#2}
}

\newcommand{\plotsmall}[1]{\includegraphics[height=3.2in]{#1}}

\subsection{Проверка гипотезы без ограничений на спектр}

Исследование было проведено на трех различных ансамблях Ричардсона-Урбанке, Галлагера и ансамбле
квазициклических кодов, двух фиксированных наборов весов базовых матриц (3,6) и (4,8), размере
базовой матрицы $24 \times 12$ и двух размерах матриц после лифинга с параметром $M=24$ --- $576 \times 288$
и $M=96$ --- $1157 \times 2304$.

Каждым способом было сгенерировано $10^6$ базовых матриц, после чего был произведен отбор трех с лучшим
спектром, трех с худшим и десяти случайных. Базовые матрицы с лучшим и худшим спектром были размечены
случайным образом десять раз. Случайные матрицы были случайно размечены один раз. После чего был
построен график вероятности ошибки от отношения сигнал-шум, представленные ниже.

Каждая линия на графике представляет один код. Линие представленные зеленым цветом соответствуют
кодам с лучшим спектром базовой матрицы гипотезе. Линии представленные красным цветом соответствуют кодам с
худшим спектром базовой матрице согласно гипотезе. Синие линии представляют собой случайно сгенерированные
базовые матрицы, для сравнения с кодами отобранными по гипотезе.

\begin{figure}[h!]
\centering
\begin{subfigure}{.5\textwidth}
  \centering
  \plotsmall{../images/r4_576.pdf}
  \caption{Ричардсон-Урбанке 4x8 576}
\end{subfigure}%
\begin{subfigure}{.5\textwidth}
  \centering
  \plotsmall{../images/r4_2304.pdf}
  \caption{Ричардсон-Урбанке 4x8 2304}
\end{subfigure}


\begin{subfigure}{.5\textwidth}
  \centering
  \plotsmall{../images/g4_576.pdf}
  \caption{Галлагер 4x8 576}
\end{subfigure}%
\begin{subfigure}{.5\textwidth}
  \centering
  \plotsmall{../images/q4_576.pdf}
  \caption{Квазициклические коды 4x8 576}
\end{subfigure}

\caption{Графики вероятности ошибки для кодов веса $4 \times 8$}
\end{figure}

\begin{figure}[h!]
\centering
\begin{subfigure}{.5\textwidth}
  \centering
  \plotsmall{../images/g3_576.pdf}
  \caption{Галлагер 3x6 576}
\end{subfigure}%
\begin{subfigure}{.5\textwidth}
  \centering
  \plotsmall{../images/g3_2304.pdf}
  \caption{Галлагер 3x6 2304}
\end{subfigure}

\begin{subfigure}{.5\textwidth}
  \centering
  \plotsmall{../images/q3_2304.pdf}
  \caption{Квазициклические коды 3x6 2304}
\end{subfigure}%
\begin{subfigure}{.5\textwidth}
  \centering
  \plotsmall{../images/q3_576.pdf}
  \caption{Квазициклические коды 3x6 576}
\end{subfigure}

\caption{Графики вероятности ошибки для кодов веса $3 \times 6$}
\end{figure}

Визуально на графиках становится понятно, что коды с лучшим спектром имеют меньшую вероятность ошибки
на блок чем случайные или коды с худшим спектром, а случайные коды ведут себя лучше, чем коды с
худшим спектром.

Проверим формально с помощью критерия Стьюдента влияние спектра на эффективность декодирования.
В качестве случайной величины будем использовать $F$ --- отношение сигнал-шум для достижения вероятности
$10^{-3}$. В качестве нулевой гипотезы $H_0$ предположим, что спектр не влияет на эффективность декодирования
--- тогда отвергнув нулевую гипотезу, исходя из того, что в среднем коды с лучшим спектром эффективнее
кодов с худшим, можно принять как альтернативную гипотезу $H_1$ --- выдвинутую ранее гипотезу. Также
можем допустить, что выборка случайная, однородная, независимая и имеет распределение близкое к нормальному.
В силу того, что дисперсии заранее неизвестны и могут быть различны для различных кодов, будем
использовать критерий Кохрена-Кокса для взвешенной оценки степени свобод распределения Стьюдента.
\pagebreak

Таким образом для опровержения нулевой гипотезы при сравнении двух выборочных средних необходимо
показать, что $t > t'_{\alpha/2}$, где 
\[
	t=\frac{\overline{x}-\overline{y}}{s}
\]
--- статистика критерия
\[
 s^2=\frac{1}{m}s_x^2+\frac{1}{n}s_y^2,
	s_x^2=\frac{1}{m-1}\sum_{i=1}^{m}(x_i-\overline{x})^2,s_y^2=\frac{1}{n-1}\sum_{i=1}^{n}(y_i-\overline{y})^2
\]
--- выборочные дисперсии
\[
\overline{x}=\frac{1}{m}\sum_{i=1}^{m}x_i, \overline{y}=\frac{1}{n}\sum_{i=1}^{n}y_i
\]
--- выборочные средние. $x_i$, $y_i$ --- значения $F$ для кодов с худшим и лучшим спектром соответственно.
\[
	t_{\alpha}=\frac{\nu_x t_{\alpha}(m-1)+\nu_y t_{\alpha}(n-1)}{\nu_x+\nu_y}
\]
--- критерий Коэрена-Кокса
и $\nu_x=\frac{s_x^2}{m}$, $\nu_y=\frac{s_y^2}{n}$, $t_{\alpha}(f)$ --- распределение Стьюдента с $f$
степенями свободы.

Приведем для справки значения $t'_{\alpha/2}$ для некоторых уровней значимости при фиксированных
размерах выборки $n=30, m=30$.

\begin{tabular}{|c|c|}
	\hline
	$\alpha$ & $t'_{\alpha/2}$ \\
	\hline
	0.0001 & 4.50 \\
	\hline
	0.001 & 3.65 \\
	\hline
	0.01 & 2.75 \\
	\hline
	0.05 & 2.04 \\
	\hline
	0.1 & 1.69 \\
	\hline
\end{tabular}

Теперь приведем таблицу значений статистик для различных ансамблей кодов.

\begin{tabular}{|c|c|c|}
	\hline
	Ансамбль кодов & $t$ & Уровень значимости $\alpha$\\
	\hline
	Ричардсон-Урбанке 4x8 576 & 4.50 &  $\alpha = 0.0001$\\
	\hline
	Ричардсон-Урбанке 4x8 2304 & 29.7 & $\alpha < 0.0001$ \\
	\hline
	Галлагер 4x8 576 & 2.57 & $\alpha \approx 0.01$ \\
	\hline
	Квазициклические коды 4x8 576 & 8.53 & $\alpha < 0.0001$ \\
	\hline
	Галлагер 3x6 576 & 6.57 & $\alpha < 0.0001$ \\
	\hline
	Галлагер 3x6 2304 & 3.32 & $\alpha \approx 0.005$ \\
	\hline
	Квазициклические коды 3x6 2304 & 9.92 & $\alpha < 0.0001$ \\
	\hline
	Квазициклические коды 3x6 576 & 12.92 & $\alpha < 0.0001$ \\
	\hline
\end{tabular}

Таким образом из таблицы видна несостоятельность нулевой гипотезы, что подтверждает выдвинутую алтернативную
гипотезу без ограничений на спектр.

\subsection{Проверка гипотезы с фиксированным префиксом спектра}

Рассмотрим гипотезу при некотором фиксированном префиксе. Например, сгенерируем несколько кодов
спектр которых имеет одинаковый префикс. Так спектр следующих кодов составляет 
$C_4=63, C_6=622, C_8=6961, C_{10}=83856$, но имеет различное количество циклов
длины 12, а именно первый имеет $C_{12}=1068487$, а второй $C_{12}=1070468$.  (Следует отметить, что без обращения Мебиуса и удаления дубликатов, первый из кодов имел бы больше
циклов длины 8, чем второй из кодов и критерий бы не выполнялся.)

\begin{figure}[h!]
\centering
\begin{subfigure}{.5\textwidth}
  \centering
  \plotsmall{../images/d4n1_576.pdf}
  \caption{Размер кода 576}
\end{subfigure}%
\begin{subfigure}{.5\textwidth}
  \centering
  \plotsmall{../images/d4n1_2304.pdf}
  \caption{Размер кода 2304}
\end{subfigure}
\caption{Два различных кода с одинаковым префиксом спектра}
\end{figure}

Критерий Стьюдента для такого набора составляет $t=2.28$ и $t=10.53$ для размеров 576 и 2304 соответственно.
Что дает уровень значимости $\alpha=0.05$ и $\alpha<0.0001$ соответственно.

Однако данное измерение свидетельствует только о том, что коды с заданным фиксированным префиксом
 имеют статистическую зависимость эффективности от $C_{12}$. Для демонстрации зависимости на всем
 ансамбле необходимо брать пары кодов префиксы которых равны между собой, но не обязательно равны
 между парами. 
 
 Некоторым аналогом эксперимента, для развития интуиции о исследовании, может являться эксперимент
 с лечением пациентов некоторым препаратом, когда измерения необходимо производить на каждом пациенте
 до применения и после. Аналогом пациента в данном случае будет являться некоторый фиксированный префикс.
 Измерениями после лечения --- коды с меньшим числом циклов минимальной длины не входящей в фиксированный
 префикс. Измерениями до лечения --- коды с большим числом циклов. Однако так как префикс кодов
 гораздо более сильно влияет на эффективность кодов чем количество текущих циклов --- измерения будут
 сильно отличаться, некоторые коды "до лечения" с хорошим префиксом будут сильно лучше кодов
 "после лечения" с плохим префиксом. Таким образом коды нужно отнормировать. Среднее значение
 величины $F$ по набору кодов с фиксированным префиксом будет смещаться в 0. Для достаточно точного
 нахождения среднего --- каждый набор с фиксированным префиксом будет содержать по крайней мере 10
 кодов. Дисперсию в разных группах будем считать одинаковой.
 
 Для проведения исследования было сгенерировано $10^6$ кодов ансамбля Ричардсона-Урбанке $4 \times 8$.
 Сначала все коды были разделены на группы с одинаковым префиксом спектра длины 1, затем длины 2, 3 и 4.
 Для каждой длины были отобраны 7 групп наибольшего размера, затем в каждой группу выбрано 3 кода с минимальным
 числом циклов минимальной незафиксированной длины и 3 кода с максимальным. Каждый из них
 был 5 раз случайно размечен до размера 2304. В случае длины 4 максимальный размер группы кодов 
 с различным спектром составлял 2, поэтому для каждого из них было сгенерировано по 15 случайных разметок. 

После моделирования, получены следующие значения критерия Стьюдента:

\begin{tabular}{|c|c|c|}
	\hline
	Длина префикса & $t$ & Уровень значимости $\alpha$\\
	\hline
	0 & 29.7 &  $\alpha < 0.0001$\\
	\hline
	1 & 11.47 &  $\alpha < 0.0001$\\
	\hline
	2 & 4.28 & $\alpha < 0.0001$ \\
	\hline
	3 & 3.05 & $\alpha \approx 0.005$ \\
	\hline
	4 & 3.36 & $\alpha = 0.001$ \\
	\hline
\end{tabular}

Таблица наглядно демонстрирует, что можно отвергнуть нулевую гипотезу также в случае когда первое
различие при лексикографическом сравнении спектров выпадает на второе, третье, четвертое или пятое значение.
Причем, явно видно, что зависимость убывает при переходе к большей длине префикса. Исключением
является длина 4, так как не удалось получить достаточное количество кодов в группе с различными 
спектрами, но одинаковым префиксом длины 4, поэтому значение критерия немного смещено.

Таким образом гипотеза о лексикографическом сравнении спектров для оценки эффективности может быть
принята и использована при поиске МППЧ-кодов.



















































