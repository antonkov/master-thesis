\documentclass{article}
\usepackage[utf8]{inputenc}
\usepackage[english]{babel}
\usepackage{amsfonts}
\usepackage{amsmath}
\usepackage{MnSymbol}
\usepackage{wasysym}
\newcommand{\pluseq}{\mathrel{+}=}

\title{Spectrum finder algorithm}

\begin{document}

\maketitle

$T = (V, E)$ --- is Tanner Graph of base matrix $B$.

$M$ is fixed expand modulo.

Edges in path have orientation, so $e^{st}$ --- edge start vertex and $e^{en}$ --- edge end vertex,
$e^{backEdge}$ --- same edge but going in opposite direction.

Closed path is a sequence of edges $e_0, e_1, \ldots, e_l$ so that $e^{en}_i = e^{st}_{i + 1}$ and $e^{en}_l = e^{st}_0$. Additional constraint for our case is $e^{backEdge}_i \neq e_{i+1}$. Different closed paths which can be constructed from one another with cycle shift or reverse are equivalent.

Algorithm finds number of closed path with length from $1$ to $L$.

Let's say that we have edge marking $F: E \rightarrow \mathbb{N}$ so that $0 \leq F(e) < M$ and $e^c = F(e)$.

Number $r$ called \textit{order} of path $p = e_0, \ldots e_l$ if it is maximal number so that $p$ can be 
presented in form $\underbrace{ss\ldots s}_{r \text{ times}}$ where $s$ is subpath.

Algorithm presented below count paths without cycle shift equivalence, e.g,
path of order $1$ and length $l$ will be counted $l$ times, path of order $2$ will be counted $l/2$ times
and so on. To fix it we can use inclusion-exclusion principle, which will be described after algorithm.

\textbf{ALGORITHM WITHOUT CYCLE SHIFT EQUIVALENCE}

Iterate over all possible starting edges $e_{start}=(startVertex,u)$.

$dp[e][l][w]$ is number of paths which starts with $e_{start}$ and ends with edge $e$ which consist of $l$ edges and $\sum_{e \in path} e^c \equiv w \pmod{M}$.

\textbf{Initialization:} 
\[
dp[e_{start}][1][e^c] =1
\]

\textbf{Iteration $l = 2..L$:} 
\[
\forall e \in E\; \forall w \in [0, M)\;  dp[e][l][w] = \sum_{\substack{e_{prev} \neq e^{backEdge} \\ e_{prev}^{en} = e^{st}}} dp[e_{prev}][l - 1][w - e_{prev}^c]
\]

Then save it to fix cycle shifts:

\[
cnt[l] = \sum_{\substack{e \\ e^{en} = startVertex \\ e^{backEdge} \neq e_{start}}} dp[e][l][0]
\]

\textbf{FIXING CYCLE SHIFTS}

We need to count every path of order $r$ exactly $r$ times, so that later we can just divide $cnt[l]$ by $2l$ because every cycle of length $l$ will be counted $2l$ times --- $l$ cycle shifts and $2$ directions.

$cntOrder[r][l]$ --- number of paths of order $r$ and length $l$.
\[
cnt[l] = \sum_{\substack{d | l}} cntOrder[d][l]
\]
and we need
\[
fcnt[l] = \sum_{\substack{d | l}} d \cdot cntOrder[d][l]
\]
Let's define additional function
\[
add[i] = i - \sum_{\substack{d | i \\ d > 1}} add[i / d]  
\]
Then
\[
i = \sum_{\substack{d | i}} add[i]
\]
We'll prove that:
\[
fcnt[l] = \sum_{\substack{d | l}} add[d] \cdot cnt[l / d]
\]
\textbf{Proof:}

\[
c[l] = cntOrder[1][l]
\]
\[
cntOrder[d][l] = c[l / d]
\]

Let's consider $fcnt[l]$ as polynomial of variables $c[l_1]$ and check that every
coefficient equals to what it should be:

\[
fcnt[l] = \sum_{\substack{d_1 | l}} add[d_1] \cdot cnt[l / d_1] = \sum_{\substack{d_1 | l}} add[d_1] \cdot \sum_{\substack{d_2 | (l/d_1)}} c[l/(d_1 d_2)] = \sum_{\substack{d | l}} k_d c[d]
\]
\[
fcnt[l] = \sum_{\substack{d | l}} d \cdot cntOrder[d][l] = \sum_{\substack{d | l}} d \cdot c[l / d]
\]
So we should prove that $k_d = \frac{l}{d}$:
\[
k_d = \sum_{\substack{d_1 | (l/d)}} add[d_1]  = (\frac{l}{d} - \sum_{\substack{d_2 | l/d \\ d_2 > 1}} add[l / (d d_2)])  + \sum_{\substack{d_1 | (l/d) \\ d_1 < l/d}} add[d_1] =
\]
\[
=\frac{l}{d} + (-\sum_{\substack{d_3 | l/d \\ d_3 < l/d}} add[d3]  + \sum_{\substack{d_1 | (l/d) \\ d_1 < l/d}} add[d_1]) = \frac{l}{d} \blacksquare
\]

\textbf{Case with weights:}

Actually it's a little bit more complicated, we need to consider weight of path as well, so that
it will be rewritten as:

\[
cnt[w][l] = \sum_{\substack{e \\ e^{en} = startVertex \\ e^{backEdge} \neq e_{start}}} dp[e][l][w]
\]
\[
cnt[w][l] = \sum_{\substack{d | l}} cntOrder[w][d][l]
\]
\[
fcnt[w][l] = \sum_{\substack{d | l}} d \cdot cntOrder[w][d][l]
\]
We'll prove that:
\[
fcnt[w][l] = \sum_{\substack{d | l \\ w_1 \in [0, M) \\ w_1 \cdot d = w \pmod {M}}} add[d] \cdot cnt[w_1][l / d]
\]
\textbf{Proof:}

\[
c[w][l] = cntOrder[w][1][l]
\]
\[
cntOrder[w][d][l] = \sum_{\substack{w_1 \in [0, M) \\ w_1 \cdot d = w \pmod{M} }} c[w_1][l / d]
\]

Let's consider $fcnt[w][l]$ as polynomial of variables $c[w_1][l_1]$ and check that every
coefficient equals to what it should be:

\[
fcnt[w][l] = \sum_{\substack{d_1 | l \\ w_1 \in [0, M) \\ w_1 \cdot d_1 = w \pmod {M}}} add[d_1] \cdot cnt[w_1][l / d_1] =
\]
\[
\sum_{\substack{d_1 | l}} add[d_1] \cdot \sum{\substack{w_1 \in [0, M) \\ w_1 \cdot d_1 = w \pmod {M}}} \sum_{\substack{d_2 | (l/d_1) \\ w_2 \in [0, M) \\ w_2 \cdot d_2 = w_1 \pmod {M}}} c[w_2][l/(d_1 d_2)] = 
\]
\[
\sum_{\substack{d_1 | l}} add[d_1] \cdot \sum_{\substack{d_2 | (l/d_1) \\ w_3 \in [0, M) \\ w_3 \cdot d_1 \cdot d_2 = w \pmod {M}}} c[w_3][l/(d_1 d_2)] = 
\]
\[
\sum_{\substack{d | l \\ w_1 \in [0, M) \\ w_1 \cdot d = w \pmod {M}}} k^{w_1}_d c[w_1][d]
\]
\[
fcnt[w][l] = \sum_{\substack{d | l}} d \cdot cntOrder[w][d][l] = \sum_{\substack{d | l}} d \cdot c[w][l/d]
\]
So we should prove that $k^w_d = \frac{l}{d}$ and it's exactly the same as without weights:
\[
k^w_d = \sum_{\substack{d_1 | (l/d)}} add[d_1]  = (\frac{l}{d} - \sum_{\substack{d_2 | l/d \\ d_2 > 1}} add[l / (d d_2)])  + \sum_{\substack{d_1 | (l/d) \\ d_1 < l/d}} add[d_1] =
\]
\[
=\frac{l}{d} + (-\sum_{\substack{d_3 | l/d \\ d_3 < l/d}} add[d3]  + \sum_{\substack{d_1 | (l/d) \\ d_1 < l/d}} add[d_1]) = \frac{l}{d}     \blacksquare
\]

Then we add $fcnt[0][l]$ to $spectrum[l]$.

And finally after processing all possible starting edges we divide every $spectrum[l]$ by $2l$
to compress equivalent pathes.


\end{document}

